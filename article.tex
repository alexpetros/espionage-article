\documentclass[14pt]{extarticle}
\usepackage{style}


\title{Draft Article}
\author{Alexander W. Petros}
\date{September 2019}

\addbibresource{article.bib}

\begin{document}
\maketitle
% \newpage

The most damaging cyberattack committed against the United States federal government by a foreign agent was remarkably subtle. Starting sometime in 2014, a small probe embedded itself deep within the network of the US Office of Personnel Management (OPM), an office of about 5,000 employees tasked with overseeing background checks, payroll, and other human resources concerns for the federal government. At regular intervals, the probe would send updates to opmsecurity.org, a domain registered to look like an official website, but not actually affiliated with the US government or the OPM security team. Unobtrusive and observant, the probe functioned in all respects like a digital spy.

By the time the breach was discovered in April 2015, the probe had accessed crucial government databases and transmitted the stolen information back to its country of origin---including a trove of applications for federal security clearance. These applications exposed not just the personal data of the applicants themselves but also the detailed information they had supplied about their family and friends, including social security numbers, job applications, and home addresses. Federal authorities determined that the hack ultimately affected 22.1 million people in total, including personnel at the highest levels of government.\footcite{nakashima_hacks_2015} When OPM announced the full extent of the damage, it caused an uproar. The American Federation of Government Employees, the largest federal workers union, filed a class-action lawsuit against OPM, seeking damages under the Privacy Act.\footcite{chalfant_court_2017} Senator Mark Warner (D-Va.) of the Senate Intelligence Committee called for OPM Director Katherine Archuleta's resignation. Because the information might be used for identity theft, the US government offered all affected employees free credit and identity monitoring services for three years.

Not only was the OPM hack a public relations disaster, it had significant national security implications as well. US officials feared that the stolen records could compromise intelligence efforts by exposing undercover CIA officers working in embassies around the world, whose names would be suspiciously absent from the OPM records. The information could also be used to identify American officials who might be susceptible to pressure, or worse, recruitment. FBI Director James Comey personally took questions about the incident. ``If you have my [application for security clearance],'' Comey said, ``you know every place I've lived since I was 18, contact people at those addresses, neighbors at those addresses, all of my family, every place I've traveled outside the United States. Just imagine if you were a foreign intelligence service and you had that data.''\footcite{nakashima_hacks_2015} With evidence that a foreign intelligence service had successfully obtained exactly that, one might expect that the United States would respond with a series of diplomatic measures intended to signal that highly invasive acts of espionage would not be tolerated from its international adversaries.

Instead, the opposite happened. US officials would only confirm that they believed a \enquote{foreign entity or government} had been behind the attack, even though it was widely known that they suspected the involvement of the Chinese government.\footcite{spetalnick_china_2015} Internal investigators quickly determined that OPM had been compromised by an Advanced Persistent Threat (APT), a formally organized and typically state-sponsored group of hackers that engage in long-term, targeted penetration operations. In this case, the APT made identification easy by leaving their calling-card: the trojan opmsecurity.org domain was registered in the name of \enquote{Steve Rogers,} the Marvel comics character better known as Captain America. This digital taunt, along with email and IP addresses that verified the country of origin was not a false flag, all but guaranteed that the attack had come from an APT sponsored by the Chinese government, likely the Chinese military's cyber-espionage division.\footcite{koerner_inside_2016} Sitting on conclusive digital forensics, the Obama administration nonetheless refused to name China as the culprit.\footnote{National Security Advisor John Bolton was the first American official to formally acknowledge that the Chinese were behind the OPM hack, three years later, in September 2018. See \cite{sanger_trump_2018}}

The US government's tepid response to the OPM hack illustrates a universally understood but rarely acknowledged fact about the relationship between cybersecurity and espionage: when a cyber operation appears to fall within the bounds of a traditional intelligence operation, it will merit the same diplomatic consequences that a traditional intelligence operation would have. This justification was explicitly deployed by senior US policymakers, albeit often through unofficial moments of candor. A senior White House official told the Congressional Research Service that \enquote{there is a vast distinction between intelligence-gathering activities that all countries do and [other forms of malicious cyber-activity].}\footcite[In the context of the quote, the White House official is distinguishing between traditional espionage and economic espionage, the latter of which is the state-sponsored theft of trade secrets from foreign private companies in order to gain a competitive advantage in international markets.]{finklea_cyber_2015} For activities that fall under the umbrella of intelligence-gathering, counterintelligence, not prosecution, is deemed to be the appropriate response. Director of the CIA John Brennan described foreign spying on domestic political institutions as \enquote{fair game.}\footcite{sanger_u.s._2016} And while speaking at an intelligence conference, James Clapper, Director of National Intelligence, was especially honest: \enquote{You have to kind of salute the Chinese for what they did. If we had the opportunity to do that, I don’t think we’d hesitate for a minute.}\footcite{pepitone_clapper_2015}

% latex table generated in R 3.4.3 by xtable 1.8-2 package
% Sat Aug 24 13:46:46 2019
\begin{table}[ht]
\centering
\begin{tabular}{lrrr}
  \hline
Type & Incidents & Official responses & Response \% \\
  \hline
Defacement &   3 &   2 & 66.67 \\
  Data destruction &   8 &   5 & 62.50 \\
  Doxing &   6 &   3 & 50.00 \\
  Sabotage &  14 &   5 & 35.71 \\
  Denial of service &  17 &   6 & 35.29 \\
  Espionage & 231 &  44 & 19.05 \\
   \hline
\end{tabular}
\caption{Cyber-incident response rates by type of incident}
\label{response-pct}
\end{table}

In fact, cyber-espionage is routinely treated as a lesser category of offense by all nations, not just the United States. Of the hundreds of state-sponsored cyberattacks about which there is public knowledge, attacks classified as espionage are by the far most common---and the least likely to garner an official response from the victim government (Table \ref{response-pct}).\footnote{These numbers were compiled from the Council on Foreign Relations' Cyber Operations Tracker, which is a publicly-sourced record of state-sponsored cyberattacks, updated quarterly. Of the 323 listed incidents, 38 were excluded from this calculation because they lacked an identifiable Victim, Sponsor, or Type. For more, see \cite{council_on_foreign_relations_new_2019}} The routine minimization of espionage is most easily observable with the plethora of cyber-incidents that we deal with today, but it has its roots in norms that are much older.

At first glance this observation might seem almost tautological. Of course cyber-espionage is treated like traditional espionage, one could argue---they're both espionage. That formulation begs the question; it doesn't explain how states determine what constitutes \enquote{traditional} espionage in cyberspace, what purpose that distinction servers, or whether it is prudent to continue making it. Therefore, to understand why the Director of the CIA treats cyber-espionage as \enquote{fair game,} one needs to understand why traditional espionage receives that treatment first.

I posit that the universal tolerance of espionage in the international system can be explained through a rational calculation of each state's incentives. On balance, states derive a benefit from preserving a system in which the consequences for espionage are minimized, so they refrain from imposing particularly harsh punishments for its practice. Even when an individual incident is especially damaging to a state's security, if it falls within the traditionally accepted boundaries of an espionage operation, then it will likely go unpunished. In the case of the OPM hack, US policymakers decided that they could not impose harsh penalties on the Chinese because it was more valuable to them that the US retain its ability to practice cyber-espionage with relative impunity in the future. This basic calculation has remained consistent since the advent of the modern peacetime intelligence service, perhaps even the nation-state itself.

This article seeks to establish a theoretical framework for analyzing the role that espionage plays in international relations (IR) and the diplomatic norms that are associated with it. Through this framework, I explain what constitutes traditional espionage and why rational states might calculate that it is within their best interests to continue tolerating it. I then examine two cases in which new technical means of gathering intelligence were introduced during the Cold War: aerial photography (spy planes) and satellite photography (spy satellites). In both cases, the norms of traditional espionage ultimately prevailed, minimizing the diplomatic consequences for their use as long as it was in service of traditional intelligence goals. Through process tracing, I demonstrate how the justifications that policymakers employed for their use of these intelligence technologies---and the resulting minimization of the diplomatic penalties associated with them---are consistent with the theoretical underpinnings for why a tacit accord permitting the use of espionage could be mutually strategic.

\subsection{Existing literature}
Before continuing, it is necessary to properly define what I mean by espionage norms. While there are a number of practices that make espionage unique in IR, the norms examined in this article are the ones that govern the diplomatic protocols for intelligence. What consequences does a state impose when it uncovers an espionage operation run by another state? Espionage norms dictate that the consequences remain minimal. Rarely does the response amount to more than a pat denunciation of the activity and implementation of the necessary countermeasures to end it. Most importantly, those consequences never rise to the level where they impose additional costs on the state that attempted the espionage in the first place; in the rare cases when costs \emph{are} imposed for a traditional espionage operation, they are not costs that would discourage the aggressor from attempting such an operation again. In this article I present a theory for why those norms exist.

Prior literature has attempted to explain espionage norms through the grammar of international law, but in the absence of any significant treaties regarding espionage, scholars must work with in ambiguities.\footnote{Because it is universally agreed that international law does not deal with peacetime espionage directly, many journal articles that deal with this question frame their work almost Socratically, devoting two sections to cases for and against the legality of espionage. In fact, the first major work on law and peacetime espionage is a series of commentaries collected in a single publication that respond to each other's arguments. See \cite{wright_essays_1962}} Espionage is illegal everywhere, so how can every state acknowledge that allies and adversaries alike are guaranteed to engage in it?\footnote{Further complicating matters, espionage is the rare crime that might not be a crime depending on who the victim is. If an American government employee walks out of The Pentagon and hands classified documents to the KGB, they are at risk prosecuted as a traitor; if an American case officer hands classified KGB documents over to the CIA, they will be lauded as a hero. This example, in various forms, is frequently used to illustrate the impossibility of creating a consistent legal doctrine for espionage.} There is no legally consistent way to resolve that, so scholars have typically treated espionage as \emph{sui generis}---starting from the assumption that espionage norms exist in a class of their own, then attempting to fit a legal doctrine around that fact. Jared Beim determines that there while peacetime espionage is often tolerated, it clearly constitutes an \enquote{intervention} as defined by the International Court of Justice, so the prohibition on espionage merits further enforcement.\footnote{Beim also suggests that a weaker country could seek justice for an espionage operation by appealing to international organizations like the United Nations or the Council of Europe. For more, see \cite{beim_enforcing_2018}} Beim acknowledges, however, that his conclusion will have little impact on the practice of espionage, where, at best, \enquote{it is perhaps possible to stop the more egregious violations.}\footcite[p.~672]{beim_enforcing_2018} A. John Radsan takes a more philosophical approach. Suggesting that we should resist the \enquote{Hegelian impulse} to resolve the tension between espionage and international law, Radsan concludes that \enquote{Beyond any international consensus, countries will continue to perform espionage to serve their national interests \textelp{} International law does not change the reality of espionage.} Legal scholars of espionage all come to the same conclusion: it would be almost impossible to argue that espionage is legal, but it will likely continue to be downplayed. Beyond simple inertia, this literature cannot explain why espionage is tolerated.

% espionage is an inherently unique category of IR, whose norms are endogenous to the insular intelligence community and therefore bear little relation to other legal or strategic considerations.

Another branch of scholarship sidesteps the legal question in favor of a \enquote{functional theory} of IR. This is closer to my work, as it attempts to explain \emph{why} states tolerate espionage rather than taking as a given that they will. A functional approach, for instance, might highlight the role of espionage in treaty verification, the way Christopher Baker does when he argues that \enquote{the advantages that espionage offers over legally-binding verification and assurance regimes tip the scales in favor of functional cooperation.}\footcite{baker_tolerance_2004} Baker correctly notes that the additional assurance provided by espionage enables states to enter into riskier agreements that they would otherwise have avoided. His analysis is limited, however, to arenas in which espionage may facilitate various forms of interstate cooperation. It doesn't resolve the gap between the absolute illegality of espionage in domestic law and its widespread practice internationally. Responding to Baker, Radsan notes that one-off deals and treaties \enquote{are a very slow and indirect way to achieve international consensus on the legality of espionage. The gap is still there.}\footcite[p.~607]{radsan_unresolved_2007} This article expands on Baker's basic premise---that there are situations in which espionage is mutually advantageous for both parties---and universalizes it to explain why the entire institution of espionage is mutually advantageous. Espionage norms are the glue that keep the institution in place, despite its legal ambiguity.

The key insight this article presents---and where it breaks from previous literature---is that far from being an irreplicable, historically contingent set of practices, espionage norms are actually preserved through an ongoing set of strategic decisions. Each time a new espionage technology is made available, policymakers are given a fresh chance to decide whether or not the norms of espionage out to extend to its use. Critically, the research will show that they were not obligated to do so by any pre-existing norms. When the US first introduced photographic satellites, for instance, it was not clear whether the Soviet Union would treat them as spies or as weapons---there were indications in both directions. Ultimately, the infrastructure of what we now call \enquote{spy satellites} was developed, and the norms of espionage prevailed.

% Because I argue that espionage norms influence cyber policy, it is worth mentioning scholarship on the role that espionage, outside of cyberspace, plays in the international system today. The intersection of espionage and international relations has been studied before---beginning, in the modern context, as far back as 1962---but usually in a historical context, not a political science one. The role of espionage in deescalating conflict is typically relegated to accounts of moments where Great People properly interpreted---or fatally misunderstood---the information that their intelligence operations provided them. Chapters 3 and 4 will primarily focus on decisions made by Eisenhower and Khrushchev, so to some extent I am guilty of that as well. Nonetheless, I will try to highlight the way in which both of those leaders set precedents designed to outlast them, across a variety of intelligence methods. I hope that this thesis starts the conversation about how espionage norms operate to deescalate conflict independent of the leaders that promote them.


While not necessarily indicative of leading scholarship, it is revealing that popular discourse about espionage often uses terms like \enquote{rules of the game} or \enquote{gentleman's agreement} to describe the understanding between governments that espionage should fall within certain parameters. That these informal rules exist in some capacity is beyond a doubt.

% espionage as deescalataory

% the second contribution is the role of espionage in deescalating conflict

% By employing both realist and constructivist schools of IR, this article develops rigorous explanations for why rational states, whether they be security-seeking and greedy, might cooperate to preserve this dynamic.

% incorporating cyber espionage into the intelligence literature, rather than the cyebr literature
% espionage first, cyber second

\section{Theory}


% In recent years, cyberattacks perpetrated by state-sponsored foreign agents have received ever-increasing levels of public scrutiny---which makes the government's response to the OPM hack especially puzzling. Why did an incident of this magnitude not merit an official response?

% This article seeks to explain why espionage efforts are not more actively discouraged by nations, even as the results of intelligence operations can oftentimes have a devastating effect on the targeted nation's security.

% Russian interference in the 2016 US Presidential election prompted the expulsion of 37 diplomats and series of follow-on sanctions


% first it will lay out how espionage is considered an occupational hazard in the business of running a nation-state, one that is counteracted but never discourages.

% next I will discuss the theoretical justifications for why this might be the case. security seeking states and types of greedy states derive a reational benefit from allowing espionage to continue.

% finally I will look at historical moments in which new technologies were deployed for espionage-like purporses, and demonstrate how the norms governing the use of these technologies always evolved to permit espionage.

\end{document}
