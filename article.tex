\documentclass[14pt]{extarticle}
\usepackage{style}


\title{Reciprocal Tolerance: Why States Allow Their Adversaries to Practice Espionage with Minimal Consequences}
\author{Alexander W. Petros}
\date{October 2020}

\addbibresource{article.bib}

\begin{document}
\maketitle

In 2015, federal authorities discovered a massive data breach inside the United States Office of Personnel Management (OPM). Hackers had, over the course of months, downloaded reams of sensitive data, including a cache of applications for federal security clearance. These applications exposed not just the personal information of the applicants themselves but also the detailed information they had supplied about their family and friends, including social security numbers, job applications, and home addresses---compromising the information of 22.1 million people in total.\footcite{nakashima_hacks_2015} US officials feared that the stolen data could help foreign intelligence services unmask undercover CIA officers, and the information could identify American officials who might be susceptible to blackmail, or worse, recruitment. \enquote{[I]f you have my [application for security clearance],} FBI Director James Comey admitted, ``you know every place I've lived since I was 18, contact people at those addresses, neighbors at those addresses, all of my family, every place I've traveled outside the United States. Just imagine if you were a foreign intelligence service and you had that data.''\footcite{nakashima_hacks_2015} In the case of the OPM hack, a vivid imagination wasn't necessary; the US had already established that the Chinese military was likely behind the sophisticated attack.\footcite{koerner_inside_2016} But even though it possessed conclusive digital forensics, the Obama administration refused to name China as the culprit.\footnote{National Security Advisor John Bolton was the first American official to formally acknowledge that the Chinese were behind the OPM hack, three years later, in September 2018. See \cite{sanger_trump_2018}}

The US government's tepid response to the OPM hack illustrates a potentially surprising aspect of \nth{21} century diplomacy: when a cyberattack looks like a traditional intelligence operation, little to no diplomatic consequences will follow. Within political and academic circles, this phenomenon is often taken for granted; cyber-espionage is treated like \enquote{traditional} espionage because they are both forms of espionage. The scale of the OPM hack, however, stretches our understanding of the kind of damage that espionage can achieve. ``This is one of those cases where you have to ask, `Does the size of the operation change the nature of it?'\thinspace'' said one anonymous senior intelligence official. \enquote{Clearly, it does.}\footcite{sanger_u.s._2015} The actions of the US government say otherwise: at least for now, even operations on the scale of China's OPM infiltration still do not merit significant retaliation.

To evaluate whether espionage norms should apply in cyberspace, one first needs to understand espionage—--specifically, why espionage has traditionally been met with few diplomatic consequences. Though not codified in US policy, the light touch applied to espionage is an open secret. A senior White House official told the Congressional Research Service that \enquote{there is a vast distinction between intelligence-gathering activities that all countries do and [other forms of malicious cyber-activity].}\footnote{In the context of the quote, the White House official is distinguishing between traditional espionage and economic espionage, the latter of which is the state-sponsored theft of trade secrets from foreign private companies in order to gain a competitive advantage in international markets. See \cite{finklea_cyber_2015}} For activities that fall under the umbrella of intelligence-gathering, counterintelligence rather than prosecution is considered the appropriate response. CIA Director John Brennan once described foreign spying on domestic political institutions as \enquote{fair game.}\footcite{sanger_u.s._2016} And while speaking at an intelligence conference, James Clapper, Director of National Intelligence, was particularly frank: \enquote{You have to kind of salute the Chinese for what they did [to OPM]. If we had the opportunity to do that, I don’t think we’d hesitate for a minute.}\footcite{pepitone_clapper_2015} This permissive attitude is norm that governs espionage diplomacy, and its strategic purpose is rarely questioned or justified.

% This article seeks to unravel that tautology by first by explaining how states determine what constitutes \enquote{traditional} espionage and then by analyzing what strategic purpose that distinction serves.

This article posits that the universal tolerance of espionage in the international system can be explained through a rational calculation of each state's incentives. On balance, all states derive benefits from preserving a system in which the consequences for espionage are minimized, so they refrain from imposing particularly harsh punishments for its practice. Even when an individual incident is especially damaging to a state's security, if it falls within the traditionally accepted boundaries of an espionage operation, it will likely go unpunished. In the case of the OPM hack, for example, US policymakers decided they could not impose harsh penalties on the Chinese because it was more valuable for the US to retain its ability to practice cyber-espionage with relative impunity in the future. This basic calculation---that espionage against one's own country is a tolerable price to pay for being allowed to practice espionage against one's rivals---has remained consistent since the advent of the modern peacetime intelligence service, perhaps even the nation-state itself.

To evaluate each state's incentive, I first establish a theoretical framework for analyzing the role that espionage plays in international relations (IR) and the diplomatic norms that are associated with it. Through this framework, I define the bounds of traditional espionage and why rational states might calculate that it is in their best interests to continue tolerating it. I then examine two cases in which new technical means of gathering intelligence were introduced during the Cold War: aerial photography (spy planes) and satellite photography (spy satellites). In both cases, the norms of traditional espionage ultimately prevailed, minimizing the diplomatic consequences for their use as long as it was in service of traditional intelligence goals. Through process tracing, I demonstrate how the justifications policymakers employed for the use of these intelligence technologies---and the resulting minimization of the diplomatic penalties associated with them---are consistent with the theoretical underpinnings for a mutually strategic tacit accord permitting peacetime espionage.

% Add information about my conclusions here

\subsection{Defining espionage norms}
There are few extant definitions of peacetime espionage, and no satisfying ones. The problem that aspiring espionage scholars inevitably encounter is that \enquote{peacetime espionage} is universally understood as a term of art but almost never defined by legal codes, or interpreted by case law. Most scholars who tackle this question settle on a definition like the one proposed by I\~{n}aki Navarrete and Russell Buchan: \enquote{\enquote{Peacetime espionage} is a colloquial rather than a legal term [used] to describe different methods of collecting confidential information from closed as opposed to open sources \textelp{} generally conducted by States, against States.}\footcite[p.~901-902]{navarrete_out_2019} This definition purposefully excludes forms of extralegal government intervention that are often conflated with espionage, such as assassinations, election interference, and covert operations.\footnote{Often papers won't take a stand on defining espionage, but rather offer a working definition just to bound the scope of that particular paper. See \cite[p.~600]{radsan_unresolved_2007}, \cite[p.~2]{kapp_spying_2007})} Excluding these activities is important: assassinations and similar covert operations are not considered espionage, so they do not typically receive the protections of espionage norms.

For the purposes of this paper, I define espionage norms as the common understanding of how states are expected to respond to activities that remain within the bounds of espionage.\footnote{There are, of course, other \enquote{norms} that involve espionage. The so-called \enquote{Moscow Rules} dictate an intelligence officer are supposed to conduct themselves in hostile territory. In order to avoid constantly specifying, espionage norms are more strictly defined to include only the ones that affect IR.} Espionage norms prescribe that when one state uncovers an espionage operation conducted against it, the consequences imposed on the spying country remain minimal. Rarely does the response amount to more than a \emph{pro forma} denunciation of the activity and implementation of the necessary countermeasures to end it---the consequences for espionage are borne by the agents who commit it, not the state that sent them. In rare cases when additional costs \emph{are} imposed on a state for conducting a traditional espionage operation, they are not costs that would discourage the aggressor from attempting such an operation again.

The limits of espionage norms are defined by what intelligence professionals call \enquote{the rules of the game.} When states do impose harsher penalties, they often justify it by saying that they are punishing an operation that did not abide by the rules. The poisoning of Sergei Skripal, allegedly by Russian operatives, was widely described a violation of the rules.\footcite{masters_has_2018} Assassinations are off-limits, especially against former spies who have been traded to safety, and the US and the EU have issued multiple rounds of sanctions in response.\footcite{reuters_e.u._2019} Assassinations, however, are not the typical intelligence operation, and when an operation is understood by both parties to have been routine espionage, the consequences are minimized accordingly.

% latex table generated in R 3.4.3 by xtable 1.8-2 package
% Sat Aug 24 13:46:46 2019
\begin{table}[ht]
\centering
\begin{tabular}{lrrr}
  \hline
Type & Incidents & Official responses & Response \% \\
  \hline
Defacement &   3 &   2 & 66.67 \\
  Data destruction &   8 &   5 & 62.50 \\
  Doxing &   6 &   3 & 50.00 \\
  Sabotage &  14 &   5 & 35.71 \\
  Denial of service &  17 &   6 & 35.29 \\
  Espionage & 231 &  44 & 19.05 \\
   \hline
\end{tabular}
\caption{Global cyber-incident response rates}
\label{response-pct}
\end{table}

While most of the information that we have on espionage norms is qualitative, there is some data to support the limited consequences of espionage norms as well. Once again we can use cyber-espionage as an analogue for traditional espionage. Of the hundreds of state-sponsored cyberattacks known to the public, attacks classified as espionage are by far the most common. Cyber-espionage is, however, easily the least likely form of cyberattack to garner an official response from the attacked state (Table \ref{response-pct}).\footnote{These numbers were compiled from the Council on Foreign Relations' Cyber Operations Tracker, which is a publicly-sourced record of state-sponsored cyberattacks, updated quarterly. Of the 323 listed incidents, 38 were excluded from this calculation because they lacked an identifiable Victim, Sponsor, or Type. For more, see \cite{council_on_foreign_relations_new_2019}} In the world of cyber-incidents, if it looks like espionage then it is espionage, and it will receive the customary diplomatic benefits associated with that characterization.\footnote{In programming, this is a principle called \enquote{Duck Typing.} If it looks like a duck, and it quacks like a duck, then it is a duck.}



% The benefits of espionage have never been underrated---\enquote{it is only the enlightened ruler and the wise general,} reads Sun Tzu, \enquote{who will use the highest intelligence of the army for purposes of spying and thereby they achieve great results}---but the inevitability of espionage has long been taken for granted. If one questions the assertion that espionage is something \enquote{all countries do,} then the way that countries treat espionage becomes considerably more confusing.

\subsection{Extant explanations for espionage norms}
Prior literature has attempted to explain espionage norms using international law, but in the absence of any significant treaties governing peacetime espionage, scholars are often left without definitive answers.\footnote{Because it is universally agreed that international law does not deal with peacetime espionage directly, many journal articles that deal with this question frame their work almost Socratically, devoting two sections to cases for and against the legality of espionage. In fact, the first major work on law and peacetime espionage is a series of commentaries collected in a single publication that respond to each other's arguments. See \cite{wright_essays_1962}} Espionage is universally illegal but every state acknowledges that allies and adversaries alike are guaranteed to engage in it.\footnote{Further complicating matters, espionage is the rare crime that might not be a crime depending on who the victim is. If an American government employee walks out of The Pentagon and hands classified documents to the KGB, they are at risk prosecuted as a traitor; if an American case officer hands classified KGB documents over to the CIA, they will be lauded as a hero. This example, in various forms, is frequently used to illustrate the impossibility of creating a consistent legal doctrine for espionage.} Because of that tension, scholars have typically treated espionage as \emph{sui generis}---starting from the assumption that espionage norms exist in a class of their own, then attempting to fit a legal doctrine around that fact. Jared Beim concludes that while peacetime espionage is often tolerated, it clearly constitutes an \enquote{intervention} as defined by the International Court of Justice, so there ought to be stronger enforcement mechanisms in place.\footnote{Beim also suggests that a weaker country could seek justice for an espionage operation by appealing to international organizations like the United Nations or the Council of Europe. For more, see \cite{beim_enforcing_2018}} Beim acknowledges, however, that at best \enquote{it is perhaps possible to stop the more egregious violations.}\footcite[p.~672]{beim_enforcing_2018} A. John Radsan employs a more philosophical approach. Suggesting that we should resist the \enquote{Hegelian impulse} to resolve the tension between espionage and international law, Radsan concludes that \enquote{beyond any international consensus, countries will continue to perform espionage to serve their national interests \textelp{} International law does not change the reality of espionage.} Legal scholars of espionage generally reach the same conclusion: it would be almost impossible to argue that espionage is legal, but the international system leaves this tension in a state of salutary neglect. Taken to its logical endpoint, this branch of scholarship would simply argue that the espionage norms we observe today result from simple inertia.

Another branch of scholarship sidesteps the legal question in favor of a \enquote{functional theory} of IR. This approach is more informative because it attempts to explain why states tolerate espionage rather than creating \emph{post hoc} justifications for it. A functional approach, for instance, might highlight the role of espionage in treaty verification, the way Christopher Baker does when he argues \enquote{the advantages that espionage offers over legally-binding verification and assurance regimes tip the scales in favor of functional cooperation.}\footcite{baker_tolerance_2004} Baker correctly notes that the additional assurance provided by espionage enables states to enter into riskier agreements that they would otherwise have avoided. His analysis is limited, however, by its focus on enabling specific policy outcomes, like treaties. It doesn't resolve the gap between the absolute illegality of espionage in domestic law and its widespread practice internationally, nor does address why espionage would be tolerated in cases where no tangible outcome, like a treaty, is forthcoming. Responding to Baker, Radsan notes that one-off deals and treaties \enquote{are a very slow and indirect way to achieve international consensus on the legality of espionage. The gap is still there.}\footcite[p.~607]{radsan_unresolved_2007}

The key insight this article presents---and where it breaks from previous literature---is that far from being an irreplicable, historically contingent set of practices, espionage norms are actually preserved through an ongoing set of strategic decisions. Each time a new espionage technology is made available, policymakers are given a fresh opportunity to decide whether or not the norms of espionage ought to extend to its use. The research that follows will demonstrate that they were not obligated to do so by any pre-existing norms. When the US first introduced photo-reconnaissance satellites, for example, it was not clear whether the Soviet Union would treat the satellites as spies or as weapons. Ultimately, the infrastructure of what we now call \enquote{spy satellites} was developed, and the norms of espionage prevailed to protect their use.

\section{Theory}
\subsection{The nuclear option}
% Add a poll about American knowledge of the CIA?

Civilian study of espionage is sometimes treated with skepticism because espionage agencies, by definition, operate with maximum possible secrecy. That secrecy does not inoculate these agencies from critical analysis. While the operations of a modern intelligence service are opaque, the decision to maintain one is not. American citizens, as well as their international counterparts, are broadly aware that their nations conduct extensive intelligence operations abroad. The Central Intelligence Agency (CIA) is funded by the United States Congress and reports directly to the President---it even promises the possibility of a career in espionage by offering internship positions for college students. Keeping these institutions intact is partly a function of inertia, but it is also an ongoing argument that espionage is a legitimate, even critical, component of statecraft.

Using traditional methods of analysis, espionage certainly seems like something that states should be doing more to discourage. The results of intelligence operations can have devastating, even existential effects on national security. In recent years China has relentlessly targeted American military secrets. A United States Navy report from March 2019 declared Chinese intelligence operations so extensive that they had substantively altered the balance of power between the two states.\footcite{lubold_navy_2019} \enquote{Long-term, US future military advantage is being diminished by years of IP exfiltration from [US military agencies], all with little to no adverse consequences to the thieves.}\footcite[p.~6]{bayer_cybersecurity_2019} The forecasts were dire: \enquote{If the current trend continues unimpeded, the US will soon lose its status as the dominant global economic power.}\footcite[p.~5]{bayer_cybersecurity_2019} The US Navy is not a disinterested party, of course, and we should be cautious about its conclusions; modern military technology is remarkably complex and China cannot simply steal its way to military dominance.\footnote{For a good discussion of these limitations, see: \cite{gilli_why_2019}} Nonetheless, espionage presents alarming opportunities for rival states to gain a comparative advantage and the consequences for taking advantage of them remain minimal.

% Since this generally doesn't happen between major world powers, one of two things must be true: either a state is miscalculating the relative gains it achieves through espionage, or the mutual tolerance of espionage has benefits that are sufficiently important to maintain even if one state gains less from it than its adversary.

% Certainly there are factors that the go into diplomatic decisions that the general public will likely never know. There are few reasons to believe, however, that this is more true for intelligence-related diplomatic matters than for generic ones. Every diplomatic decision is made with the backing of classified intelligence.

Not only are the risks associated with practicing espionage minimal, espionage norms limit the acceptable repercussions to ones that cannot possibly threaten the structure of civilian intelligence services. Were it more common to respond to espionage by imposing financial sanctions or even inflicting a military cost, a state would have to consider the possibility that attempting espionage could adversely impact other aspects of its security, leaving it worse off than when it started. States often need to make these calculations in other high-stakes arenas; pursuing a nuclear weapons program or protectionist trade policies could lead to severe economic retaliation. Instead, when an especially damaging espionage operation is uncovered, the traditional punishment is to declare certain embassy officials \emph{persona non grata}, ejecting them from the country. This action often results in an equal and opposite retaliation: \enquote{tit-for-tat} diplomatic expulsions. Both states will expel a portion of the other's embassy staff that they suspect of being spies. Losing intelligence officers in an embassy temporarily makes it more difficult to conduct espionage, but it doesn't make a state less likely to try. According to John Sipher, former head of the CIA's Russia operations, previous Western expulsions have never crippled Russian intelligence collection. \enquote{If you have 150-200 intelligence officers in-country, losing 50 is painful, but hardly debilitating.} \footcite{dettmer_united_2018}

If removing a quarter of the officers doesn't do lasting damage, then why not remove the full 200? The swift expulsion of all suspected intelligence officers from a country---and the inevitable removal of the country's own intelligence officers in retaliation---would deeply injure both their espionage capabilities indefinitely.\footnote{Nothing on this scale has ever been attempted, but Operation Foot, the largest diplomatic expulsion in the Cold War, expelled 105 Soviet diplomats from the UK. While still not the total shutdown proposed in this section, Operation Foot \enquote{was a blow from which the Soviet intelligence effort in the UK never recovered.} For more, see: \cite{hughes_giving_2006-1}} This nuclear option is always available to both parties and yet, in modern history, it has never been exercised. We can reasonably infer that each nation's leadership is making an identical calculation: the benefits of continuing to practice espionage in my adversary's backyard outweigh the risks associated with allowing my adversary to do the same to me. The question remains how they can both be correct.

\subsection{A model for maintaining espionage norms}
I propose an iterated decision-making model to describe how states cooperate to preserve the liminal legal status of espionage, based on the same tit-for-tat strategy associated with diplomatic expulsions. In game theory, tit-for-tat is highly-successful strategy wherein the player simply mimics the choice that that their opponent made in the last round.\footcite[p.~8]{axelrod_effective_1980} By employing this strategy, the player never gets taken advantage of for more than one turn and can always return to cooperation if their opponent unilaterally signals a willingness to do the same. Diplomatic expulsions play out in exactly the same fashion: when one country expels the diplomats of another, the other country will often respond in kind. Each side will continue to expel each other's diplomats until one side signals that it would like for the reciprocal removal of diplomats to stop.

% Inevitably, both sides return to a cooperative state while they each still have \emph{some} diplomats left. Because neither side can gain an a decisive advantage, the back-and-forth punishment never ends in a mutual defection.

My model changes the parameters so that the decision to cooperate or defect is based not on the amount of punishment received, but rather the type. Each time a state is forced to respond to a successful intelligence operation by a rival---such as the reveal of a high-level spy or a major data breach---it can either limit its response to one which is traditionally associated with espionage, or it can impose unprecedented costs that might discourage such an operation in the future (Table \ref{espionage-decision-matrix}). If the state chose to impose strict financial sanctions or expel every suspected intelligence officer from its embassy, for instance, both of those would be considered punishments well in excess of the norm. In the status quo, State 1 signals its interest in cooperating by limiting its retaliation in response to espionage, no matter how damaging the operation is to their national security. State 2 responds by doing the same. Each iteration is a chance to play the game with fresh information; each espionage case is an opportunity to cooperate, reaffirming the norm that states always respond to espionage in a specific way, or defect, adding an additional cost to the attempt.

% https://tex.stackexchange.com/questions/249480/creating-a-payoff-matrix-using-latex-tabular-environment
\begin{table}[ht]
\centering
\setlength{\extrarowheight}{2pt}
\small
\begin{tabular}{cc|c|c|}
  & \multicolumn{1}{c}{} & \multicolumn{2}{c}{State 2}\\
  & \multicolumn{1}{c}{} & \multicolumn{1}{c}{Tolerate Espionage}  & \multicolumn{1}{c}{Punish Espionage} \\\cline{3-4}
  \multirow{3}*{State 1}  & Tolerate Espionage & \makecell{~\\Both sides gain intelligence \\~} & \makecell{~\\ Only State 2 continues spying \\ ~} \\\cline{3-4}
  & Punish Espionage & \makecell{~\\ Only State 1 continues spying \\~} & \makecell{~\\ Limited intelligence either direction \\~} \\\cline{3-4}
\end{tabular}
\caption{Espionage consequences decision matrix}
\label{espionage-decision-matrix}
\end{table}

While there is unquestionably a cultural aspect to espionage norms, tradition cannot fully explain why every state remains in a state of mutual cooperation. Espionage consequences are an iterated game that both parties play tit-for-tat, so, in the long run, the only two possible equilibria are mutual defection or mutual cooperation. In the realm of traditional espionage, we've only seen states engage in mutual cooperation, but any state has the power to change that by defecting. Even if the defector's rival would have preferred to maintain the status quo, the rival has no option other than to defect as well or else be left at a senseless disadvantage. The relatively open flow of information is only possible with a universal buy-in. Stripped of its mystique, the benefits of allowing intelligence operations to continue are, like any other state decision, a simple cost-benefit analysis; as illustrated above, states decide that it best serves their interests to remain in the top left corner, rather than the bottom right. We can expect that there is \emph{some} rational explanation that political leaders employ to justify it, where they weigh the benefits of preserving their ability to spy against the damage that results from espionage conducted against them.

% First, even though either state has the opportunity to defect by imposing harsher punishments in response to espionage, its gains would be short-lived because the other state is guaranteed to immediately do the same---tit-for-tat.

One possible explanation is that some states are simply miscalculating. Theoretically, the basic premise is persuasive. Every policymaker knows exactly how crucial espionage is to their job. With years of experience making difficult policy choices informed by intelligence, it is certainly much easier to imagine how one's decisions will be made worse without espionage than it is to imagine how one's enemies' decisions will be made better. It follows that there would be a strong psychological bias towards attempting to discover as much information as possible. While this is likely true to some extent, mass-miscalculation could not fully explain the widespread persistence of espionage.

The surprisingly straightforward conclusion is that if only two worlds are possible---one with widespread espionage and one without---then all the relevant nations must be concluding that the world without espionage is completely unacceptable. Even if one nation correctly concludes that its rival gains more from a world with a free flow of information than it does, the alternative---a world with significantly harsher punishments for espionage and a staunched flow of intelligence---still leaves the former worse off.

\subsection{Defensive realism}
To explain why states might be willing to accept relative losses in security, I employ a theory called defensive realism, particularly the version formulated in Charles Glaser's \emph{Rational Theory of International Politics}. Glaser describes his theory as integrating defensive realism with neoclassical realism.\footnote{For the sake of simplicity, I am going to refer to it as defensive realism, because they both put the security dilemma at the center of their explanation for international competition and cooperation, which I believe to be the key observation. He would have you know that defensive realism \enquote{can be viewed as a way station along the route to the full theory} that he develops, which is \enquote{significantly more general and complete than is defensive realism.} (p. 13-14)} Glaser begins from the core assumption that the reason for competition between security-seeking states is insecurity. Consider a hypothetical security seeker interacting with its adversary, which it believes to be a security-seeking state as well. If the first state feels insecure, it will build up armaments to increase its security. The rival state, seeing that its adversary is building up arms, now feels less secure and builds up its own arsenal in response, a classic security dilemma.

Like espionage norms, competition between security-seeking states has two natural equilibria: mutual cooperation and mutual defection. The ideal outcome for both is when neither state builds up arms---with their reciprocal lack of ability to deal damage, both states are confident that no war will result. Unilateral build-up leads to short-term relative gains in security, but in the next turn the other state will doubtless respond with a defection. The resulting mutual defection is worse than mutual cooperation. There might be peace, but it will be an uneasy peace; both states are heavily militarized and comparatively much less secure. When examined in the form of a payoff matrix, this game-theoretic choice is known as a Stag Hunt.\footnote{Just like the "Prisoner's Dilemma" takes its name from a hypothetical choice to snitch on a fellow prisoner, the "Stag Hunt" simulates a two-person hunting party. If both hunters cooperate they can take down a stag, but each also has the choice of hunting a hare on their own. If one person hunts a hare while the other hunts a stag, the hunter who went after the stag gets nothing. The hare is a smaller payoff, but doesn't rely on the other hunter's cooperation.} Mutual cooperation is the preferred outcome, but achieving it relies on both parties' expectations about what the other will choose. The incentives for a security dilemma are actually skewed in favor of cooperation slightly more than the traditional Stag Hunt, because mutual defection leaves both states slightly worse off than a single betrayal. A state that builds up its arms will be rewarded with an adversary threatening the same---at least in a Stag Hunt the two distrustful hunters will each end up with a hare.

\begin{table}[ht]
\centering
\setlength{\extrarowheight}{2pt}
\small
\begin{tabular}{cc|c|c|}
  & \multicolumn{1}{c}{} & \multicolumn{2}{c}{State 2}\\
  & \multicolumn{1}{c}{} & \multicolumn{1}{c}{Retain current arms}  & \multicolumn{1}{c}{Build up arms} \\\cline{3-4}
  \multirow{3}*{State 1}  & Retain current arms & \makecell{~\\(5,5)\\~} & \makecell{~\\ (1,3) \\ ~} \\\cline{3-4}
  & Build up arms & \makecell{~\\ (3,1) \\~} & \makecell{~\\ (2,2) \\~} \\\cline{3-4}
\end{tabular}
\caption{Defensive Realism payoff matrix}
\label{defensive-realism-payoff-matrix}
\end{table}

To arrive at mutual cooperation, a security-seeking state needs to not only determine that its rival is a security seeker, it needs its rival to make that exact same judgment, because it is the adversary's response to securitization that ultimately makes the first state insecure. Defensive realism therefore introduces a counterintuitive conclusion: it is often logical for a security seeker to take actions that increase not only its own security, but its \emph{adversary's} security as well.\footcite[p.~7]{glaser_rational_2010} If the adversary is indeed a security seeker, then, no longer concerned about its security, it will respond by not building up arms, and resolve the security dilemma for both of them. The problem for a defensive realist is that unilaterally taking steps to increase the adversary's security is obviously quite risky. The state could be wrong that its adversary is a security seeker---what Glaser calls a \enquote{greedy state}---in which case its optimistic overtures of cooperation will have exposed it to greater risk of attack. When states are unwilling to make those signaling measures because they fear compromising their own security, then a security dilemma results.

Espionage creates the conditions that make solving the security dilemma possible. Consider a situation in which one state is a genuine security-seeker and so is its adversary. The state conducts espionage on its adversary, and the resulting intelligence increases its internal estimate of how likely the adversary is to be a security-seeker. As a result, the state decides that some form of arms control is no longer too risky to attempt in order to signal its benign intentions. Because the state receiving the signal is also a security-seeker, this is a welcome development; it could be empowered to signal its benign intentions as well, based on its own intelligence. For both states, this is the optimal outcome, contributing to both information variables that Glaser identifies as key to the security dilemma: \enquote{the state’s estimate of the adversary’s motives, and the state’s estimate of the adversary’s estimate of the state’s motives.}\footcite[p.~34-35]{glaser_rational_2010} Both states therefore benefit not just from spying, but from being spied \emph{upon}; their adversary's intelligence made that adversary more likely to propose or accept a risky signal for deescalation.

States take the risk that espionage will reveal more than just a state's (hopefully benign) intent---it might also reveal a state's capabilities and strategy. I interviewed senior Obama Administration official Jake Sullivan about the benefits of allowing foreign states to spy on the US. According to him, the United States would have no problem with adversaries being able to independently verify American intent because, as he sees it, American intent is benign. Capabilities, however, are more complicated. Some defense capabilities would probably be fine to have an adversary independently verify, and some the US might want to keep secret because their secrecy confers a strategic advantage. And under almost no circumstances would the US want to have its strategies and warfighting plans revealed, so every measure should be taken to ensure their secrecy.\footcite{sullivan_personal_2019} States might prefer to reveal only the \enquote{good} information, but they can't. Espionage is by definition illicit; the fact the information was obtained without the target's permission lends it credibility.

\begin{table}[ht]
\centering
\setlength{\extrarowheight}{2pt}
\small
\begin{tabular}{cc|c|c|}
  & \multicolumn{1}{c}{} & \multicolumn{2}{c}{State 2}\\
  & \multicolumn{1}{c}{} & \multicolumn{1}{c}{Tolerate Espionage}  & \multicolumn{1}{c}{Punish Espionage} \\\cline{3-4}
  \multirow{3}*{State 1}  & Tolerate Espionage & \makecell{~\\(5,5) \\~} & \makecell{~\\ (0,4) \\ ~} \\\cline{3-4}
  & Punish Espionage & \makecell{~\\ (4,0) \\~} & \makecell{~\\ (2,2) \\~} \\\cline{3-4}
\end{tabular}
\caption{Espionage payoff matrix for two security-seeking states}
\label{espionage-payoff-matrix}
\end{table}

Therefore, states face a binary choice whether to tolerate espionage or punish it. We can translate that conclusion to a payoff matrix (for two security-seeking states) that looks almost exactly like the  security dilemma Stag Hunt: the greatest rewards for both are when they cooperate. Non-reciprocal espionage would generate significant relative advantages, but wouldn't confer the benefit of easing an adversary's fears, and neither side would allow it to last. The worst situation for both is when neither allows access to the other's information---that equilibrium minimizes the risk of losing of military secrets, but it significantly heightens the potential for conflict and miscalculation.

%%% %%%
% A second thing that is a little bit confusing to me about the theory is whether the theory is intelligence revealing information about intentions, or capabilities, or about the other actors "type."  Most of the examples you have in the empirical section seem to be really about gathering intelligence on capabilities rather than intentions. When you get to the cases it's never clear whether the US is discovering that the Soviet Union is truly a security seeker and hence promises escaping from the security dilemma, or whether the United States is actually gaining intelligence on Soviet capabilities that might, at the extreme, be used aggressively to “win” in a war with the Soviet Union.  It is unclear from your theory why actors would tolerate this espionage if it is unclear whether it is in fact making both secure or actually is making one side better able to plan an offensive operation against the other.
%%% %%%

\subsection{Assumptions and alternate explanations}
The analysis in this section relies on one crucial assumption: that the states in question are security-seekers. Analyses of the risks of espionage must contend with possibility that one of the states in question is \enquote{fundamentally dissatisfied with the status quo, desiring additional territory even when it is not required for security}---a greedy state.\footcite[p.~4]{glaser_rational_2010} Such a state would obviously want to protect both its motives and its plans, and therefore derive no benefit from having espionage conducted against it. Nonetheless, we know that essentially every major nation-state abides by the accepted norms of espionage. I have already addressed the possibility that states---including greedy states---are bound to these norms for reasons that extend beyond rational calculus, and do not consider that explanation to be sufficient. To fully evaluate the implications of greedy states on this theory of mutual cooperation is beyond the scope of this paper, but I will propose potential answers to the question of why greedy states would tolerate espionage as a starting point for future analysis.

Perhaps greedy states simply do not exist. This is a very provocative assertion---one that is, in its most stringent sense, almost certainly wrong---but consider the possibility that no states which abide by espionage norms are greedy \emph{to a destabilizing degree.} The concept of a greedy state is one that naturally operates on a spectrum.\footcite[p.~39. Throughout his book, Glaser simplifies and complicates his theory as is necessary for the particular analysis he is performing. Greedy states can be understood as the opposite of security-seeking states, or it can be its own independant variable, allowing for the possibility of states that are greedy \emph{and} security-seeking, or just one of the two.]{glaser_rational_2010} The quintessential \enquote{greediest} state, 1937 Germany, was planning to dominate the European continent. In 2020, if no states are worried about espionage revealing that they intend to launch a significant campaign of conquest, perhaps that is because there are no states intending to launch a significant campaign of conquest. To use a modern example, the CIA warned that Russia might try to move into the Crimean peninsula shortly before it happened in 2014.\footcite{hosenball_ukraine_2014} Russia practiced good operational security to protect its intent from espionage, but it still took a risk that its operation would be discovered and counteracted. If in fact the annexation of Crimea is the act of a greedy state, that level of greed is likely tolerable to Russia's adversaries in a way that a nuclear first-strike operation would not.

The international community is likely to view a sudden withdrawal from universal espionage norms as suspicious. States that are farther along on the greed spectrum might decide that it is better to risk the chance that they'll be discovered as greedy states rather send a signal that will all but confirm it. And it cannot be discounted that sometimes a state might not even be entirely confident of whether its \emph{own} intent is greedy. So far this paper has used the nation-state as its base level of analysis. As I present the empirical evidence for this theory however, it becomes necessary to analyze the motives of individual actors that make up the state. At this level, even officials within the same administration have conflicting ideas about how to use intelligence, and those collecting the intelligence can't be confident whose purposes they are serving.

\subsection{Research design}
In the following section, I will use empirical case studies to demonstrate how associating an operation with espionage alters its diplomatic response. In order to guide that research, it is important to establish a series of clear, testable hypotheses. I first present a null hypothesis:

\begin{quote}
$H_0 =$ Associating a state action with the traditional practice of espionage will not meaningfully alter the diplomatic response.
\end{quote}

The null hypothesis posits that there is insufficient evidence linking the perception that an act \enquote{is} espionage to the response that the affected state ultimately delivers. If this hypothesis holds, espionage may have some symbolic importance, but it isn't one that can be meaningfully tied to a diplomatic response. By contrast, this is the alternative hypothesis, the one that would prove true if the theory holds:

\begin{quote}
$H_1 =$ Associating a state action with espionage will result in a reduced diplomatic response when compared with equivalent actions that do not share that association.
\end{quote}

The argument presented by this article risks a tautology: perhaps we associate certain actions with espionage \emph{because} they received a reduced diplomatic response. For this reason, the relationship in $H_1$ is explicitly framed as causal. Unfortunately, defining the relationship that way makes it difficult to prove directly; no two state actions take place under equivalent circumstances, so no natural experiment could demonstrate that equivalent actions, perceived differently, received different responses. I therefore propose a corollary hypothesis to aid us in proving the alternative hypothesis:

\begin{quote}
$H_2 =$ States purposefully pursue measures to signal that their actions constitute espionage, because they apply reduced diplomatic responses to espionage and hope that their adversary will do the same.
\end{quote}

Proving $H_2$ is significantly easier than proving $H_1$. State actions---and the reasons policymakers provided for pursuing those actions---are often documented in the historical record. Careful research could demonstrate that \enquote{espionage} is an ontologically significant concept, not just for academics, but for the very people who engage in it. If $H_2$ holds, one would find evidence of policymakers being willing to accept the cost of enemy espionage in order to practice espionage themselves, based on the value of mutual observation. Furthermore, when $H_2$ is proven bidirectionally, it implies $H_1$. Two states that are both maintaining reduced responses to espionage (while pursuing espionage themselves) create a relationship in which $H_1$ holds: state actions successfully associated with espionage will receive a reduced response.

The following section will demonstrate that both the United States and the USSR were forced to contend with the benefits and risks of a mutually permissive espionage regime during the Cold War. When faced with opportunities to either shut down the espionage apparatus or expand it, they both chose to expand it. Policymakers on either side of the Iron Curtain repeatedly decided that preserving the bilateral flow of information ensured a more secure world environment. This article will present two such cases: aerial espionage and spy satellites.

\section{Case Studies}

\subsection{Aerial espionage}
Just a few weeks before the \enquote{Four Powers Summit} in Summer 1960, Soviet Premeir Nikita Khrushchev announced that the USSR had shot down an American plane.\footcite[p.~112]{powers_operation_2004} The CIA knew tgat we was referring to the U-2, an advanced spy plane that had bedeviled the USSR for years, proving beyond the reach of Soviet surface-to-air missiles.\footcite{orlov_u-2_2007} Nevertheless, the CIA was still confident that its cover story would hold, thanks to a morbid fact of the U-2's design: in the event of a shootdown, it should have been impossible for the pilot to survive.\footcite[p.~35]{lindgren_trust_2000} When the United States put out at a statement that the plane was part of a \enquote{weather reconnaissance} mission gone awry, Khrushchev sprung his trap---he revealed that the U-2 pilot, Gary Powers, had been taken alive.

In a few days, the previously pleasant trend in US-Soviet relations completely reversed. President Dwight D. Eisenhower admitted that the Gary Powers had been flying a spy plane, declaring in a chagrined press conference that espionage was ``a distasteful but vital necessity'' to prevent another attack like Pearl Harbor.\footcite{eisenhower_news_1960} At the summit, Khrushchev launched into a tirade about U-2 planes and walked out on the first day, effectively ending the summit. He also canceled a trip to the USSR that President Eisenhower was scheduled to take the next month. A contemporary newsreel about the conference announced that \enquote{in the course of two hours, Khrushchev brought US-Soviet relations to their lowest point since the end of World War II.}\footcite{universal_studios_summit_1960}

In the traditional telling, the U-2 Incident is cited as a major failure of the Eisenhower administration, one which significantly worsened US-Soviet relations at time when lasting peace appeared possible. Only in the under-studied context of Cold War reconnaissance flights does it become clear that the U-2 Incident actually shows both the US and USSR exhibiting restraint and sound judgment. Over a period of about 15 years, they each allowed provocative aerial intelligence operations to take place without military escalation. These operations were permitted because, consistent with $H_2$, the United States took great pains to signal that its various actions were within the bounds of espionage. The Soviet Union responded by honoring that signal and not escalating the situation beyond the immediately necessary measures to counteract it. The U-2 Incident should be understood as a uniquely embarrassing but otherwise typical failure of an intelligence operation---the maximal case for the minimal consequences typically applied to espionage.

% The shootdown of Gary Powers clearly violated the only \enquote{rule} of the Cold War. Official Soviet air defenses shot down a US aircraft. All parties involved had cause to escalate militarily---the Soviets for the violation of their airspace, the US for the attack on their \enquote{weather craft}---but instead they chose to treat the U-2 missions as an intelligence operation, not a military action. The Soviets therefore applied the traditional consequences associated with espionage; Khrushchev engaged in some saber-rattling and denounced the violation of his airspace, but no material actions were taken, and the United States escaped the incident with few tangible consequences. The striking disparity between the diplomatic furor and the lack of material consequences can be explained by an apparent technicality: the U-2 was not a military aircraft, it was a spy plane.

% No single attribute defines a spy plane. A military aircraft can forgo weapons but, by virtue of its pilot, its mission, its timing, or its branding still conduct what is unquestionably a military operation. Inversely, intelligence missions can be flown by uniformed Air Force pilots in re-purposed B-52s and somehow still be understood as espionage. Espionage is an ontological designation, one that reveals itself to the adversary through a series of social and physical signifiers that are not replicable across time. Germany in 1937 would likely have reacted to a U-2 overflight differently than the Soviet Union did in 1960. But even if the way different states handled espionage was perfectly consistent, there still wouldn't be a checklist to guarantee that a mission would be treated as espionage. It is not enough make a military mission look like an intelligence operation; the reality is both simpler and more frustrating---it has \emph{to be one}.


\subsubsection{Military reconnaissance flights}
Beginning in 1947, the US began a campaign of reconnaissance flights that routinely violated Soviet airspace. Converted Air Force bombers---called \enquote{ferrets}---were outfitted with advanced radio equipment and sent to intercept as much information about Soviet radio transmissions as possible.\footcite[p.~4]{peterson_maybe_1993} The pilots who flew these missions were conducting one of the earliest forms of Cold War peacetime espionage: gathering SIGINT (signals intelligence) that identified the location of critical radar installations along the Soviet border. Some of these flights only flew near the border, remaining over traditionally recognized international waters, while others were \enquote{overflights,} missions that deliberately violated Soviet airspace.

The tense battle over Soviet border surveillance was never formally acknowledged by either state---and neither were its combatants. Over the course of the Cold War, more than 200 American pilots were shot down while spying on the Soviet Union.\footcite{glenshaw_secret_2017} These missions were so highly classified that in many cases the families of the pilots never learned how they died. 126 of these American airmen remain unaccounted for today, casualties of missions whose existence both the US government and the Soviet Union collaborated for decades to deny. Not once did these shootdowns lead to retaliation from the US, or significant pressure from the Soviets to stop the flights. Instead, the cases resolved as if the pilots were spies, captured or killed in the course of counterintelligence, whereabouts undisclosed.

% The earliest definitive postwar overflight of the Soviet Union took place on May 10, 1949, when two RF-80As surveilled the Kurile islands\footcite[p.~8]{peebles_shadow_2000}

Without access to classified archives it is impossible to state for certain exactly how many of these flights were conducted.\footnote{Some of the necessary information was so highly classified that it is likely lost forever.} Nonetheless, the civilian researcher is aided by the declassified research of military and intelligence archivists. Writing for the internal CIA publication \emph{Cryptologic Quarterly}, Michael Peterson compiled a now-declassified list of American SIGINT flights shot down by the USSR (Table \ref{soviet-shootdowns}).\footnote{Michael Peterson is a former intelligence analyst and section chief for the CIA.} Peterson uses two limiting criteria that perfectly suit the purposes of this article: the only flights he lists are those that were (a) shot down by Soviet attacks, not Chinese, North Korean, etc., and (b) involved US aircraft exclusively performing reconnaissance.\footcite[p.~4. Regardless of how thorough declassified CIA information appears, one cannot rule out the possibility that there were other, even more secret missions which, for whatever reason, the United States government has chosen to conceal to this day. Nonetheless, a few aspects of the \emph{Cryptologic Quarterly} article suggest that the information is relatively complete. First, the article was written for a classified audience, likely given that designation so that it could be comprehensive, and declassified after 16 years. Additionally, it appears to have been declassified without redactions or missing pages. Lastly, if there had been another mission so highly classified that it was omitted from this document, it is difficult to imagine how diplomatic consequences could have been imposed with a degree of secrecy that would prevent us from knowing about it today.]{peterson_maybe_1993} In effect, the table catalogs the 13 times where the USSR shot down an American aircraft designed for collecting intelligence between 1950 and 1964---intercepted espionage attempts.\footcite[p.~5. In the original document, this table lists the first incident as having taken place over the Barents Sea, not the Baltic Sea. Because the description of the mission---including a map of its route in the same document---takes place entirely over the Baltic sea, I have concluded that this must be a typographical error, and corrected it here.]{peterson_maybe_1993}


\begin{table}[ht]
\centering
\begin{tabular}{llr}
\textbf{Date}     & \textbf{\makecell[l]{U.S. Service \&\\ Aircraft Type}}   & \textbf{General Location} \\
8 April 1950      & USN PB4Y2 Privateer           & Baltic Sea                           \\
6 November 1951   & USN P2V Neptune               & Sea of Japan                         \\
13 June 1952      & USAF RB-29                    & Sea of Japan                         \\
7 October 1952    & USAF RB-29                    & East of Hokkaido/Kuril Is.           \\
29 July 1953      & USAF RB-50                    & Sea of Japan                         \\
4 September 1954  & USN P2V Neptune               & Sea of Japan                         \\
7 November 1954   & USAD RB-29                    & East of Hokkaido. Kuril Is.           \\
18 April 1955     & USAF RB-47                    & Off Kamchatka Peninsula              \\
10 September 1956 & USAF RB-50                    & Sea of Japan                         \\
2 September 1958  & USADF C-130                   & Soviet Armenia
\\
1 May 1960        & CIA U-2                       & Sverdolsk, USSR                      \\
1 July 1960       & USAF RB-47                    & Barents Sea                          \\
10 March 1964     & USAF RB-66                    & East Germany
\end{tabular}
\caption{Summary of Soviet Shootdowns, 1950-1964}
\label{soviet-shootdowns}
\end{table}

Peterson's writings represent the CIA's internal accounting for SIGNIT shootdowns, but he does not comprehensively investigate how those various missions were perceived by the Soviet Union. For that, I cross-referenced the Peterson list with historian Dr. John Farquhar's analysis of Cold War shootdowns that had a diplomatic impact.\footcite[Dr. John T Farquhar is a retired Lieutenant Colonel in the United States Air Force, and currently an Associate Professor of Military \& Strategic Studies at the US Air Force Academy.]{farquhar_aerial_2015} The combination of these two lists is instructive---some incidents in the CIA list are missing from Farquhar's list because they did not have a diplomatic impact, and some incidents that Farquhar analyzes are not mentioned by Peterson because they were not SIGINT reconnaissance missions. Taken together, the full spectrum of aerial reconnaissance incidents demonstrates a clear pattern: shootdowns that were interpreted as espionage were quickly resolved by both parties, while the shootdown of a flight unrelated to espionage produced swift, significant backlash.

% First, many of the incidents analyzed in this subsection are taken from clippings compiled by Air Force Academy historian Dr. John Farquhar, who specializes in aerial reconnaissance and its diplomatic consequences. Second,

The first of these shootdowns brought the aerial reconnaissance program to the attention of the American people on April 8, 1950.\footnote{Note that the flights analyzed here were not necessary overflights; many took place over international waters or dispute waters when they were shot down.} An unarmed Navy patrol plane was downed by Soviet fighters over the Baltic sea. The Soviet ambassador to the US lodged a formal note of protest, and a few days later, the American ambassador issued a formal response, claiming that at no point did the plane cross into any territorial waters.\footcite{kirk_ambassador_1950} He demanded the Soviets conduct \enquote{a prompt and thorough investigation} and \enquote{see to it that those responsible for this action [were] promptly and severely punished.}\footcite{the_associated_press_text_1950} The Soviet Union recognized this for the toothless statement that it was, and rejected the demand in its entirety.\footcite{salisbury_kremlin_1950} No further action was taken.

Consistent with $H_2$, the United States took steps to code this flight as espionage, which was successful not just in the eyes of the Soviet Union, but in the American press as well. \emph{Washington Post} columnist Walter Lippmann wrote: \enquote{the known facts indicate that Soviet intelligence \textelp{} believed [the plane] was carrying important electronic equipment and that orders were given to the Soviet fighter command to intercept it.}\footcite{lippmann_baltic_1950} A \emph{Post} article written a few days later was almost sympathetic to the Soviet commanders. \enquote{Electronics make the old delimitations for border coastal flights ridiculous. A plane flying on a course perfectly legal by standards accepted today might still be engaged in reconnaissance of the first importance that an unfriendly power would try to frustrate.}\footcite{childs_baltic_1950}

% This is true, both in terms of the mission itself and the technological environment, but he went even further. \enquote{The only sound and safe assumption is that the Russians have a thorough and far-reaching espionage system. And at the same time we must hope that our system, and particularly on the side of counterespionage, is effective.} Even the press, though outraged at the loss of American life, apparently took it as a given that this kind of espionage is necessary---and expected from both sides.

The Baltic incident exemplifies the reduced consequences for espionage that $H_2$ predicts, in both directions. The Soviet Union opened fire on a US Navy aircraft, which based on all available evidence did not violate the traditional 12-mile airspace boundary.\footcite[p.~7. To be extra specific]{peterson_maybe_1993} Given its last known position, it would have taken ``a navigational error of nearly 90 degrees to cause the craft accidentally to wander over the Baltic states.''\footcite{the_new_york_times_soviet_1950} Under other circumstances, this could have been a major incident, but the United States chose not to publicize it. The Soviet Union, for its part, chose not to escalate the issue either, despite the ample evidence that the US was engaging in military reconnaissance.

The resolution of the Baltic Incident set a precedent for reconnaissance flights. From then on until 1960, the United States lost nine more aircraft to Soviet air defenses under similar circumstances. Each time, the pattern of responses was the same---each side released dueling statements of protest, then the situation continued more or less as before. A B-29 that went missing on October 7, 1952 hit the front page of the \emph{New York Times} two days later. This mission is listed by Peterson as a ferret flight, and Farquhar notes that the media thought the attack was intended to lower American prestige during election season, but does not mention any diplomatic consequences.\footcite[p.~43-44]{farquhar_aerial_2015} The official US response, however, is preserved on the State Department website. As with the Baltic Incident, the US simply demanded that the Soviets pay for the plane and return any survivors.\footcite{the_new_york_times_u.s._1952} Two more shootdowns in March 1953---one involving an American F–84 Thunderjet and the other an RAF Lincoln bomber---actually resulted in conciliatory responses from both sides, including a secret meeting to reduce aerial tensions \enquote{of which little is known.}\footcite[p.~45]{farquhar_aerial_2015}

In another incident on July 29, 1953, the United States protested the shootdown of a B-50 (reconnaissance flight, per Peterson) over the Sea of Japan. The Soviet Union responded by protesting the alleged American shootdown of a Soviet passenger plane. Despite press outrage, no further action was taken.\footcite[p.~47]{farquhar_aerial_2015} After the September 4, 1954 shootdown of a Navy P2V Neptune, a US Senator publicly denounced the continued anodyne response: \enquote{just another note from our State Department to the Kremlin hierarchy will not impress these uncivilized rulers.}\footcite{the_associated_press_ending_1954} Instead of a harsher punishment, President Eisenhower brought the issue to the UN with the full knowledge that an unfavorable judgment would be vetoed by the Soviets. A further signal of deescalation, the UN stunt demonstrated Eisenhower's ability to resist calls for harsher action while still placating the American public.\footcite[p.~47]{farquhar_aerial_2015} By late 1954, the \enquote{cycle of hostility} over aerial reconnaissance was already breaking down.\footcite[p.~49]{farquhar_aerial_2015} In terms of both media coverage and diplomatic significance, no shootdown that followed would require more significant consequences than the ones that had already happened.

During this period of aerial reconnaissance, the United States also suggested that $H_1$ could be true by showcasing its inverse.\footnote{The inverse of $p \implies q$ is $\neg p \implies \neg q$ i.e. if an act is not espionage, then it will not receive the reduced consequences typically applied to espionage.} On November 20, 1951, an American C-47 transport was forced down in Hungary. Of all the cases that Farquhar examines, this is the only case that provoked a sustained diplomatic response from the United States---and the only case where the plane in question definitively carried no espionage equipment.\footcite[While the ferret flights were perfectly calibrated to fit as much radio surveillance equipment as possible, the only evidence of espionage that the Soviets were able to produce from the C-47---a plane that they recovered intact---was a portable radio, two extra parachutes, and some packets of warm blankets. Because it was not a ferret flight, the C-47 is omitted from the Peterson list as well.]{the_united_press_soviet_1951} He summarizes the American response to this incident as follows: \enquote{Responding to the press attention, the Truman Administration acted swiftly, attempting to gain the fliers release through diplomatic pressure. The President ordered the Hungarian consulates in New York and Cleveland closed and banned private travel to the country. Legislatively, Truman asked Congress to pass a \$100 million Mutual Security Act to aid \enquote{selected persons residing in Soviet bloc states or refugees who wanted to form armed units} in opposition to Communism.}\footcite[p.~43]{farquhar_aerial_2015} Truman immediately took action against material Hungarian interests in the United States and threatened to empower \emph{armed dissidents} if the prisoners were not released. The US government's response to an aviation incident when no espionage was involved stands in sharp relief to the \enquote{prompt and thorough investigation} that it demanded after the Baltic Incident.

The United States conducted many more overflights than the failed ones mentioned here, using a variety of retrofitted planes. Project Heart Throb, for example, outfitted RB-57s with surveillance equipment and flew between 15 and 19 missions over Eastern Europe from 1955 to 1956.\footcite[p.~194. Based on the recollection of Gerald E. Cooke, a Air Force pilot assigned to the project.]{hall_early_2003} Project Slick Chick set up F-100 jets with rapid-fire 20mm cameras.\footcite[p.~176]{hall_early_2003} Though we know a lot less about these missions than we do about the ferret flights, the basic contours are familiar: missions with no weaponry, operating at a high level of secrecy, subject to countermeasures (tracked by radar and often chased by MiG fighters) but no evidence of diplomatic pressure. In meetings between the State department and high-level Soviet diplomats, the subject of captured pilots \enquote{was broached only perfunctorily in relation to other things being discussed.}\footcite[p.~72]{brugioni_eyes_2010} Some family members of lost airmen received posthumous awards, but most simply received the pilot's personal effects and no explanation.

The quantity and consistency of these flights demonstrates that coding reconnaissance flights as espionage, prototyped by the Baltic Incident, had settled into a practiced routine. The Soviet Union understood that these planes only carried reconnaissance equipment, the United States understood that these planes might get shot down, and both sides understood that these clashes would not escalate into a larger diplomatic battle. As predicted by $H_2$, the US took steps to associate its activities with espionage, and it applied reduced consequences to the Soviet Union's counterespionage. And while we lack primary sources that betray the Soviet rationale, the circumstances suggest that, consistent with $H_1$, the Soviet Union applied reduced consequences to espionage over this period of time as well.

\subsubsection{Inventing the spy plane}
% For the spy plane to serve its intended purpose as a minimally provocative agent of espionage two things must be true: its operation must be understood by adversaries to be non-military, and that distinction must fulfill the promise of $H_1$ by meaningfully reducing the associated response.

The first recorded reconnaissance flight took place in 1911, but that was not the first spy plane.\footcite[Surveilling the Turkish infantry near Tripoli, \enquote{Lieut. Piazzi today, for the first time in the history of warfare, made from this place an aerial reconnaissance against a hostile power}]{special_cable_to_the_new_york_times_air_1911} Ontologically, the spy plane---distinct from a plane that is doing spying---was invented by the Eisenhower administration. As its name suggests, a spy plane is the physical manifestation of a state's attempt to code aerial reconnaissance as espionage. For such a device, what it actually does is less important than what the adversary understands it to be; a spy plane must garner the protection of espionage norms while performing far more aggressive missions than a traditional reconnaissance flight. To create this aircraft, President Eisenhower approved Project Aquatone: the U-2.

$H_2$ posits that states will pursue measures to signal that their activities are espionage, and the US did everything possible to ensure that the U-2 would be read as a civilian aircraft. Eisenhower required that the pilots of U-2 aircraft be CIA officers, specifically forbidding uniformed Air Force personnel.\footcite[p.~33, Though many of the pilots did have Air Force backgrounds.]{lindgren_trust_2000} The CIA also had operational control of the program instead of the Air Force. \enquote{I want this whole thing to be a civilian operation,} Eisenhower wrote, to settle an operational dispute between the two departments. \enquote{If uniformed personnel of the armed services of the United States fly over Russia, it is an act of war---legally---and I don't want any part of it.}\footcite[p.~60. The original source for this quote is an \emph{OSA History} that requires codeword clearance. It is quoted here by the History Staff of the CIA.]{pedlow_central_1992} This allowed the United States to, in the phrasing of the CIA, \enquote{truthfully deny} that any US \enquote{military planes} had flown over the USSR---which they had to do when the inevitable Soviet protest notes were filed after U-2 overflights began in 1956.\footcite[p.~109]{pedlow_central_1992}

The steps that Eisenhower took to ensure the U-2 civilian's character went beyond mere deception. If his only goal had been to achieve a level of plausible deniability in the event of a shootdown, then simply not wearing an Air Force uniform during the mission likely would have been sufficient. The classic \enquote{weather reconnaissance craft} cover had been used before with the ferret flights.\footcite[p.~45]{farquhar_aerial_2015} But policymakers didn't just want to pretend that Aquatone was a civilian operation, they wanted it to \emph{be} a civilian operation. \enquote{It is of utmost importance to differentiate in our minds, and to cause the Russians to differentiate in theirs, between Aquatone-type operations and reconnaissance by military aircraft} reads a top-secret CIA memo from 1956.\footcite[p.~1]{miller_suggestions_1956} No finite set of attributes could ensure that the Soviet Union wouldn't treat the U-2 plane---a massive leap in military surveillance technology---as a military development. But if the people in charge of Project Aquatone could convince \emph{themselves} that it was a civilian project, they stood a better chance of persuading the Soviet Union. Both halves of that equation were equally important: the Russians must believe that these operations are peacetime civilian operations, and they must be correct about it.

These precautions had the intended effect, even though it initially seemed like the Soviets were attempting to extract significant concessions. A Soviet colonel in the USSR Air Defense later recalled that Khrushchev ``clearly viewed the violation of their nation's skies by a foreign reconnaissance aircraft \textelp{} as a political provocation,'' and yet few real consequences followed.\footcite{orlov_u-2_2007} No one disputes that the collapse of the Paris Summit was directly attributable to the Gary Powers shootdown but, in practice, summits are not where important issues of the day get resolved.\footnote{The main issue that did not get resolved in 1960 was the status of Berlin. It also did not get resolved at the next summit in 1961, a comparatively warmer affair between Khrushchev and Eisenhower's successor, John F. Kennedy.} From 1953 to the end of the Cold War in 1991, there were 23 total US-USSR summits.\footcite{fain_chronology_2011} With the surprise rollout of the captured pilot, the dramatic scene at the summit, and the cancellation of Eisenhower's visit, Khrushchev publicly embarrassed the United States without meaningfully altering Soviet policy towards it.

Nor did the U-2 Incident mark the end of reconnaissance flights. As is characteristic of espionage, when the US really needed intelligence that only an overflight could provide, the diplomatic consequences that the USSR had imposed were not sufficient to discourage it. Two months to the day after Powers was shot down, a Soviet MiG-19 opened fire on an American RB-47H and captured the plane's navigator and co-pilot. While Powers was serving a 10-year sentence in Soviet prison, the two airmen were returned to the US in less than six months with only mild fanfare. Khrushchev wrote in his memoir that he wanted to continue \enquote{our general line of peaceful coexistence} and cited the resolution of this incident as an example of such, because the US was forced to make a formal request for the return of the airmen.\footcite[p.~256-257]{khrushchev_memoirs_2007} The only concessions the US made in return were to announce the discontinuation of its U-2 overflights (which Kennedy was already committed to) and to not make an issue of the illegal detention of the pilots.\footcite{time_cold_1961} The precedents from the Baltic Incident still applied: the USSR simply took the necessary immediate countermeasures and shot down the planes that it could.\footnote{While I argue that the response to the U-2 flight was within the reasonable bounds for espionage, I acknowledge that it was still unprecedented in the context of reconnaissance flights. There are a number of reasonable explanations for why the U-2 shootdown escalated. Partially the Soviets felt humiliated by a technology for which they had no countermeasures and the uniquely insulting way the overflights flew deep into Soviet territory. Eisenhower thought that Khrushchev might have been looking for an excuse to cancel his visit to Russia. See: \cite[p.~555]{eisenhower_waging_1965}}

With respect to the practice of aerial reconnaissance, the U-2 Incident changed very little. Khrushchev greatly exaggerated the event's importance until the day he died. He wrote in his memoir that \enquote{the commander of American forces in West Germany gave the order not to fly any closer than 50 kilometers from the border between East and West Germany. And no more incidents of that kind occurred.}\footcite[p.~256]{khrushchev_memoirs_2007} His son Sergei, who annotated the memoir, admitted that this statement was false: \enquote{In practice such incidents occurred again from 1961 onward, but instead of the U-2 the Americans now used various types of Phantom or SR-71.}\footcite[p.~258]{khrushchev_memoirs_2007} On at least two occasions after the U-2 Incident, an American overflight crossed over into Soviet territory and the Soviets responded with a diplomatic protest. The US quickly apologized, and that settled the matter.\footcite{orlov_u-2_2007} The United States continued to use the U-2 extensively in other regions as well.\footnote{A recently declassified CIA report details the many locations in which U-2 photography was employed to gather intelligence after 1960, including China, India, Indonesia, Thailand, Tibet, Laos, North Vietnam, and Venezuela. Lyndon Johnson even clarified that his predecessor's order to end U-2 flights over the Soviet Bloc was not indefinitely binding---it was valid only until countermanded. For more, see: \cite[p.~195]{pedlow_central_1992}}


% What would a major punishment have looked like? One proportional response Khrushchev could have considered was to press harder for the remove of regional air bases. Since the US was using nearby bases to launch invasive reconnaissance missions, the USSR could have demanded that the US either close some of them or agree to halt surveillance flights. This likely would have involved some sort of trade, but it is not an unreasonable hypothetical. The Soviets were incredibly preoccupied with removing the forward-deployed Jupiter missiles from Turkey. \footnote{Although the missiles were deployed to bases well within NATO territory, and arguably less threatening to Soviet security than American overflights of the Soviet bloc, the existence of these missile

% Here's where we get into why Khruschev kinda sorta demonstrates evidence of H2

Both heads of state made clear in their writings that they considered Powers to be spy. On capturing Gary Powers, Khrushchev wrote: ``This was a hostile act by the leaders of the U.S. government, and they made no attempt to conceal it. They didn't think we had the capability of \textelp{} acquiring irrefutable proof that the United States was using methods that were impermissible in peacetime.''\footcite[p.~239]{khrushchev_memoirs_2007} Sergei's annotations once again contradicted his father's claim that these methods were impermissible; more than forty American reconnaissance aircraft were shot down by the USSR during the Cold War. Eisenhower, an insightful thinker on issues of intelligence, understood that beneath the grandstanding the fundamentals of international espionage still applied. \enquote{[Khrushchev's] government,} he wrote, \enquote{had been so notoriously involved in spying, especially in the United States, as to dwarf our activities, but by separating this particular type of espionage from all others he hoped to make convincing his charge of \enquote{warmongering.} To claim that because the equipment employed was an airplane with a camera, and therefore provocative of war, was plain silly, and I felt it necessary that the matter be put in perspective.} \footcite[p.~551]{eisenhower_waging_1965}


\subsubsection{Espionage norms in aerial reconnaissance}

In this section, we see direct evidence of $H_2$ in the actions of the Unite States and somewhat less direct evidence from the Soviet Union. All the way up to President Eisenhower, the United States placed a clear value on coding their actions as espionage, even when undertaking military reconnaissance missions of unprecedented daring. At the same time, American diplomats were willing to resolve disputes that arose from these missions quietly---even when loss of life was involved---which demonstrates the reduced consequences associated with espionage missions. While evidence of intent is hard to come by across the Iron Curtain, Khrushchev appears to have downplayed the incidents as well. The delta between his notes and the actual events suggests he was aware that he would have been expected to respond to these incursions more aggressively. and his son's notes confirm that. The espionage coding was significant to the Soviet apparatus, even if is unlikely to find primary sources that admit it.

That salutary neglect had the intended effect. Project Aquatone channeled the norms of espionage in the most crucial respect: its strategic value easily outweighed the consequences of being caught. No one understood this better than President Eisenhower, who, when he was occasionally questioned about the wisdom of the U-2 flights, would always reply: \enquote{Would you be ready to give back all of the information we secured from our U-2 flights over Russia if there had been no disaster to one of our planes in Russia?} In his telling, he never received an affirmative response.\footcite[p.~559]{eisenhower_waging_1965}

In the aggregate, the period from 1947-1962 provides evidence of $H_1$ as well. The few scholars who study this period argue that the cumulative tension created by these flights contributed to an overall chilly tone in US-Soviet relations. Farquhar, for instance, refers to this as the \enquote{cycle of hostility,} in which the series of aerial incidents increased suspicions of military buildup on both sides, leading to even more aerial incidents as both powers attempted to verify their suspicions.\footcite[p.~43]{farquhar_aerial_2015} While it is true that the there were many aerial incidents, it is not at all clear how this affected any other aspect of US-Soviet diplomacy. The US-Soviet relationship saw huge shifts, including periods of optimism, over the 11 years preceding the U-2 incident. During this period, hundreds of flights took place in contentious territory---many of them purposefully violating Soviet boundaries, causing some of them to be shot down---and the most significant consequence was a failed summit meeting, after which the flights continued on almost as before. What ended this era of aerial reconnaissance wasn't diplomatic pressure---it was an advancement in espionage so massive that it made reconnaissance flights seem quaint.

\subsection{Spy satellites}
Gary Powers had been in Soviet prison for less than three months when the United States and launched an instrument of espionage far more damaging than the U-2 could ever be. The first generation of spy satellite had no technological antecedents, no airspace restrictions, and no countermeasures. The Soviet defense posture in the early 1960s relied on the impression that the Soviet Union possessed more ballistic missile capabilities than it actually did; a spy satellite would quickly reveal that to be a lie. But when the first spy satellite launched in 1962, the diplomatic objections were minimal. Even after developing a feasible anti-satellite (ASAT) weapon, the Soviet Union never made use of it.

The United States and the Soviet Union normalized the use of spy satellites in exactly the manner predicted by $H_2$. First, the United States took extraordinary measures to ensure that its new technology would understood within the context of espionage norms. These measures signaled that the US would be willing to tolerate a reciprocal form of espionage, as long as the Soviet Union was willing to do the same. And once the Soviet Union was able to make its own spy satellites, it ceased objecting to their use. By attempting to \enquote{mark} these satellites as espionage to shield their use from severe diplomatic consequences, the spy satellite programs of both the US and USSR support $H_2$; because it was successful, we can infer that $H_1$ holds as well.

\subsubsection{Freedom of space}

In 1954, President Eisenhower established the Technological Capabilities Panel (TCP), a task force to tackle the problem of America's vulnerability to a surprise attack. When the panel returned with its results, its report concluded that any great power conflict would result in heavy American losses, so all necessary steps should be taken to avoid one.\footcite[p.~67. The TCP report is also variously referred to as the \enquote{Killian Report,} \enquote{Surprise Attack Study,} or its formal title, \enquote{Meeting the Threat of a Surprise Attack.}]{killian_sputnik_1977} Its recommendations included traditional deterrence through force---furthering American missile capabilities---but intelligence-gathering capabilities were prioritized as well. ``We \emph{must} find ways to increase the number of hard facts upon which our intelligence estimates are based,'' the TCP Report states, ``to provide better strategic warning, to minimize surprise in the kind of attack, and to reduce the danger of gross overestimation or gross underestimation of the threat.''\footcite{technological_capabilities_panel_meeting_1955} The TCP recommendations---personally reviewed by Eisenhower---included two projects foundational to the US reconnaissance regime: the U-2 plane and the artificial spy satellite.

The U-2 can take incredibly high-resolution photographs---as long as you already know where to look. A satellite can cover far more ground, photographing wide slices of the planet, uncovering new military installations or verifying that purported ones do not exist.\footnote{In the time it takes a satellite to complete a full orbit, the Earth has rotated slightly underneath it, altering the satellite's field of view. With each pass an artificial satellite launched into polar orbit will photograph an area that has been longitudinally shifted. Imagine winding a piece of string around a ball with your right hand, while slowly rotating the ball with your left; after winding it a number of times, you've covered a broad slice of the ball with string.} Together, planes and satellites operate for complementary photo-reconnaissance purposes: a plane dispatched when high quality localized intelligence was required, and a satellite to form a comprehensive picture of the Soviet military posture. And of course, satellites can surveil dangerous territory without risking the life of a pilot. \enquote{Camera-toting satellites,} historian Walter McDougall explains, \enquote{circling the earth south to north [sic] in a polar orbit, could view the entire surface of the earth as it rotated below, return to any location in a few days' time, hone in on suspicious areas, and do it all under the legal cover of freedom of space---if such legal cover could be established.}\footcite[p.~117]{mcdougall_heavens_1985}

Such legal cover was not guaranteed. The undefined legal status of space presented both an opportunity and a challenge. The Soviet Union would likely recognize the enormous strategic advantage of a photo-reconnaissance satellite and object to its deployment using any international norms possible; even the U-2 overflight program was considered so risky that it was originally intended to last only a year or two.\footcite[p.~33]{lindgren_trust_2000} It was therefore necessary to present reconnaissance satellites in a manner that would limit the grounds for objection. By the time that a reconnaissance satellite launched, it would have to have already been established that any nation had the unquestioned right to launch a satellite of any kind: a principle called \enquote{freedom of space.}

Eisenhower accepted the recommendations of NSC-5520, a memo arguing that a scientific satellite could be used as diplomatic tool to promote a norm establishing ``freedom of space.'' The proposal tracked closely with an earlier report suggesting that the US first launch an innocuous ``experimental satellite'' that did not cross over Russian territory, and gauge the diplomatic reaction.\footcite[p.~21]{kecskemetic_satellite_1950} The political framing of the satellite is discussed at the top of NSC-5520, in a section entitled \emph{General Considerations} (emphasis mine):
\newline

\begin{quoteblock}
6. \textelp{} \textbf{Furthermore, a small scientific satellite will provide a test of the principle of “Freedom of Space.” The implications of this principle are being studied within the Executive Branch.} However, preliminary studies indicate that there is no obstacle under international law to the launching of such a satellite.

7. \textbf{It should be emphasized that a satellite would constitute no active military offensive threat to any country over which it might pass.} Although a large satellite might conceivably serve to launch a guided missile at a ground target, it will always be a poor choice for the purpose. A bomb could not be dropped from a satellite on a target below, because anything dropped from a satellite would simply continue alongside in the orbit. \footcite{nsc_planning_board_draft_1955}
\newline
\end{quoteblock}

The plan was straightforward: use the civilian space program to promote the freedom of space principle, which, once established, would protect the American right to quietly launch a spy satellite.\footcite[p.~119]{day_eye_2015} In fact, since the Soviet Union had already announced plans to launch a satellite, the existence of the Soviet project could be use to blunt the inevitable Soviet protests against American photo-reconnaissance satellites.\footcite[p.~120]{mcdougall_heavens_1985} When the Soviet Union launched the world's first artificial satellite on October 4, 1957, the plan succeeded. The \emph{Sputnik} launch put the United States in an unbeatable negotiating position. The Soviet Union could hardly denounce the American satellite program as suspicious when it had just launched the world's first scientific satellite---without asking permission to orbit over the United States.\footcite[p.~40]{peebles_corona_1997} By allowing \emph{Sputnik} to orbit over American territory without diplomatic objection, the US got to appear magnanimous while fulfilling one of its highest-priority strategic goals.

The available evidence of Eisenhower's internal deliberations overwhelmingly suggests that his administration never intended to be first in space. Launching the world's first artificial satellite was assigned the lowest possible priority because he determined that other goals---finishing the ICBM, monitoring Soviet R\&D, and framing the satellite program as a civilian scientific endeavor---were of greater strategic importance.\footcite[p.~123-124]{mcdougall_heavens_1985} Eisenhower himself repeatedly dismissed concerns about the prestige of being the first nation to orbit an object in space, openly contemptuous of the idea that space was any kind of race at all.\footcite[p.~100]{lindgren_trust_2000} His complete disdain for these ``stunts'' would prove disastrous politically, but his restraint was crucial to establishing a strategically beneficial precedent.\footcite[p.~134]{day_eye_2015}

Just as Eisenhower limited the involvement of the Air Force in Aquatone to paint the U-2 as a civilian project, his guidance led to technical decisions that, consistent with $H_2$, were intended to make the satellite program seem as non-military as possible. The group tasked with implementing NSC-5520 made the surprising choice to commission the unproven Navy Research Laboratory (NRL) \emph{Vanguard} rocket, instead of the Army Ballistic Missile Agency (ABMA) proposal to launch the satellite with better-tested missile technology. Launching the first artificial satellite with a ballistic missile threatened to compromise two national security priorities: it might delay the development of the ICBM, an existential threat to American military superiority, and it would jeopardize the civilian character of the satellite program, which the committee members knew was necessary to establish ``freedom of space.''\footcite[p.~122]{mcdougall_heavens_1985} Led by famous rocketeer (and ex-Nazi) Wernher von Braun, the ABMA engineers vigorously protested the decision, because they believed---correctly---that their proposal would succeed first. But being first in space was not the goal, establishing a legal precedent was. Launching a satellite with no scientific purpose other than demonstrating superiority would compromise the moral high ground which the US desperately needed to secure.\footcite[p.~129-131. Though the NRL is run by the Navy, it functioned like a civilian scientific operation. All parties involved believed that the NRL proposal would be seen as a civilian project---especially when compared with the Army agency building nuclear delivery systems.]{day_eye_2015}

Operationally, \emph{Vanguard} was a disaster. The rocket experienced an embarrassing setback on December 6, 1957, when it failed to launch a three-and-a-half pound satellite, just after the USSR had successfully launched \emph{Sputnik II}.\footcite[p.~119. This time, the satellite carried a dog named Laika. The Soviets put a dog in space before the Americans even managed to get a satellite up there.]{killian_sputnik_1977} When then United States eventually succeeded in launching a satellite, it was thanks to the ABMA team that had been shunned by the administration. Finally given the chance, Von Braun and his engineers quickly accomplished what they had always said they would: modify a Jupiter-C ballistic missile and send a satellite into orbit. The Army launched \emph{Explorer 1} on January 31, and it carried a pioneering array of miniaturized electronics, including two micrometeoroid detectors and a Geiger counter.\footcite[p.~168]{mcdougall_heavens_1985} \emph{Vanguard 1} finally launched soon after.

Choosing the NRL proposal ensured the public humiliation of the American space program---and the private victory of its espionage program. Applying espionage-coding measures pursuant to $H_2$ can be costly: had Eisenhower not enforced the overriding priority to minimize military involvement it is highly likely that the ABMA could have launched a satellite before \emph{Sputnik}. Even though the US did end up launching its first satellite using a ballistic missile, its visible defeat in the space race and clear scientific aims effectively thwarted Soviet efforts to characterize the American satellite program as a military development. \enquote{Freedom of space} had been established---it was time to launch a spy satellite.

\subsubsection{The Corona Project}

% Now that the space age was underway, the next step was to normalize photographic satellite overflights. Success was going to require tight discipline across all levels of government, but information about WS-117L, the Air Force reconnaissance satellite program, had already begun to leak.\footcite[p.~96]{lindgren_trust_2000}

The Soviet Union unintentionally set an early precedent for ``freedom of space,'' but that did not automatically justify the use of outer space for reconnaissance. As far as the international community was concerned, it had only been established that overflights were permissible for scientific purposes; military reconnaissance was still an open question.\footcite[p.~47-48]{peebles_corona_1997} Like many of Eisenhower's space policy decisions, his solution for reconnaissance satellites was a brilliant long-term success---but a complete mystery to the public and utterly baffling to most of the military establishment. Eisenhower simply selected the most immediately promising reconnaissance satellite---a single-use film-retrieval satellite, codenamed Corona---and canceled it. Air Force personnel were ``thunderstruck.''\footcite[p.~45]{peebles_corona_1997} Now that American reconnaissance satellites now seemingly a few years out, the administration began a multi-pronged approach to legitimizing spy satellites. Publicly, the US government advocated for a ``freedom of space'' policy that explicitly included reconnaissance satellites; secretly, it resurrected the Corona project under deep-cover CIA administration.

The public half of Eisenhower's strategy required international approval for space-based overflights, so the US advocated for a United Nations Ad Hoc Committee on the Peaceful Uses of Outer Space (COPUOS) to deal with this issue. The Soviets, worried about having the limitations of their ICBM program exposed, refused to participate for years.\footcite[p.~140]{day_eye_2015} Meanwhile, the actual spy satellite program progressed at an accelerated pace, and with unmatched secrecy.\footcite[p.~51. The team started with only 30 personnel and eventually swelled to 300, almost none of whom knew the entire scope of the project. Team members had to take different routes to work to avoid being followed and were never allowed to say the word ``Corona'' on the telephone, or even the abbreviation ``C.'']{peebles_shadow_2000} When the satellite began making test launches, the US heavily promoted its new \emph{Discoverer} program as a scientific achievement---confident that its status as a front for the Corona project could not be proven false the way the U-2's cover had been. When it finally succeeded in recovering a \emph{Discoverer} satellite from orbit, the US government released photographs of the first successful capsule triumphantly floating in the water, and an American flag retrieved from within was presented to the president.\footcite[p.~83]{peebles_shadow_2000}

Today's Earth-imaging satellites have enormous solar panels, radio their images down to command stations, and have mission durations in a scale of years.\footnote{This design simply did not work with film. A rival project that developed the film midair and transmitted the data back down failed to provide intelligence of sufficient quality. (\cite[p.~203-204]{brugioni_eyes_2010}) The last film-based orbiting reconnaissance camera developed for the U.S. Government was the KH-9 HEXAGON, which operated between 1971 and 1986. It was famously complicated, with four separate film-reentry pods (\cite{pressel_spy_2013}). I presume the rest since then have been digital, but because the HEXAGON was only declassified in September 2011, it will likely be quite a long time before we find out how they work.} Corona was nothing like that. Instead, each satellite was little more than a monstrously expensive disposable camera. The satellite would be sent into space carrying enough film for a 24-hour mission and its camera would switch on every time the satellite flew over denied territory. After completing seventeen orbits, the satellite would pitch down 60 degrees and eject the film capsule back down to Earth. Various rocket systems would stabilize it, a heat shield would protect it, and a series of parachutes would decelerate it. What remained of the capsule would hopefully descend within a 200-by-60 mile recovery zone (``the ballpark''), but at the very least within the boundaries of an additional 400-mile section (``the outfield''). Ideally, it would be caught in midair by a cargo plane.\footcite[p.~56]{peebles_corona_1997}

Why is that significant? Because at no point was satellite reconnaissance a \emph{fait accompli}---each launch was another opportunity for the Soviet Union to object to the continued use of a potentially illegal surveillance method. And there were a great many launches. From February to August 1960, right in the middle of the U-2 incident, twelve test launches failed before the US achieved the first successful recovery of a Corona satellite---of any object from orbit in the human history, for that matter---when it fished \emph{Discoverer 13} out of the sea.\footnote{Ships on standby were able to retrieve \emph{Discoverer 13} even though it fell outside the ballpark. Diplomatically and operationally, the mission was a home run.} The US repeated that feat with \emph{Discoverer 14} a week later, and this time it carried a camera, bringing invaluable intelligence back from space that dealt a mortal blow to the myth of the missile gap.\footcite[p.~101-102. The previous launch failures had been so demoralizing that \emph{Discoverer 13} only carried diagnostic equipment, no cameras.]{lindgren_trust_2000} The launches kept coming---the next one to return high quality photos was \emph{Discoverer 18} in December. Eisenhower left office soon after, but between \emph{Discoverer 18} and \emph{Discoverer 38}, twenty launches were performed, two of which carried actual radiometry equipment (in support of Corona's cover story), and seven produced film of high enough quality for the mission to be considered a complete success.\footcite[p.~103]{lindgren_trust_2000}

The Soviet Union protested the very first \emph{Discoverer} launch, correctly alleging its military utility even though the satellite had failed in its purpose (and would fail eleven more times).\footcite[p.~140]{day_eye_2015} Until 1963, Soviet propaganda continually railed against the Americans, in domestic media and international forums, for attempting to militarize space.\footcite[p.~271]{mcdougall_heavens_1985} The USSR sponsored a UN resolution to prohibit espionage from space, and Khrushchev vowed that his military would shoot down spy satellites, just as they had the U-2.\footcite[p.~166]{day_eye_2015} In the first few years after the United States developed satellite reconnaissance capability, it appeared that the USSR was using all the means at its disposal to end the satellite overflights.

Khrushchev had every incentive to discourage satellite reconnaissance, because the intelligence that the US gained through Corona was absolutely devastating to Soviet security. One of Khrushchev's top national security concerns at this time was concealing the Soviet Union's relative nuclear weakness.\footcite[p.~133]{brugioni_eyes_2010} The moment at which the US discovered that the USSR had far fewer missiles than it claimed was the exact moment at which that disparity was greatest. The missile gap actually went in the other direction---in 1962, the US had more than double the ICBMs, maintained a substantial lead in manned bombers, and had just begun to deploy nuclear submarines.\footcite[p.~251]{mcdougall_heavens_1985} If Khrushchev genuinely feared an American attack, absolutely nothing would have been a greater threat to Soviet security than the collapse of the USSR's most significant deterrent argument---a collapse brought about by American spy satellites. \emph{Discoverer 14} and \emph{Discoverer 18} successfully photographed the sites that intelligence analysts considered most likely to host operational ICBMs, and found nothing.\footcite[p.~379]{brugioni_eyes_2010}


% The U-2 overflights had sufficiently debunked the \enquote{bomber gap,} and seemed to suggest that no \enquote{missile gap} existed either.\footcite{goodpaster_cold_2003} But as the CIA report on U-2 photography showed, being unable to find ICBM sites was not conclusive proof that they did not exist, and the vast majority of possible missile sites remained un-photographed. Confusion about Soviet long-range attack capabilities still existed at various levels of government, confusion which ultimately led to the doomed flight of Gary Powers.\footcite[p.~344]{brugioni_eyes_2010}


Not only did Corona make a theoretical offensive attack more likely, it made the plans for such an attack significantly more effective. The impact of the project on military planning is best summarized by how it affected the joint American-British plan for an offensive against the Soviet Union, which aerospace historian Curtis Peebles describes here:

\begin{quoteblock}
Starting in 1962, the Corona photos were also used for nuclear targeting. Corona photos provided the first accurate coordinates of Soviet military facilities. This resulted in a shift in U.S. policy. During the 1950s, little was known about the location of Soviet military bases. The lack of information was reflected in the first join SAC-RAF Bomber Command strike plan. The plan, which went into effect on October 1, 1958, assigned 106 Soviet targets to the RAF bomber force: 69 cities, 17 airfields, and 20 air defense targets. A full two-thirds of the targets were cities. This disparity was ever greater with the unilateral RAF strike plan. If Great Britain should be forced to attack alone, ninety-eight Soviet cities would be bombed.

On August 1, 1962, a revised joint plan went into effect. The changes, which reflected a full two years of Corona operations, were profound. The RAF Thor IRBMs and bomber force would strike forty-four ``offensive capability'' targets such as airfields, ten ``defensive capability'' targets such as air defense control centers, twenty-eight Soviet IRBMs sites, and only sixteen cities. Urban targets had gone from the majority to next to last in number. The unilateral British plan also underwent a similar change.\footcite[p.~139]{peebles_corona_1997}
\end{quoteblock}

The Soviet Union lost its most significant military advantage over the West in a little under two years. No subsequent missile buildup or bunker hardening or second-strike capability could undo the damage done to the Soviet military posture when \emph{Discoverer XIV} and its successors sailed past the Iron Curtain. Aerial reconnaissance neutralized one of Soviet Union's asymmetric structural advantages---the inherent secrecy of Soviet society---and put a dent in its claims to an arms advantage as well. And yet the only thing the Soviet Union did in response was publicly protest the spy satellites, and refuse to accept American proposals for a legal regime that included the rights of reconnaissance overflights. Those protests continued for about 3 years, until the Soviets developed their own working spy satellites, and ``freedom of space'' became a mutually-agreeable reality.\footcite[p.~271-275]{mcdougall_heavens_1985}

\subsubsection{Espionage norms in satellite reconnaissance}
The American pursuit of spy satellites cleanly demonstrates the core premise of $H_2$. President Eisenhower afforded an attention to espionage uncharacteristic even for the leader of a Great Power, and his intent is reflected in the actions of the state to an unusual degree. The United States also incurred significant costs for diverting resources to signal that its satellite activities constituted espionage, highlighting the contrast with the more direct, militaristic approach. Though there is less direct evidence of its intent, the Soviet Union functionally applied the same playbook when it launched its own satellites a few years later---Khrushchev even claimed in 1960 that the Soviet Union had already put cameras in some of its satellites.\footcite[p.~353. As far as we know, these camera-equipped satellites, if they actually existed, were not successful enough to be considered spy satellites.]{brugioni_eyes_2010} The effective cessation of diplomatic complaints regarding spy satellites, from either side, is the expected outcome from two sides that are both acting to code their actions as espionage. The $H_1$ hypothesis is implied by exactly this situation, in which $H_2$ is reciprocal.

Alternate explanations for why the Soviet Union allowed the legitimization of spy satellites rely exclusively on one's faith in diplomatic norms. Most boil down to gamesmanship: the United States effectively leveraged scientific diplomacy and covert operations; \emph{Sputnik} made it much harder for the USSR to argue that reconnaissance satellites were an imperialist invasion of territorial sovereignty; Nikita Khrushchev, in a heated moment at the 1960 summit, was said to have claimed that ``any nation in the world who wanted to photograph the Soviet areas by satellite was completely free to do so.''\footcite[p.~556]{eisenhower_waging_1965} None of these explanations, on their own, fully capture why the Soviet Union would assent to an espionage regime that asymmetrically neutralized one of its most significant security advantages.

Each of the circumstances surrounding the normalization of spy satellites takes on an elevated significance if one allows for $H_1$ to be true---that states place importance on the \emph{appearance} of espionage and adjust their diplomatic consequences accordingly. Applying reduced consequences to espionage, while itself certainly a norm, has realist implications as well. By explicitly accepting the American justification for their reconnaissance satellites, at great cost to it security, the Soviet Union was signaling that the reciprocal tolerance of espionage was a pact that it is willing to preserve. When the United States allowed the Soviet Union to launch its own spy satellites a few years later, the mutual flow of intelligence was assured, along with the security benefits that both states derive from it.

\section{Conclusion}
The promise of an international community devoid of secrets is, at its core, a utopian promise. Complete, universal verification of capabilities and intents would make it impossible for an adversary to maintain hidden military capabilities or furtive plans for war; theoretically, if applied to our current system of nation-states, bad actors would be rendered obsolete. While diplomacy based on perfect information is impossible right now---and likely always will be---espionage norms that reduce the diplomatic impact of intelligence gathering are one means to approach a truly open society.

In order to best demonstrate the novel theory of this paper, I focused on a uniquely evidence-rich context: two nuclear superpowers who both maintained robust intelligence services. This is, of course, not the situation in which most espionage norms live---the stakes might be lower, the capabilities different, or the overall balance of power more lopsided. For example, the relationship between the espionage services of Israel and Iran could look very different from the US-USSR espionage relationship examined here, as could the relationship between the US and present-day Russia. This paper does not claim to consider all these variations, rather, it lays a foundation for future analysis of how other variables impact the espionage norms observed.

At scale, espionage removes risk from the international system, and the rare instances where espionage is codified suggests that states generally view espionage this way as well. The Strategic Arms Limitation Treaty of 1972 (otherwise known as The ABM Treaty, or SALT 1) enshrined in law that signatories are not permitted to interfere with each others' ``national technical means of verification,'' a tacit ban on ASAT weapons.\footcite[p.~431]{mcdougall_heavens_1985} Without having to say ``spy satellites,'' SALT 1 made them not just legal, but essential. The peaceful uses of aerial reconnaissance were formalized at the end of the Cold War, when members of NATO and the Warsaw Pact signed the \enquote{Treaty on Open Skies,} which established guidelines for unarmed aerial surveillance flights in the spirit of open information and de-escalation.\footcite{organization_for_security_and_co-operation_in_europe_treaty_1992} This treaty was almost identical to the one first suggested by Eisenhower in 1955.\footcite{center_for_arms_control_and_non-proliferation_fact_2017} With certain restrictions, the type of flight that altered the course of the Cold War is now officially routine and permissible.

Even amongst the superpowers, these norms are not inviolate, and they are always subject to defection. In May 2020, President Trump announced his intent to withdraw the United States from the Open Skies Treaty.\footcite{sanger_trump_2020} If the withdrawal takes place, we can expect that Russia will increasingly deny access to its airspace as well. China destroyed one of its own satellites with an ASAT weapon in 2007, which a Harvard Astronomer described as ``the first real escalation in the weaponization of space that we’ve seen in 20 years.''\footcite{broad_china_2007} India successfully tested its own ASAT in March 2019.\footcite{gettleman_india_2019} Though the threat of weaponization continues to escalate, the US still refuses to pursue an official ban on weapons on space.\footcite{oconnor_u.s._2019}

The value of an international community based on perfect information is still under contention. Despite recent worrying developments, at the time of writing no ASAT has even been used against an adversary's satellite, and the Open Skies treaty remains precariously in effect. Maintaining these norms requires formulating a positive case for their utility, and quickly. The need to present this case in a compelling manner will only become more urgent.

\end{document}

% There is a fundamental appeal to a world with tacit mutual reconnaissance, an appeal that is rooted in defensive realism and manifested in espionage norms. Spy satellites are the single best proof that even when the relative gains are lopsided, states will choose to live in a world with more information, rather than less. The advent of mutual reconnaissance solved what had, until then, been the single most intractable problem of the Cold War---treaty verification. Rae Huffstuttler, later a director of the CIA's National Photographic Interpretation Center (the same institution that conducted the survey of U-2 photography), said that ``imagery set the stage for the arms limitation talks \textelp{} With imagery, we could go to the numbers-based strategic arms limitations negotiation with a high degree of confidence.''\footcite[p.~403]{brugioni_eyes_2010} With satellite imagery in hand, the United States could enter the negotiations with appropriate estimates and leave with a means of verifying that the treaty had been carried out. Khrushchev claimed for years to be in favor of arms control, but talks always collapsed because of the US demand for on-site inspections, which the Soviet Union always refused.\footcite[p.~255]{mcdougall_heavens_1985} Satellites served the same purpose, and the Strategic Arms Limitation Treaty of 1972 (otherwise known as The ABM Treaty, or SALT 1) enshrined in law that signatories are not permitted to interfere with each others' ``national technical means of verification,'' a tacit ban on ASAT weapons.\footcite[p.~431]{mcdougall_heavens_1985} Without having to say ``spy satellites,'' SALT 1 made them not just legal, but essential.

% In February 1962, Francis Gary Powers walked across a bridge into West Berlin at the same time as Russian Colonel Rudolph Abel, an American prisoner. Powers' father had suggested to the State Department that they might be traded, spy for a spy.\footcite[p.~239]{powers_operation_2004} \enquote{This,} Eisenhower wrote, ``was a tacit admission by Khrushchev that our `outrageous' U-2 pilots have their opposite numbers operating within the borders of this country.''\footcite[p.~558]{eisenhower_waging_1965}

