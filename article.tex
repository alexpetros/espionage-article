\documentclass[14pt]{extarticle}
\usepackage{style}


\title{Draft Article}
\author{Alexander W. Petros}
\date{September 2019}

\addbibresource{article.bib}

\begin{document}
\maketitle
% \newpage

The most damaging cyberattack committed against the United States government by a foreign agent was remarkably subtle. Starting sometime in 2014, a small probe embedded itself deep within the network of the US Office of Personnel Management (OPM), an office of about 5,000 employees tasked with overseeing background checks, payroll, and other human resources concerns for the federal government. At regular intervals, the probe would send updates to opmsecurity.org, a domain registered to look like an official website, but not actually affiliated with the US government or the OPM security team. Unobtrusive and observant, the probe functioned in all respects like a digital spy.

By the time the breach was discovered in April 2015, the probe had accessed sensitive government databases and transmitted the stolen information back to its country of origin---including a trove of applications for federal security clearance. These applications exposed not just the personal data of the applicants themselves but also the detailed information they had supplied about their family and friends, including social security numbers, job applications, and home addresses. Federal authorities determined that the hack ultimately affected 22.1 million people, including personnel at the highest levels of government.\footcite{nakashima_hacks_2015} When OPM announced the full extent of the damage, it caused an uproar. The American Federation of Government Employees, the largest federal workers union, filed a class-action lawsuit against OPM, seeking damages under the Privacy Act.\footcite{chalfant_court_2017} Senator Mark Warner (D-Va.) of the Senate Intelligence Committee called for OPM Director Katherine Archuleta's resignation. Because the information might be used for identity theft, the US government offered all affected employees free credit and identity monitoring services for three years.

Not only was the OPM hack a public relations disaster, it had significant national security implications as well. US officials feared that the stolen data could compromise intelligence efforts by exposing undercover CIA officers working in embassies around the world, whose names would be suspiciously absent from OPM records. The information could also be used to identify American officials who might be susceptible to pressure, or worse, recruitment. FBI Director James Comey personally took questions about the incident. ``If you have my [application for security clearance],'' Comey said, ``you know every place I've lived since I was 18, contact people at those addresses, neighbors at those addresses, all of my family, every place I've traveled outside the United States. Just imagine if you were a foreign intelligence service and you had that data.''\footcite{nakashima_hacks_2015} With evidence that a foreign intelligence service had successfully obtained exactly that, one might expect that the United States would respond with a series of diplomatic measures intended to signal that highly invasive acts of espionage would not be tolerated from its international adversaries.

Instead, the opposite happened. US officials would only confirm that they believed a \enquote{foreign entity or government} had been behind the attack, even though it was widely known that they suspected the involvement of the Chinese government.\footcite{spetalnick_china_2015} Internal investigators quickly determined that OPM had been compromised by an Advanced Persistent Threat (APT), a formally organized and typically state-sponsored group of hackers that engage in long-term, targeted penetration operations. In this case, the APT made identification easy by leaving their calling-card: the trojan opmsecurity.org domain was registered in the name of \enquote{Steve Rogers,} the Marvel comics character better known as Captain America. This digital taunt, along with email and IP addresses that verified the country of origin was not a false flag, all but guaranteed that the attack had come from an APT sponsored by the Chinese government, likely the Chinese military's cyber-espionage division.\footcite{koerner_inside_2016} Though it possessed conclusive digital forensics, the Obama administration nonetheless refused to name China as the culprit.\footnote{National Security Advisor John Bolton was the first American official to formally acknowledge that the Chinese were behind the OPM hack, three years later, in September 2018. See \cite{sanger_trump_2018}}

The US government's tepid response to the OPM hack illustrates a crucial fact about the relationship between cybersecurity and espionage: when a cyberattack appears to fall within the bounds of a traditional intelligence operation, it will merit the diplomatic consequences of a traditional intelligence operation. US policymakers rarely acknowledge that this distinction exists---officially, all cyberattacks are illegal international behavior and will receive a strong response---but occasionally, in moments of candor, they will admit that cyber-espionage is different. A senior White House official told the Congressional Research Service that \enquote{there is a vast distinction between intelligence-gathering activities that all countries do and [other forms of malicious cyber-activity].}\footnote{In the context of the quote, the White House official is distinguishing between traditional espionage and economic espionage, the latter of which is the state-sponsored theft of trade secrets from foreign private companies in order to gain a competitive advantage in international markets. See \cite{finklea_cyber_2015}} For activities that fall under the umbrella of intelligence-gathering, counterintelligence, not prosecution, is deemed to be the appropriate response. Director of the CIA John Brennan described foreign spying on domestic political institutions as \enquote{fair game.}\footcite{sanger_u.s._2016} And while speaking at an intelligence conference, James Clapper, Director of National Intelligence, was particularly frank: \enquote{You have to kind of salute the Chinese for what they did. If we had the opportunity to do that, I don’t think we’d hesitate for a minute.}\footcite{pepitone_clapper_2015}

% latex table generated in R 3.4.3 by xtable 1.8-2 package
% Sat Aug 24 13:46:46 2019
\begin{table}[ht]
\centering
\begin{tabular}{lrrr}
  \hline
Type & Incidents & Official responses & Response \% \\
  \hline
Defacement &   3 &   2 & 66.67 \\
  Data destruction &   8 &   5 & 62.50 \\
  Doxing &   6 &   3 & 50.00 \\
  Sabotage &  14 &   5 & 35.71 \\
  Denial of service &  17 &   6 & 35.29 \\
  Espionage & 231 &  44 & 19.05 \\
   \hline
\end{tabular}
\caption{Cyber-incident response rates by type of incident}
\label{response-pct}
\end{table}

In fact, cyber-espionage is routinely treated as a lesser category of offense by all nations, not just the United States. Of the hundreds of state-sponsored cyberattacks about which there is public knowledge, attacks classified as espionage are by the far most common---and the least likely to garner an official response from the victim government (Table \ref{response-pct}).\footnote{These numbers were compiled from the Council on Foreign Relations' Cyber Operations Tracker, which is a publicly-sourced record of state-sponsored cyberattacks, updated quarterly. Of the 323 listed incidents, 38 were excluded from this calculation because they lacked an identifiable Victim, Sponsor, or Type. For more, see \cite{council_on_foreign_relations_new_2019}} The routine minimization of espionage is most easily observed through the plethora of modern cyber-incidents, but it has its roots in norms that are much older.

At first glance this observation might seem almost tautological. Of course cyber-espionage is treated like traditional espionage, one could argue---they're both espionage. That formulation, however, begs the question; it doesn't explain how states determine what constitutes \enquote{traditional} espionage in cyberspace, what purpose that distinction servers, or whether it is prudent to continue making it. Certainly the US should respond proportionally to attacks on its systems, but what if the damage done to national security by espionage is especially significant? ``This is one of those cases where you have to ask, `Does the size of the operation change the nature of it?'\thinspace'' said one anonymous senior intelligence official of the OPM hack. ``Clearly, it does.''\footcite{sanger_u.s._2015} As as far as the public knows, however, the size of operation did \emph{not} change the nature of the diplomatic response. Therefore, to understand why the Director of the CIA treats cyber-espionage as \enquote{fair game,} one first needs to understand why traditional espionage has historically that treatment as well.

I posit that universal tolerance of espionage in the international system can be explained through a rational calculation of each state's incentives. On balance, states derive a benefit from preserving a system in which the consequences for espionage are minimized, so they refrain from imposing particularly harsh punishments for its practice. Even when an individual incident is especially damaging to a state's security, if it falls within the traditionally accepted boundaries of an espionage operation, it will likely go unpunished. In the case of the OPM hack, for example, US policymakers decided they could not impose harsh penalties on the Chinese because it was more valuable for the US to retain its ability to practice cyber-espionage with relative impunity in the future. This basic calculation---that espionage against one's own country is a tolerable price to pay for being allowed to practice espionage---has remained consistent since the advent of the modern peacetime intelligence service, perhaps even the nation-state itself.

This raises a fascinating paradox: for espionage to be mutually permissible, both states must agree that it is so. But that requires \emph{both} states to determine they are getting more from practicing espionage against their adversary than their adversary is getting from practicing it against them. Otherwise, one would simply defect from that equilibrium and punish espionage at a level that imposes direct costs. Since this generally doesn't happen between major world powers, one of two things must be true: either a state is miscalculating the relative gains it gets from espionage, or the mutual tolerance of espionage has benefits that are sufficiently important to maintain even if one state gains less from it than its adversary. I argue that the latter is true.

This article seeks to establish a theoretical framework for analyzing the role that espionage plays in international relations (IR) and the diplomatic norms that are associated with it. Through this framework, I explain what constitutes traditional espionage and why rational states might calculate that it is in their best interests to continue tolerating it. I then examine two cases in which new technical means of gathering intelligence were introduced during the Cold War: aerial photography (spy planes) and satellite photography (spy satellites). In both cases, the norms of traditional espionage ultimately prevailed, minimizing the diplomatic consequences for their use as long as it was in service of traditional intelligence goals. Through process tracing, I demonstrate how the justifications policymakers employed for the use of these intelligence technologies---and the resulting minimization of the diplomatic penalties associated with them---are consistent with the theoretical underpinnings for how a tacit accord permitting general espionage might be mutually strategic.

\subsection{Existing literature}
The benefits of espionage have never been underrated---\enquote{it is only the enlightened ruler and the wise general,} reads Sun Tzu, \enquote{who will use the highest intelligence of the army for purposes of spying and thereby they achieve great results}---but the inevitability of espionage has long been taken for granted. If you question that inevitably, the assertion that espionage is something \enquote{all counties do,} then the way that countries treat espionage becomes considerably more confusing. Until now, few, if any, scholars have analyzed why espionage norms exist and what strategic purpose they serve.

Before continuing, it is necessary to properly define what I mean by espionage norms. Simply put, espionage norms dictate what constitutes espionage and how states are expected to respond when it is conducted against them.\footnote{There are, of course, other \enquote{norms} that involve espionage. The so-called \enquote{Moscow Rules} dictate how an agent or case officer is supposed to conduct themselves in hostile territory. In order to avoid constantly specifying, espionage norms are more strictly defined to include only the ones that affect IR.} What consequences does a state impose when it uncovers an espionage operation run by another state? Espionage norms prescribe that the consequences remain minimal. Rarely does the response amount to more than a pat denunciation of the activity and implementation of the necessary countermeasures to end it---the consequences for espionage are borne by the agents who commit it, not the state that sent them. In the rare cases when additional costs \emph{are} imposed on a state for conducting a traditional espionage operation, they are not costs that would discourage the aggressor from attempting such an operation again. In this article I present a theory for why those norms exist.

Prior literature has attempted to explain espionage norms using international law, but in the absence of any significant treaties governing peacetime espionage, scholars are often left without definitive answers.\footnote{Because it is universally agreed that international law does not deal with peacetime espionage directly, many journal articles that deal with this question frame their work almost Socratically, devoting two sections to cases for and against the legality of espionage. In fact, the first major work on law and peacetime espionage is a series of commentaries collected in a single publication that respond to each other's arguments. See \cite{wright_essays_1962}} Since espionage is illegal everywhere, how can every state acknowledge that allies and adversaries alike are guaranteed to engage in it?\footnote{Further complicating matters, espionage is the rare crime that might not be a crime depending on who the victim is. If an American government employee walks out of The Pentagon and hands classified documents to the KGB, they are at risk prosecuted as a traitor; if an American case officer hands classified KGB documents over to the CIA, they will be lauded as a hero. This example, in various forms, is frequently used to illustrate the impossibility of creating a consistent legal doctrine for espionage.} There is no legally consistent way to resolve that paradox, so scholars have typically treated espionage as \emph{sui generis}---starting from the assumption that espionage norms exist in a class of their own, then attempting to fit a legal doctrine around that fact. Jared Beim determines that there while peacetime espionage is often tolerated, it clearly constitutes an \enquote{intervention} as defined by the International Court of Justice, so the prohibition on espionage merits further enforcement.\footnote{Beim also suggests that a weaker country could seek justice for an espionage operation by appealing to international organizations like the United Nations or the Council of Europe. For more, see \cite{beim_enforcing_2018}} Beim acknowledges, however, that his conclusion will have little impact on the practice of espionage, where, at best, \enquote{it is perhaps possible to stop the more egregious violations.}\footcite[p.~672]{beim_enforcing_2018} A. John Radsan employs a more philosophical approach. Suggesting that we should resist the \enquote{Hegelian impulse} to resolve the tension between espionage and international law, Radsan concludes that \enquote{beyond any international consensus, countries will continue to perform espionage to serve their national interests \textelp{} International law does not change the reality of espionage.} Legal scholars of espionage all reach the same conclusion: it would be almost impossible to argue that espionage is legal, but it will likely continue to be downplayed. Beyond simple inertia, however, this literature cannot explain why espionage is tolerated.

Another branch of scholarship sidesteps the legal question in favor of a \enquote{functional theory} of IR. This is closer to my work, as it attempts to explain \emph{why} states tolerate espionage rather than creating \emph{post hoc} justifications for it. A functional approach, for instance, might highlight the role of espionage in treaty verification, the way Christopher Baker does when he argues that \enquote{the advantages that espionage offers over legally-binding verification and assurance regimes tip the scales in favor of functional cooperation.}\footcite{baker_tolerance_2004} Baker correctly notes that the additional assurance provided by espionage enables states to enter into riskier agreements that they would otherwise have avoided. His analysis is limited, however, to arenas in which espionage may facilitate various forms of interstate cooperation. It doesn't resolve the gap between the absolute illegality of espionage in domestic law and its widespread practice internationally. Responding to Baker, Radsan notes that one-off deals and treaties \enquote{are a very slow and indirect way to achieve international consensus on the legality of espionage. The gap is still there.}\footcite[p.~607]{radsan_unresolved_2007} This article expands on Baker's basic premise---that there are situations in which espionage is mutually advantageous for both parties---and universalizes it to explain why the entire institution of espionage is mutually advantageous. Espionage norms provide the glue that keep the institution in place, despite its legal ambiguity.

The key insight this article presents---and where it breaks from previous literature---is that far from being an irreplicable, historically contingent set of practices, espionage norms are actually preserved through an ongoing set of strategic decisions. Each time a new espionage technology is made available, policymakers are given a fresh chance to decide whether or not the norms of espionage ought to extend to its use. Critically, the research that follows will show that they were not obligated to do so by any pre-existing norms. When the US first introduced photographic satellites, for example, it was not clear whether the Soviet Union would treat the satellites as spies or as weapons---there were indications in both directions. Ultimately, the infrastructure of what we now call \enquote{spy satellites} was developed, and the norms of espionage prevailed to protect their use.

% Because I argue that espionage norms influence cyber policy, it is worth mentioning scholarship on the role that espionage, outside of cyberspace, plays in the international system today. The intersection of espionage and international relations has been studied before---beginning, in the modern context, as far back as 1962---but usually in a historical context, not a political science one. The role of espionage in deescalating conflict is typically relegated to accounts of moments where Great People properly interpreted---or fatally misunderstood---the information that their intelligence operations provided them. Chapters 3 and 4 will primarily focus on decisions made by Eisenhower and Khrushchev, so to some extent I am guilty of that as well. Nonetheless, I will try to highlight the way in which both of those leaders set precedents designed to outlast them, across a variety of intelligence methods. I hope that this thesis starts the conversation about how espionage norms operate to deescalate conflict independent of the leaders that promote them.

% By employing both realist and constructivist schools of IR, this article develops rigorous explanations for why rational states, whether they be security-seeking and greedy, might cooperate to preserve this dynamic.

% incorporating cyber espionage into the intelligence literature, rather than the cyebr literature
% espionage first, cyber second

\section{Theory}
\subsection{Establishing the puzzle}
One of the most prevalent clich\'es in spy literature and journalism is the so-called \enquote{rules of the game.} Media-friendly lists of rules like \enquote{You are never alone} and \enquote{Don't trust anyone} make the spy world seem insular, mysterious, and impenetrable.\footcite{myre_moscow_2019} They hint at a larger set of processes whose motivations the average citizen, not given access to this privileged information, could not possibly begin to understand. In reality, while the \emph{operations} of a modern intelligence service are incredibly opaque, the geopolitical conditions in which one operates are not. Every aspect of a nation-state's interactions with another ought to be subject to the same critical analysis. Scholars can use rationalist principles to analyze everything from trade policy to the space race; espionage, which by definition involves deeply invasive interactions with another state, ought to be no different.

% Certainly there are factors that the go into diplomatic decisions that the general public will likely never know. There are few reasons to believe, however, that this is more true for intelligence-related diplomatic matters than for generic ones. Every diplomatic decision is made with the backing of classified intelligence.

Under traditional methods of analysis, espionage certainly seems like something that states should be doing more to discourage. Then results of intelligence operations can have devastating, even existential effects on the targeted nation's security. One of the most traditional form of espionage is stealing military secrets from a foreign power, and in that arena China has been relentless. A US Navy report from March 2019 declared Chinese intelligence operations so extensive that they had substantively altered the balance of power between the two states.\footcite{lubold_navy_2019} ``Long-term, US future military advantage is being diminished by years of IP exfiltration from the DoD, DoN, and DIB,'' the report read, ``all with little to no adverse consequences to the thieves.''\footcite[p.~6]{bayer_cybersecurity_2019} The Navy's forecasts were dire: ``If the current trend continues unimpeded, the US will soon lose its status as the dominant global economic power.''\footcite[p.~5]{bayer_cybersecurity_2019} The Navy is not a disinterested party, of course, and there are reasons to take its conclusions with a grain of salt; modern military technology is remarkably complex and China cannot simply steal its way to military dominance.\footnote{For a good discussion of these limitations, see \cite{gilli_why_2019}} Nonetheless, the possibilities are alarming and the consequences for remain minimal.

Not only does espionage have potentially disastrous consequences but the system is set up so that the structure of civilian intelligence services is never fundamentally threatened. The low percentage of responses for cyber-espionage have their basis in a similarly low response rate for traditional espionage. The traditional punishment for espionage is to declare certain embassy officials \emph{persona non grata}, which often results in ``tit-for-tat'' diplomatic expulsions. The name is taken from a highly-successful strategy for the prisoner's dilemma wherein the player simply mimics the choice that that their opponent made in the last round. By employing the tit-for-tat strategy, the player never gets taken advantage of for more than one turn and can always return to cooperation if their opponent unilaterally signals a willingness to do the same.

Diplomatic expulsions play out in exactly the same fashion: when one country expels the diplomats of another, the other country will often respond in kind. Each side will continue to expel each other's diplomats until one side signals that it would like for the reciprocal PNG-ing of diplomats to stop. Then, both sides return to a cooperative state, each preserving some of their intelligence operations. Neither side can really gain a decisive advantage, so the back-and-forth punishment never ends in a mutual defection. It might seem obvious to note, but \emph{both} states must have decided that preserving this system is valuable for them, otherwise just one of them refusing to cooperate would result in the mutual devastation of human intelligence operations.

\subsection{Espionage norms as a discrete series of choices}
Recall: the key feature of espionage norms is that the scope of the punishment is structurally incapable of discouraging a future, identical operation. Expelling suspected intelligence officers from an embassy temporarily makes it more difficult to conduct espionage, but it doesn't make a state less likely to try; the worst that can happen is a net-neutral result---going back to \enquote{square one.} By contrast, were it more common to respond to espionage by imposing financial sanctions or even inflicting a military cost, a state would have to consider the possibility that attempting espionage could adversely impact other aspects of their security, leaving it worse off than when it started. Such a response would be a violation of the well-established norms of espionage and it is essentially unheard-of.

Therefore, I propose that an iterated model can describe how states cooperate to preserve the liminal legal status of espionage. My model changes the parameters from above so that the decision to cooperate or defect is based not in the amount of punishment received (essentially measured by the number of diplomats expelled), but rather the type. Each time a state is forced to respond to a successful intelligence operation by a rival---such as the reveal of a high-level spy or a major data breach---it can either limit its response to that which is traditionally associated with espionage, or it can impose unprecedented costs that might discourage such an operation in the future. Table \ref{espionage-matrix} describes a single instance of this choice. States signal their interest in cooperating by limiting their retaliation in response to espionage, no matter how damaging the operation is to their national security. Each instance is a chance to play the game with fresh information; each espionage case is an opportunity to cooperate, reaffirming the norm that states always respond to espionage in a specific way, or defect, adding an additional cost to the attempt.

% https://tex.stackexchange.com/questions/249480/creating-a-payoff-matrix-using-latex-tabular-environment
\begin{table}[ht]
\centering
\setlength{\extrarowheight}{2pt}
\small
\begin{tabular}{cc|c|c|}
  & \multicolumn{1}{c}{} & \multicolumn{2}{c}{USSR}\\
  & \multicolumn{1}{c}{} & \multicolumn{1}{c}{Tolerate Espionage}  & \multicolumn{1}{c}{Punish Espionage} \\\cline{3-4}
  \multirow{3}*{USA}  & Tolerate Espionage & \makecell{~\\Both sides gain intelligence \\~} & \makecell{~\\ Only the USSR continues spying \\ ~} \\\cline{3-4}
  & Punish Espionage & \makecell{~\\ Only the US continues spying \\~} & \makecell{~\\ No intelligence either direction \\~} \\\cline{3-4}
\end{tabular}
\caption{Espionage consequences decision matrix}
\label{espionage-matrix}
\end{table}

The purpose of analyzing espionage norms this way is to remove the mystique.

Even though the US and the USSR have the opportunity to defect by imposing harsher punishments in response to espionage, the gains would be short-lived, because the other state is guaranteed to immediately do the same---tit-for-tat. Espionage consequences are an iterated game that both parties are playing the exactly the same way, so the only two possible outcomes long-term are mutual defection or mutual cooperation.

Given those two options, it makes intuitive sense that in the case of Cold War espionage both the US and the USSR chose to cooperate. Both sides gained a great deal from their intelligence operations; Soviet intelligence had an easier time operating in Washington than the US did in the USSR, but American intelligence was still formidable and Soviet defections were frequent.\footcite[p.~380]{macrakis_technophilic_2010} When the relative capabilities are not too lopsided, there is a pleasing reciprocity to espionage that makes cooperation easier.

% first it will lay out how espionage is considered an occupational hazard in the business of running a nation-state, one that is counteracted but never discourages.

% next I will discuss the theoretical justifications for why this might be the case. security seeking states and types of greedy states derive a reational benefit from allowing espionage to continue.

% finally I will look at historical moments in which new technologies were deployed for espionage-like purporses, and demonstrate how the norms governing the use of these technologies always evolved to permit espionage.

\end{document}
