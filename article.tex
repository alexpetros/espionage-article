\documentclass[14pt]{extarticle}
\usepackage{style}


\title{Draft Article}
\author{Alexander W. Petros}
\date{September 2019}

\addbibresource{article.bib}

\begin{document}
\maketitle
% \newpage

The most damaging cyberattack committed against the United States federal government by a foreign agent was remarkably subtle. Starting sometime in 2014, a small probe embedded itself deep within the network of the US Office of Personnel Management (OPM), an office of about 5,000 employees tasked with overseeing background checks, payroll, and other human resources concerns for the federal government. At regular intervals, the probe would send updates to opmsecurity.org, a domain registered to look like an official website, but not actually affiliated with the US government or the OPM security team. Unobtrusive and observant, the probe functioned in all respects like a digital spy.

By the time the breach was discovered in April 2015, the probe had accessed crucial government databases and transmitted the stolen information back to its country of origin---including a trove of applications for federal security clearance. These applications exposed not just the personal data of the applicants themselves but also the detailed information they had supplied about their family and friends, including social security numbers, job applications, and home addresses. Federal authorities determined that the hack ultimately affected 22.1 million people in total, including personnel at the highest levels of government.\footcite{nakashima_hacks_2015} When OPM announced the full extent of the damage, it caused an uproar. The American Federation of Government Employees, the largest federal workers union, filed a class-action lawsuit against OPM, seeking damages under the Privacy Act.\footcite{chalfant_court_2017} Senator Mark Warner (D-Va.) of the Senate Intelligence Committee called for OPM Director Katherine Archuleta's resignation. Because the information might be used for identity theft, the US government offered all affected employees free credit and identity monitoring services for three years.

Not only was the OPM hack a public relations disaster, it had significant national security implications as well. US officials feared that the stolen records could compromise intelligence efforts by exposing undercover CIA officers working in embassies around the world, whose names would be suspiciously absent from the OPM records. The information could also be used to identify American officials who might be susceptible to pressure, or worse, recruitment. FBI Director James Comey personally took questions about the incident. ``If you have my [application for security clearance],'' Comey said, ``you know every place I've lived since I was 18, contact people at those addresses, neighbors at those addresses, all of my family, every place I've traveled outside the United States. Just imagine if you were a foreign intelligence service and you had that data.''\footcite{nakashima_hacks_2015} With evidence that a foreign intelligence service had successfully obtained exact that, one might expect that the United States would respond with a series of diplomatic measures intended to signal that highly invasive acts of espionage would not be tolerated from its international adversaries.

Instead, the opposite happened. US officials would only confirm that they believed a \enquote{foreign entity or government} had been behind the attack, even though it was widely known that they suspected the involvement of the Chinese government.\footcite{spetalnick_china_2015} Internal investigators quickly determined that OPM had been compromised by an Advanced Persistent Threat (APT), a formally organized and typically state-sponsored group of hackers that engage in long-term, targeted penetration operations. In this case, the APT made identification easy by leaving their calling-card: the trojan opmsecurity.org domain was registered in the name of \enquote{Steve Rogers,} the Marvel comics character better known as Captain America. This digital taunt, along with email and IP addresses that verified the country of origin, all but guaranteed that the attack had come from an APT sponsored by the Chinese government, likely the Chinese military's cyber-espionage division.\footcite{koerner_inside_2016} Sitting on conclusive digital forensics, the Obama administration nonetheless refused to name China as the culprit.\footnote{National Security Advisor John Bolton was the first American official to formally acknowledge that the Chinese were behind the OPM hack, three years later, in September 2018. See \cite{sanger_trump_2018}}

In recent years, cyberattacks perpetrated by state-sponsored foreign agents have received ever-increasing levels of public scrutiny---which makes the government's response to the OPM hack especially puzzling. Russian interference in the 2016 US Presidential election prompted the expulsion of 37 diplomats and series of follow-on sanctions.

% This article seeks to explain why espionage efforts are not more actively discouraged by nations, even as the results of intelligence operations can oftentimes have a devastating effect on the targeted nation's security.

% first it will lay out how espionage is considered an occupational hazard in the business of running a nation-state, one that is counteracted but never discourages.

% next I will discuss the theoretical justifications for why this might be the case. security seeking states and types of greedy states derive a reational benefit from allowing espionage to continue.

% finally I will look at historical moments in which new technologies were deployed for espionage-like purporses, and demonstrate how the norms governing the use of these technologies always evolved to permit espionage.

\end{document}
