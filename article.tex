\documentclass[14pt]{extarticle}
\usepackage{style}


\title{Draft Article}
\author{Alexander W. Petros}
\date{September 2019}

\addbibresource{article.bib}

\begin{document}
\maketitle

In 2015, federal authorities discovered that malicious actors had for months maintained access to databases within the United States Office of Personnel Management (OPM). Among other types of sensitive data, these hackers stole a trove of applications for federal security clearances, securing not just the personal information of the applicants themselves but also the detailed information they had supplied about their family and friends, including social security numbers, job applications, and home addresses. The breach ultimately compromised the information of 22.1 million people, including personnel at the highest levels of government.\footcite{nakashima_hacks_2015} US officials feared that the stolen data could jeopardize intelligence efforts by exposing undercover CIA officers, or identify the American officials who might be susceptible to blackmail, or worse, recruitment. \enquote{if you have my [application for security clearance],} FBI Director James Comey admitted, ``you know every place I've lived since I was 18, contact people at those addresses, neighbors at those addresses, all of my family, every place I've traveled outside the United States. Just imagine if you were a foreign intelligence service and you had that data.''\footcite{nakashima_hacks_2015}. In the case of the OPM hack, imagination wasn't necessary; the US had already established that the Chinese military was likely behind the sophisticated attack.\footcite{koerner_inside_2016}. But even though it possessed conclusive digital forensics, the Obama administration refused to name China as the culprit.\footnote{National Security Advisor John Bolton was the first American official to formally acknowledge that the Chinese were behind the OPM hack, three years later, in September 2018. See \cite{sanger_trump_2018}}

% When OPM announced the full extent of the damage, it caused an uproar. The American Federation of Government Employees, the largest federal workers union, filed a class-action lawsuit against OPM, seeking damages under the Privacy Act.\footcite{chalfant_court_2017} Senator Mark Warner (D-Va.) of the Senate Intelligence Committee called for OPM Director Katherine Archuleta's resignation. Because the information might be used for identity theft, the US government offered all affected employees free credit and identity monitoring services for three years.

The US government's tepid response to the OPM hack illustrates a crucial fact about the relationship between cybersecurity and espionage: when a cyberattack looks like a traditional intelligence operation, it merits little to no diplomatic consequences. To the extent that phenomenon has been studied politically or academically, it is often explained away as a logical extension of existing policy: cyber-espionage is treated like \enquote{traditional} espionage because they are both forms of espionage. This article seeks to unravel that tautology by explaining how states determine what constitutes \enquote{traditional} espionage, what purpose that distinction serves, and whether it is prudent to continue making it. The transfer of espionage into the digital realm has made that task increasingly urgent, as the internet offers new vectors of attack that magnify the potential of espionage to do harm. ``This is one of those cases where you have to ask, `Does the size of the operation change the nature of it?'\thinspace'' said one anonymous senior intelligence official of the OPM hack. \enquote{Clearly, it does.}\footcite{sanger_u.s._2015} As far as the public knows, however, the size of operation did \emph{not} change the nature the diplomatic response. Espionage, regardless of the damage it does, is always handled more lightly than other forms of international aggression.

% Therefore, to understand why the Director of the CIA considers cyber-espionage to be \enquote{fair game,} one first needs to understand why traditional espionage has historically received the same treatment.



This article posits that the universal tolerance of espionage in the international system can be explained through a rational calculation of each state's incentives. On balance, all states derive benefits from preserving a system in which the consequences for espionage are minimized, so they refrain from imposing particularly harsh punishments for its practice. Even when an individual incident is especially damaging to a state's security, if it falls within the traditionally accepted boundaries of an espionage operation, it will likely go unpunished. In the case of the OPM hack, for example, US policymakers decided they could not impose harsh penalties on the Chinese because it was more valuable for the US to retain its ability to practice cyber-espionage with relative impunity in the future. This basic calculation---that espionage against one's own country is a tolerable price to pay for being allowed to practice espionage against one's rivals---has remained consistent since the advent of the modern peacetime intelligence service, perhaps even the nation-state itself.

To evaluate each state's incentives I first establish a theoretical framework for analyzing the role that espionage plays in international relations (IR) and the diplomatic norms that are associated with it. Through this framework, I define the bounds of traditional espionage and why rational states might calculate that it is in their best interests to continue tolerating it. I then examine two cases in which new technical means of gathering intelligence were introduced during the Cold War: aerial photography (spy planes) and satellite photography (spy satellites). In both cases, the norms of traditional espionage ultimately prevailed, minimizing the diplomatic consequences for their use as long as it was in service of traditional intelligence goals. Through process tracing, I demonstrate how the justifications policymakers employed for the use of these intelligence technologies---and the resulting minimization of the diplomatic penalties associated with them---are consistent with the theoretical underpinnings for how a tacit accord permitting general espionage might be mutually strategic.

% Add information about my conclusions here

\subsection{Espionage norms}
There are few extant definitions of peacetime espionage, and no satisfying ones. The problem that aspiring scholars of espionage inevitably encounter is that \enquote{peacetime espionage} is universally understood as a term of art but almost entirely missing from any legal codes or case law. Most scholars who tackle this question settle on a definition like the one proposed by I\~{n}aki Navarrete and Russell Buchan: \enquote{\enquote{Peacetime espionage} is a colloquial rather than a legal term [used] to describe different methods of collecting confidential information from closed as opposed to open sources \textelp{} generally conducted by States, against States.}\footcite[p.~901-902]{navarrete_out_2019} This definition purposefully excludes forms of extralegal government intervention that are often conflated with espionage, such as assassinations, election interference, and covert operations.\footnote{Often papers won't attempt to \enquote{define} espionage as such, but rather offer a working definition that bounds the scope of the paper. See \cite[p.~600]{radsan_unresolved_2007},\cite[p.~2]{kapp_spying_2007})} Excluding this activities is important; because assassinations and similar covert operations are not considered espionage they do not typically receive the protections of espionage norms.

For the purposes of this paper, I extend that definition to describe espionage \emph{norms} as the common understanding of how states are expected to respond to respond to activities that remain within the bounds of espionage.\footnote{There are, of course, other \enquote{norms} that involve espionage. The so-called \enquote{Moscow Rules} dictate an intelligence officer are supposed to conduct themselves in hostile territory. In order to avoid constantly specifying, espionage norms are more strictly defined to include only the ones that affect IR.} Espionage norms prescribe that when one state uncovers an espionage operation conducted against them, the consequences imposed on the spier remain minimal. Rarely does the response amount to more than a pat denunciation of the activity and implementation of the necessary countermeasures to end it---the consequences for espionage are borne by the agents who commit it, not the state that sent them. In rare cases when additional costs \emph{are} imposed on a state for conducting a traditional espionage operation, they are not costs that would discourage the aggressor from attempting such an operation again.

Once again, we can use cyber-espionage as an analogue for traditional espionage writ large. Though not codified in US policy, the informal regime of espionage norms is an open secret. A senior White House official told the Congressional Research Service that \enquote{there is a vast distinction between intelligence-gathering activities that all countries do and [other forms of malicious cyber-activity].}\footnote{In the context of the quote, the White House official is distinguishing between traditional espionage and economic espionage, the latter of which is the state-sponsored theft of trade secrets from foreign private companies in order to gain a competitive advantage in international markets. See \cite{finklea_cyber_2015}} For activities that fall under the umbrella of intelligence-gathering, counterintelligence rather than prosecution is considered the appropriate response. CIA Director John Brennan described foreign spying on domestic political institutions as \enquote{fair game.}\footcite{sanger_u.s._2016} And while speaking at an intelligence conference, James Clapper, Director of National Intelligence, was particularly frank: \enquote{You have to kind of salute the Chinese for what they did. If we had the opportunity to do that, I don’t think we’d hesitate for a minute.}\footcite{pepitone_clapper_2015}

There is solid data to support the presumption that cyber-espionage is routinely treated as a lesser category of offense by all nations, not just the United States. Of the hundreds of state-sponsored cyberattacks known to the public, attacks classified as espionage are by far the most common. By percentages, however, cyber-espionage is by far the least likely form of cyberattack to garner an official response from the victim (Table \ref{response-pct}).\footnote{These numbers were compiled from the Council on Foreign Relations' Cyber Operations Tracker, which is a publicly-sourced record of state-sponsored cyberattacks, updated quarterly. Of the 323 listed incidents, 38 were excluded from this calculation because they lacked an identifiable Victim, Sponsor, or Type. For more, see \cite{council_on_foreign_relations_new_2019}}

% latex table generated in R 3.4.3 by xtable 1.8-2 package
% Sat Aug 24 13:46:46 2019
\begin{table}[ht]
\centering
\begin{tabular}{lrrr}
  \hline
Type & Incidents & Official responses & Response \% \\
  \hline
Defacement &   3 &   2 & 66.67 \\
  Data destruction &   8 &   5 & 62.50 \\
  Doxing &   6 &   3 & 50.00 \\
  Sabotage &  14 &   5 & 35.71 \\
  Denial of service &  17 &   6 & 35.29 \\
  Espionage & 231 &  44 & 19.05 \\
   \hline
\end{tabular}
\caption{Global cyber-incident response rates}
\label{response-pct}
\end{table}

% The benefits of espionage have never been underrated---\enquote{it is only the enlightened ruler and the wise general,} reads Sun Tzu, \enquote{who will use the highest intelligence of the army for purposes of spying and thereby they achieve great results}---but the inevitability of espionage has long been taken for granted. If one questions the assertion that espionage is something \enquote{all countries do,} then the way that countries treat espionage becomes considerably more confusing.

\subsection{Existing literature}
Prior literature has attempted to explain espionage norms using international law, but in the absence of any significant treaties governing peacetime espionage, scholars are often left without definitive answers.\footnote{Because it is universally agreed that international law does not deal with peacetime espionage directly, many journal articles that deal with this question frame their work almost Socratically, devoting two sections to cases for and against the legality of espionage. In fact, the first major work on law and peacetime espionage is a series of commentaries collected in a single publication that respond to each other's arguments. See \cite{wright_essays_1962}} Espionage is universally illegal but every state acknowledges that allies and adversaries alike are guaranteed to engage in it.\footnote{Further complicating matters, espionage is the rare crime that might not be a crime depending on who the victim is. If an American government employee walks out of The Pentagon and hands classified documents to the KGB, they are at risk prosecuted as a traitor; if an American case officer hands classified KGB documents over to the CIA, they will be lauded as a hero. This example, in various forms, is frequently used to illustrate the impossibility of creating a consistent legal doctrine for espionage.} Because of that tension, scholars have typically treated espionage as \emph{sui generis}---starting from the assumption that espionage norms exist in a class of their own, then attempting to fit a legal doctrine around that fact. Jared Beim determines that while peacetime espionage is often tolerated, it clearly constitutes an \enquote{intervention} as defined by the International Court of Justice, so the prohibition on espionage merits further enforcement.\footnote{Beim also suggests that a weaker country could seek justice for an espionage operation by appealing to international organizations like the United Nations or the Council of Europe. For more, see \cite{beim_enforcing_2018}} Beim acknowledges, however, that his conclusion will have little impact on the practice of espionage, where, at best, \enquote{it is perhaps possible to stop the more egregious violations.}\footcite[p.~672]{beim_enforcing_2018} A. John Radsan employs a more philosophical approach. Suggesting that we should resist the \enquote{Hegelian impulse} to resolve the tension between espionage and international law, Radsan concludes that \enquote{beyond any international consensus, countries will continue to perform espionage to serve their national interests \textelp{} International law does not change the reality of espionage.} Legal scholars of espionage generally reach the same conclusion: it would be almost impossible to argue that espionage is legal, but no one really seems to care. Taken to its logical endpoint, this branch of scholarship would simply argue that the espionage norms we observe today result from simple inertia.

Another branch of scholarship sidesteps the legal question in favor of a \enquote{functional theory} of IR. This approach is more informative because it attempts to explain why states tolerate espionage rather than creating \emph{post hoc} justifications for it. A functional approach, for instance, might highlight the role of espionage in treaty verification, the way Christopher Baker does when he argues \enquote{the advantages that espionage offers over legally-binding verification and assurance regimes tip the scales in favor of functional cooperation.}\footcite{baker_tolerance_2004} Baker correctly notes that the additional assurance provided by espionage enables states to enter into riskier agreements that they would otherwise have avoided. His analysis is limited, however, by its focus on enabling specific policy outcomes, like treaties. It doesn't resolve the gap between the absolute illegality of espionage in domestic law and its widespread practice internationally. Responding to Baker, Radsan notes that one-off deals and treaties \enquote{are a very slow and indirect way to achieve international consensus on the legality of espionage. The gap is still there.}\footcite[p.~607]{radsan_unresolved_2007}

% This article expands on Baker's basic premise---that there are situations in which espionage is mutually advantageous for both parties---and universalizes it to explain why the entire institution of espionage is mutually advantageous.

The key insight this article presents---and where it breaks from previous literature---is that far from being an irreplicable, historically contingent set of practices, espionage norms are actually preserved through an ongoing set of strategic decisions. Each time a new espionage technology is made available, policymakers are given a fresh opportunity to decide whether or not the norms of espionage ought to extend to its use. Critically, the research that follows will demonstrate that they were not obligated to do so by any pre-existing norms. When the US first introduced photographic satellites, for example, it was not clear whether the Soviet Union would treat the satellites as spies or as weapons---there were indications in both directions. Ultimately, the infrastructure of what we now call \enquote{spy satellites} was developed, and the norms of espionage prevailed to protect their use.

% Because I argue that espionage norms influence cyber policy, it is worth mentioning scholarship on the role that espionage, outside of cyberspace, plays in the international system today. The intersection of espionage and international relations has been studied before---beginning, in the modern context, as far back as 1962---but usually in a historical context, not a political science one. The role of espionage in deescalating conflict is typically relegated to accounts of moments where Great People properly interpreted---or fatally misunderstood---the information that their intelligence operations provided them. Chapters 3 and 4 will primarily focus on decisions made by Eisenhower and Khrushchev, so to some extent I am guilty of that as well. Nonetheless, I will try to highlight the way in which both of those leaders set precedents designed to outlast them, across a variety of intelligence methods. I hope that this thesis starts the conversation about how espionage norms operate to deescalate conflict independent of the leaders that promote them.

% By employing both realist and constructivist schools of IR, this article develops rigorous explanations for why rational states, whether they be security-seeking and greedy, might cooperate to preserve this dynamic.

% incorporating cyber espionage into the intelligence literature, rather than the cyebr literature
% espionage first, cyber second

\section{Theory}
\subsection{Establishing the puzzle}
One of the most prevalent clich\'es in spy literature and journalism is the so-called \enquote{rules of the game.} The supposed rules of the spy world---\enquote{You are never alone}, \enquote{Don't trust anyone}---make the intelligence community seem insular, mysterious, and impenetrable.\footcite{myre_moscow_2019} They hint at a larger set of processes whose motivations the average citizen, not given access to this privileged information, could not possibly begin to understand. In reality, while the \emph{operation} of a modern intelligence service is incredibly opaque, the geopolitical situation in which one operates is not. Every aspect of a nation-state's interactions with another ought to be subject to the same critical analysis. Scholars can use rationalist principles to analyze everything from trade policy to the space race; espionage, which by definition involves deeply invasive interactions with another state, ought to be no different.


% Of course, for espionage to be mutually permissible, both states must agree to treat it that way, but, in a world with perfect information, that requires \emph{both} states to determine they are getting more from practicing espionage against their adversary than their adversary is getting from practicing it against them. Otherwise, the state with less to gain would simply defect from that equilibrium and punish espionage at a level that imposes direct costs.

% Since this generally doesn't happen between major world powers, one of two things must be true: either a state is miscalculating the relative gains it achieves through espionage, or the mutual tolerance of espionage has benefits that are sufficiently important to maintain even if one state gains less from it than its adversary.


% Certainly there are factors that the go into diplomatic decisions that the general public will likely never know. There are few reasons to believe, however, that this is more true for intelligence-related diplomatic matters than for generic ones. Every diplomatic decision is made with the backing of classified intelligence.

Using traditional methods of analysis, espionage certainly seems like something that states should be doing more to discourage. The results of intelligence operations can have devastating, even existential effects on the targeted nation's security. Arguably the most traditional form of espionage is stealing military secrets from a foreign power, and in that arena China has been a relentless combatant. A United States Navy report from March 2019 declared Chinese intelligence operations so extensive that they had substantively altered the balance of power between the two states.\footcite{lubold_navy_2019} \enquote{Long-term, US future military advantage is being diminished by years of IP exfiltration from the DoD, DoN, and DIB,} the report read, \enquote{all with little to no adverse consequences to the thieves.}\footcite[p.~6]{bayer_cybersecurity_2019} The forecasts were dire: \enquote{If the current trend continues unimpeded, the US will soon lose its status as the dominant global economic power.}\footcite[p.~5]{bayer_cybersecurity_2019} The US Navy is not a disinterested party, of course, and there are reasons to take its conclusions with a grain of salt; modern military technology is remarkably complex and China cannot simply steal its way to military dominance.\footnote{For a good discussion of these limitations, see \cite{gilli_why_2019}} Nonetheless, espionage presents alarming opportunities for rival states and the consequences for taking advantage of them remain minimal.

Espionage has potentially disastrous consequences but the steps that might actually curb it are out of the question. The system of repercussions is set up so that the structure of civilian intelligence services is never fundamentally threatened. When an especially damaging espionage operation is uncovered, the traditional punishment is to declare certain embassy officials \emph{persona non grata}, which often results in \enquote{tit-for-tat} diplomatic expulsions. These expulsions make it more difficult for the affected state to conduct intelligence operations in the short-term, but they don't fundamentally punish the state for having attempted espionage in the first place.

The name \enquote{tit-for-tat} is taken from a highly-successful strategy for the prisoner's dilemma wherein the player simply mimics the choice that that their opponent made in the last round. By employing the tit-for-tat strategy, the player never gets taken advantage of for more than one turn and can always return to cooperation if their opponent unilaterally signals a willingness to do the same. Diplomatic expulsions play out in exactly the same fashion: when one country expels the diplomats of another, the other country will often respond in kind. Each side will continue to expel each other's diplomats until one side signals that it would like for the reciprocal removal of diplomats to stop. Inevitably, both sides return to a cooperative state while they each still have \emph{some} diplomats left. Because neither side can gain an a decisive advantage, the back-and-forth punishment never ends in a mutual defection.

\subsection{Espionage norms as a discrete series of choices}
The key feature of \enquote{tit-for-tat} diplomatic expulsions is that the scope of the punishment is structurally incapable of discouraging a future, identical operation. Losing intelligence officers in an embassy temporarily makes it more difficult to conduct espionage, but it doesn't make a state less likely to try; the worst that can happen is a net-neutral result---going back to \enquote{square one,} so to speak. By contrast, were it more common to respond to espionage by imposing financial sanctions or even inflicting a military cost, a state would have to consider the possibility that attempting espionage could adversely impact other aspects of its security, leaving it worse off than when it started. These are the sorts of calculations that states need to make in other high-stakes arenas. Pursuing a nuclear weapons program, for instance, could lead to severe economic retaliation, as could protectionist trade policies. By comparison, engaging in espionage has very little potential for negative externalities because espionage norms take the possibility of a severe response off the table.

% https://tex.stackexchange.com/questions/249480/creating-a-payoff-matrix-using-latex-tabular-environment
\begin{table}[ht]
\centering
\setlength{\extrarowheight}{2pt}
\small
\begin{tabular}{cc|c|c|}
  & \multicolumn{1}{c}{} & \multicolumn{2}{c}{Country 2}\\
  & \multicolumn{1}{c}{} & \multicolumn{1}{c}{Tolerate Espionage}  & \multicolumn{1}{c}{Punish Espionage} \\\cline{3-4}
  \multirow{3}*{Country 1}  & Tolerate Espionage & \makecell{~\\Both sides gain intelligence \\~} & \makecell{~\\ Only the USSR continues spying \\ ~} \\\cline{3-4}
  & Punish Espionage & \makecell{~\\ Only the US continues spying \\~} & \makecell{~\\ No intelligence either direction \\~} \\\cline{3-4}
\end{tabular}
\caption{Espionage consequences decision matrix}
\label{espionage-matrix}
\end{table}

% Possibly swap this with the previous paragraph

To describe how states cooperate to preserve the liminal legal status of espionage, I propose an iterated decision-making model. My model changes the parameters so that the decision to cooperate or defect is based not in the amount of punishment received, but rather the type. Each time a state is forced to respond to a successful intelligence operation by a rival---such as the reveal of a high-level spy or a major data breach---it can either limit its response to one which is traditionally associated with espionage, or it can impose unprecedented costs that might discourage such an operation in the future. Table \ref{espionage-matrix} describes a single instance of this choice. In the status quo, State 1 signals its interest in cooperating by limiting its retaliation in response to espionage, no matter how damaging the operation is to their national security. State 2 responds by doing the same. Each iteration is a chance to play the game with fresh information; each espionage case is an opportunity to cooperate, reaffirming the norm that states always respond to espionage in a specific way, or defect, adding an additional cost to the attempt.

Through this model two important things become clear. First, even though either state has the opportunity to defect by imposing harsher punishments in response to espionage, its gains would be short-lived because the other state is guaranteed to immediately do the same---tit-for-tat. Espionage consequences are an iterated game in which both parties play the exact same way, so, in the long run, the only two possible outcomes are mutual defection or mutual cooperation. In the realm of traditional espionage, we've only seen states engage in mutual cooperation, but any state has the power to change that by defecting. Even if the defector's rival would have preferred to maintain the status quo, the rival has no option other than to defect as well or else be left at a senseless disadvantage. The relatively open flow of information is only possible with a universal buy-in.

Second, while there is unquestionably a cultural norm informing the response to intelligence operations, that norm is doing a lot of work to maintain the status quo. So much work, in fact, that tradition could not possibly explain the entire phenomenon. Stripped of its mystique, the benefits of allowing intelligence operations to continue are, like any other state decision, a simple cost-benefit analysis; states decide that it best serves their interests to remain in the top left corner, rather than the bottom right. We can expect then that there is \emph{some} rational explanation that political leaders employ to justify it, where they weigh the benefits of preserving their ability to spy against the damage that results from espionage conducted against them.

One possible explanation is that some states are simply miscalculating. Theoretically, the basic premise is persuasive. Every policymaker knows exactly how crucial espionage is to their job. With years of experience making difficult policy calls informed by intelligence, it is certainly much easier to imagine how one's decisions will be made worse without espionage than it is to imagine how one's enemies' decisions will be made better. It follows that there would be a strong psychological bias towards attempting to discover as much information as possible. While this is likely true to some extent, mass-miscalculation could not fully explain the widespread persistence of espionage.

The surprisingly straightforward conclusion is that, if only two worlds are possible---one with widespread espionage and one without---then all the relevant nations must be concluding that the world without espionage is incredibly bad---so bad that even if one nation correctly concludes that its rival gains more from a world with a free flow of information than it does, the alternative, a world with significantly harsher punishments for espionage and a staunched flow of intelligence, still leaves that nation worse off. That calculation has a name: defensive realism.

\subsection{Defensive realism}
To explain why states might be willing to accept relative losses in security, I employ Charles Glaser's \emph{Rational Theory of International Politics}. Glaser describes his theory as integrating defensive realism with neoclassical realism.\footnote{With apologies to Mr. Glaser, I am going to refer to it as defensive realism, because they both put the security dilemma at the center of their explanation for international competition, which is the key observation for me. He would have you know that defensive realism \enquote{can be viewed as a way station along the route to the full theory} that he develops, which is \enquote{significantly more general and complete than is defensive realism.} (p. 13-14)} Glaser begins from the core assumption that the reason for competition between security-seeking states is insecurity. Consider a hypothetical security seeker interacting with its adversary, which it believes to be a security-seeking state as well. The first state feels insecure, so it builds up arms to increase its security. The rival state, seeing that its adversary is building up arms, now feels less secure and builds up its own arsenal in response, and a classic security dilemma ensues. Even though both states were just security seekers, they're now both much less secure.

% If both states are genuine security-seekers, then this situation is clearly sub-optimal, because now both states are heavily militarized and comparatively much less secure.

Defensive realism introduces a counterintuitive observation: the key to interrupting the security dilemma is to make the adversary feel more secure, because it is the adversary's response to securitization that ultimately makes the first state insecure. Therefore, defensive realism posits that it is logical for a security seeker to take actions that increase not just its own security, but its \emph{adversary's} security as well.\footcite[p.~7]{glaser_rational_2010} These actions would signal to the adversary that the state is a security seeker, hoping to reach a mutually agreeable outcome. If the adversary is indeed a security seeker, then, no longer concerned about its security, it will respond in kind and resolve the security dilemma for both of them.

The problem for a defensive realist is that unilaterally taking steps to increase the adversary's security is obviously quite risky. The state could be wrong that its adversary is a security seeker, in which case its optimistic overtures of cooperation will have exposed it to greater risk of attack. When states are unwilling to make those signaling measures because they fear compromising their own security, then a security dilemma results. This is where intelligence comes into play.

Prior to making any decisions, states assess the security situation and determine which options are most likely to increase their own security. Glaser identifies two types of variables that mediate the magnitude of a security dilemma. The first are material variables, such as offense-defense balance and offense-defense distinguishability. Espionage plays a significant role in properly assessing material variables; during the Cold War, for example, the United States placed a premium on generating quality estimates of how many bombers, missiles, and bases the Soviet Union had deployed. By knowing which of its adversary's capabilities are offensive, a state can better target its efforts towards reducing them.

In addition to material variables, Glaser notes that information variables play an important and comparatively under-studied role in the security dilemma. Better information about its adversary's motives can give the state reason to trust that risky signaling measures might succeed---or fail. Even if the state is uncertain, espionage can provide some information about its adversary's intent, which, in turn, can moderate the security dilemma enough to make cooperation possible. In the best case-scenario, the security dilemma is simply eliminated if both states are confident that the other is a security seeker. This is the optimal outcome for two genuine security-seekers, and it is made much more likely from the intelligence that espionage can provide.

Consider a situation in which one state is a genuine security-seeker and so is its adversary. The state conducts espionage on its adversary, and the resulting intelligence increases its internal estimate of how likely the adversary is to be a security-seeker. As a result, the state decides that some form of arms control is no longer too risky to attempt in order to signal its benign intentions. Because the state receiving the signal is also a security-seeker, this is a welcome development; it could be empowered to signal its benign intentions as well, based on its own intelligence. For both states, this is the optimal outcome. The counterintuitive observation here is that both states benefited not just from spying, but from being spied \emph{upon}; their adversary's intelligence made them more likely to make a risky signal for deescalation.

% States now recognize that some forms of espionage facilitate international cooperation.\footcite{baker_tolerance_2004}

A complete theoretical justification for espionage---that maintaining a mutual exchange of illicit information is better for both states than the alternative---can be constructed by combining the work of previous scholars, and finding archival evidence to support their claims. Christopher Baker explains how espionage makes functional cooperation possible through treaties, allowing states to verify each others' compliance.\footcite[p.~6]{baker_tolerance_2004} Charles Glaser uses the 1972 Anti-Ballistic missile treaty as an example of precisely the type of signal that could lead states to resolve the security dilemma, but he does not acknowledge that the intelligence operations of both states are what make signals like it possible.\footnote{It is entirely possible that Glaser himself realizes this, it's just not written down; the purpose of the book is to defend the theory, not explain the mechanics.} And Julius Stone, a professor of international law, foresaw this exact role for espionage in 1962, well before the public had any real idea about the US spy satellite program: \enquote{If you do not have a system of international inspection and if you can't get one (and I think it is quite likely that we can't) then the function which international inspection is supposed to serve still needs fulfilling. You may not be able to fulfill it to the optimum extent by reciprocally tolerated espionage, but you may be able to reduce the dangers.}\footcite[p.~41]{stone_legal_1962}

In 1962, Stone defended espionage from a practical perspective in exactly the same terms that I am now defending it from a theoretical one. Permitting at least some kinds of espionage is actually beneficial for both sides because it allows us to independently verify the intent and capabilities of other nations. Doing so actually reduces the likelihood that miscalculation will lead to open hostilities. As a result, both states allow it to happen, because they each benefit from a less-tense security environment. Stone aptly described this phenomenon as \enquote{reciprocally tolerated espionage.}

\subsection{Introducing the empirical examples}

The best way to prove that this theory has explanatory power for state decision-making is to find historical examples where states made decisions that showed they were willing to accept the cost of enemy espionage in order to practice espionage themselves. In the ideal scenario, we would find evidence of policymakers justifying their decisions based on the value of mutual observation. Such decisions were made frequently during the Cold War, and with the explicit intent to increase the bilateral flow of information to create a more secure world environment. This article will present two such cases: aerial espionage and spy satellites.

Both planes and satellites played an under-explored role in the formulation of some of the Cold War's landmark diplomatic achievements. SALT I (the ABM Treaty) is widely considered to be the landmark diplomatic achievement of Nixon's \emph{d\'etente} policy, which set the stage for a broader easing of relations between the US and the USSR, and spy satellites were crucial to reducing the cost that each side would pay to sign it. Limiting ballistic missile defense systems was a relatively minor concession for both sides, but in the highly risk-averse context of the Cold War, the signal was significant.\footcite[p.~66]{glaser_rational_2010} Underneath the omnipresent threat of nuclear annihilation, \emph{d\'etente} provided a rational strategic choice for both states---made possible by their reciprocal tolerance of espionage.

% I have yet to find a quote from Nikita Khrushchev in which he states that it was good for the Soviet Union to be have its state secrets compromised. I don't expect that I ever will. The realist approach to espionage is difficult to acknowledge politically; it cuts against national pride and power in order to reach a pragmatic, somewhat emotionally unsatisfying conclusion. I asked a senior national security policymaker, Jake Sullivan, whether he thought American espionage was a net positive for global security. ``I wouldn't work for the American government if I did not believe that it was a net positive actor,'' he told me, \enquote{so I want the United States to be able to make the most informed decisions possible.} When I asked whether it promoted global stability for a foreign nation---Russia for instance---to spy on the United States, he qualified his answer.

% He split the types of intelligence that foreign state might gather on the United States into three types: intent, capabilities, and strategy. \enquote{I have no problem with the Russians knowing that we do not have aggressive intent,} he told me. Capabilities, however, are more complicated. Some defense capabilities would probably be fine to have another country independently verify, and some we might want to keep secret, such as the atomic bomb before the end of the Second World War. And under almost no circumstances would a state want to have its strategies and warfighting plans revealed, so those should be kept as secret as possible.\footcite{sullivan_personal_2019} Without invoking the theory itself, Sullivan described exactly the type of intelligence that a defensive realist would need to solve the security dilemma.\footnote{At the point in my research when I conducted this interview, I had never even heard the term \enquote{defensive realism,} lest you think I led him to it.}

% The problem is that a state will never be able to control what type of information is being leaked. Espionage is an all-or-nothing proposition; either a state is willing to risk compromising some government secrets---including the ones that it is always disadvantageous to leak---in order to preserve the right to attempt it itself, or the state defects and begins imposing severe diplomatic consequences for espionage, inevitably inviting the same consequences upon itself, at which point the information pipeline tightens drastically. Unfailingly, states will choose to continue cooperating, so they must all be making the same calculus that the information they lose is worth the security they gain from getting to practice espionage themselves.

% If you believe that American intent is benign, then tacitly permitting espionage is one way out of the security dilemma. Sullivan and Glaser did not conclusively explain, however, why \emph{every} major power continues to cooperate. More likely than not, some states today are not pure security-seekers. Glaser calls these greedy states, and when a state's motives are mixed between greed and security, those get assigned the label of greedy as well. For a greedy state, having another state correctly evaluate its intent could prove disastrous. The state might be pretending to have a benign intent in order to gain an advantage and espionage might reveal the truth.

% That is only partially true. I certainly believe that elements of espionage and its place in diplomacy are socially constructed. The practice of expelling diplomats, for instance, predates all the events discussed in this thesis, and almost certainly evolved from some sort of customary protocol, rather than a rational calculation about the most effective way of responding to espionage.

% An institutional theory of politics cannot explain the endurance of espionage norms, because there are no international institutions dedicated to governing espionage. As the legal section notes, peacetime espionage in most traditional sense---activities associated with human spies---has virtually no legal recognition at all. The first institutional recognition of espionage for conflict-reduction purposes is the SALT I treaty in 1972, which protected \enquote{national means of verification,} but espionage-as-verification is limited to the specific treaties in which it is in included. Espionage is probably just too uncouth to formally acknowledge.

% Likewise, the only legal regime that applied to a spy plane was the one that applied to planes in general, and that explicitly prohibited the U-2 overflights; the Treaty on Open Skies was signed in 1992, well after the events discussed here. The clear illegality of the U-2 overflights should have given the Soviet Union more leverage to gain concessions, not less. As for space, the Outer Space Treaty entered into effect in 1967, eight years after \emph{Discoverer 14}; the strange timing of that treaty, right as the USSR was developing ASAT capabilities, was noted in Chapter 4.


% I posit that the desire from all states to continue the practice of espionage writ large, even as they seek to stop particular instances of it against themselves, has an important psychological component. Every policymaker would like to their decisions to be as well-informed as possible. Without the benefit of hindsight, it is very difficult objectively gauge how much \enquote{security} is lost from a successful intelligence operation. By contrast, it is very easy for someone in charge of making an important decision to notice how much better off they are for having had access to some illicitly-obtained information. I don't mean to suggest that these trade-offs are always impossible to evaluate. Khrushchev was keenly aware of how the satellite photography would affect his missile bluff; preserving the missile bluff was just not worth impacting Khrushchevs's own ability to make use of satellite photography. In situations where it is difficult to evaluate relative gains, there exists a strong psychological bias towards preserving ones own view of the situation, even if it means allowing the adversary to have its view as well.

% If espionage often leads to mutually agreeable outcomes, then why is not just completely legal? Well, in some cases it is. A true Open Skies Treaty followed soon after the collapse of the Soviet Union. In just over a decade, spy satellites went from \enquote{espionage in space} to \enquote{national technical means of verification.} But even as espionage gained some legitimacy in statecraft, the vagueness of that wording, \enquote{means of verification,} is indicative of what keeps espionage from true public acknowledgment. The intelligence business is often looked at---not entirely unfairly---as being shady. Reframing espionage as treaty verification procedure can help mitigate that, but if formally acknowledging spy satellites feels uncouth, its hard to imagine that human spies will ever receive a legal regime of their own.


% Regardless of theoretical explanation, the cooperative outcomes of the Cold War is remarkable. The USSR had reason to believe that it would have benefited from a mutual prohibition on the technology that the United States was using. The next two chapters will go in-depth on the process that led to this decision---from both sides---to make the case that espionage norms have a fundamental appeal that the United States was able to exploit for its own benefit, and induce the Soviet Union to cooperate.

% I use defensive realism to formulate a theoretical foundation to explain the existence of espionage norms generally, and the Soviet Union's extension of espionage norms to spy satellites in particular. Between two security-seeking states, both states will be less secure in the event of an arms race, so preventing one is a mutually-agreeable outcome. Resolving the security dilemma requires that the states signal their benign intentions, which states are more likely to do if they have good intelligence---intelligence that informs them about their adversary's capabilities and intent. Therefore, cooperating to preserve the practice of espionage as a semi-legitimate state institution makes rational sense, both because of the benefits espionage offers the state doing the spying, and the benefits the same state derives \emph{from being spied upon}.

\section{Aerial espionage}
The line between a \enquote{cold} and a \enquote{hot} war is supposed to be obvious. A superpower's own uniformed military personnel opening fire on the other superpower's military or territory would be \enquote{hot,} everything else would be \enquote{cold.} Broadly speaking, the Cold War is remembered as lacking direct military confrontation between the United States and Russia; though the two superpowers of the Cold War were open rivals, they supposedly competed for influence by funding, arming, or otherwise influencing proxies, rather than attacking each other directly. That impression is not exactly accurate. Not only did the US and USSR often come close to openly attacking each other, but a surprising number of times they actually did—--including one of the Cold War's most famous moments.

% citation needed on the definition of cold

Pilot Francis Gary Powers knew that U-2 overflights of the Soviet Union were incredibly risky. The Soviets were deeply unhappy that the US was violating their airspace in such a brazen fashion, but for years they had kept quiet about the U-2 missions, embarrassed that they were technologically incapable of intercepting the high-flying aircraft.\footcite[p.~59]{powers_operation_2004} When Soviet air defenses finally managed to shoot one down in 1960, Soviet leadership was equal parts triumphant and furious. At a planned summit in Paris only a few weeks later, Soviet Premier Khrushchev launched into a rant about the U-2 planes and stormed out on the first day, effectively ending the summit---and any hope that pressing issues like the fate of West Berlin would be resolved there. He also canceled a planned visit by US President Dwight Eisenhower to the USSR that was scheduled to take place the next month. A contemporary newsreel about the conference said that \enquote{in the course of two hours, Khrushchev brought US-Soviet relations to their lowest point since the end of World War II.}\footcite{universal_studios_summit_1960}

The shootdown of Gary Powers, in what would later come to be known as \enquote{The U-2 Incident}, clearly violated the only \enquote{rule} of the Cold War. Official Soviet air defenses shot down a US aircraft. All parties involved had cause to escalate militarily---the Soviets for the violation of their airspace, the US for the attack on their \enquote{weather craft}---but instead they chose to treat the U-2 missions as an intelligence operation, not a military action. The Soviets therefore applied the traditional consequences associated with espionage; Khrushchev engaged in some saber-rattling and denounced the violation of his airspace, but no material actions were taken, and the United States escaped the incident with few tangible consequences. The striking disparity between the diplomatic furor and the lack of material consequences can be explained by an apparent technicality: the U-2 was not a military aircraft, it was a spy plane.

No single attribute defines a spy plane. A military aircraft can forgo weapons but, by virtue of its pilot, its mission, its timing, or its branding still conduct what is unquestionably a military operation. Inversely, intelligence missions can be flown by uniformed Air Force pilots in re-purposed B-52s and somehow still be understood as conducting espionage. A spy plane is an ontological designation, one that reveals itself to the adversary through a series of social and physical signifiers that are not replicable across time. Germany in 1937 might not have reacted to a U-2 overflight the same way that the Soviet Union did in 1960. But even if the way different states handled espionage was perfectly consistent, there still wouldn't be a checklist to guarantee that a mission would be treated as espionage; it is not enough make a military mission look like an intelligence operation. The reality is both simpler and more frustrating---it has \emph{to be one}.

The U-2 incident was the most visible in a series of aerial intelligence operations that the United States undertook early in the Cold War. The US did not always employ civilian pilots nor it did always use obviously weaponless aircraft, yet it managed to run a shocking number of missions that violated Soviet airspace without significantly altering the relative level of tension between the two superpowers. On a surprisingly frequent basis, these missions would turn \enquote{hot} when Soviet air defenses shot them down. Not once, however, did these shootdowns lead to retaliation from the US, or significant pressure from the Soviets to stop the flights. Instead, each one was resolved as if the pilots who survived were simply spies, captured in the normal course of counterintelligence. This article seeks to answers how and why states ensure that certain operations reveal themselves as espionage; the decade of covert aerial espionage was one such successful operation.

\subsection{Military reconnaissance flights}
The tense battle between the US and USSR over surveillance of Soviet border installations was never formally acknowledged by either party---and neither were its combatants. Over the course of the Cold War, more than 200 American pilots were shot down while spying on the Soviet Union.\footcite{glenshaw_secret_2017} These missions were so highly classified that in many cases the families of the pilots never learned how they died. 126 of these American airmen remain unaccounted for today, casualties of missions whose existence both the US government and the Soviet Union collaborated for decades to deny.

Though they flew Air Force planes, these men were actually conducting one of the earliest forms of Cold War peacetime espionage. Their purpose was to gather SIGINT (signals intelligence) identifying the location of critical radar installations along the Soviet border. Converted bombers---called \enquote{ferrets} by the Air Force---were outfitted with advanced radio equipment and sent to intercept as much information about Soviet radio transmissions as possible.\footcite[p.~4]{peterson_maybe_1993} According to the CIA, these missions began as early as 1947.\footcite[p.~4]{peterson_maybe_1993} Some of these flights only flew near the border, remaining over traditionally recognized international waters, while others were \enquote{overflights,} missions that deliberately violated Soviet airspace. The earliest definitive postwar overflight of the Soviet Union took place on May 10, 1949, when two RF-80As surveilled the Kurile islands\footcite[p.~8]{peebles_shadow_2000}

Without access to classified archives it is impossible to state for certain exactly how many of these flights were conducted. Even if one did have unlimited access, some of the necessary information was so highly classified that it is likely lost forever. Nonetheless, the civilian researcher is aided by the declassified research of military archivists before them and the works of two scholars in particular are highlighted here. First, Air Force Academy historian Dr. John Farquhar specializes in aerial reconnaissance and its diplomatic consequences. Many of the incidents analyzed in this section are taken from newspaper clippings that he compiled and analyzed. Second, former intelligence analyst and section chief Michael Peterson wrote an article for the CIA history magazine \emph{Cryptologic Quarterly} that analyzed the Soviet shootdowns of US reconnaissance aircraft in depth.

Peterson's article was written for a classified audience in 1993 and declassified in 2009, providing the public with a comprehensive (as far as we know) list of all American SIGINT flights shot down by the USSR (Table \ref{soviet-shootdowns}). Peterson uses two limiting criteria that perfectly suit the purposes of this article: the only flights he lists are those that were (a) shot down by Soviet attacks, not Chinese, North Korean, etc., and (b) explicitly US reconnaissance aircraft.\footcite[p.~4]{peterson_maybe_1993} Note that not all of these flights are necessarily overflights; many took place over international waters when they were shot down. In effect, the table catalogs the 13 times where the USSR shot down an American aircraft designed for collecting intelligence between 1950 and 1964---intercepted espionage attempts.\footcite[p.~5. In the original document, this table lists the first incident as having taken place over the Barents Sea, not the Baltic Sea. Because the description of the mission---including a map of its route in the same document---takes place entirely over the Baltic sea, I have concluded that this has to be a typographical error, and corrected it here.]{peterson_maybe_1993}

\begin{table}[ht]
\centering
\begin{tabular}{llr}
\textbf{Date}     & \textbf{\makecell[l]{U.S. Service \&\\ Aircraft Type}}   & \textbf{General Location} \\
8 April 1950      & USN PB4Y2 Privateer           & Baltic Sea                           \\
6 November 1951   & USN P2V Neptune               & Sea of Japan                         \\
13 June 1952      & USAF RB-29                    & Sea of Japan                         \\
7 October 1952    & USAF RB-29                    & East of Hokkaido/Kuril Is.           \\
29 July 1953      & USAF RB-50                    & Sea of Japan                         \\
4 September 1954  & USN P2V Neptune               & Sea of Japan                         \\
7 November 1954   & USAD RB-29                    & East of Hokkaido.Kuril Is.           \\
18 April 1955     & USAF RB-47                    & Off Kamchatka Peninsula              \\
10 September 1956 & USAF RB-50                    & Sea of Japan                         \\
2 September 1958  & USADF C-130                   & Soviet Armenia
\\
1 May 1960        & CIA U-2                       & Sverdolsk, USSR                      \\
1 July 1960       & USAF RB-47                    & Barents Sea                          \\
10 March 1964     & USAF RB-66                    & East Germany
\end{tabular}
\caption{Summary of Soviet Shootdowns, 1950-1964}
\label{soviet-shootdowns}
\end{table}

Regardless of how thorough declassified CIA information appears, one cannot rule out the possibility that there were other, even more secret missions which, for whatever reason, the United States government has chosen to conceal to this day. Nonetheless, a few aspects of the \emph{Cryptologic Quarterly} article are promising, and suggest that the information is relatively complete. First, the article was written for a classified audience as an overview of reconnaissance shootdowns during the 1950s, so it is likely intended to be comprehensive. Shootdowns are a sufficiently significant event such that it is unlikely the author would have just missed one. Additionally, when the article was declassified 16 years later, it appears to have been declassified without redactions or missing pages. Lastly, if there had been another mission so highly classified that it was omitted from this document, it is difficult to imagine how diplomatic consequences could have been imposed with a degree of secrecy that would prevent us from knowing about it today.

To comprehensively analyze the diplomatic effects of aerial espionage, I cross-referenced the shootdowns that the CIA says took place with Farquhar's analysis of their media and diplomatic impact. A shootdown is either present in both lists, or just one of the two---some incidents in the CIA list are missing from Farquhar's list because they did not have a diplomatic impact, and some incidents that Farquhar analyzes are not mentioned by Peterson because they were not SIGINT reconnaissance missions and likely (though not definitively) served little espionage function. Together, the full spectrum of aerial reconnaissance incidents demonstrates a clear pattern in which shootdowns that were interpreted as espionage were quickly resolved by both parties, while the shootdown of a flight unrelated to espionage produced swift, significant backlash.

The first of these shootdowns, on April 1950, brought the previously-hidden aerial reconnaissance program to the attention of the American people. An unarmed Navy Patrol plane, a PB4Y2 Privateer, was intercepted and downed by Soviet fighters over the Baltic sea. The Soviet ambassador to the US lodged a formal note of protest, incorrectly alleging that the US aircraft which had been shot down was a B-29 bomber.\footcite{kirk_ambassador_1950} A few days later, the American ambassador issued a formal response, in which he correctly identified the model of plane and claimed that at no point did it cross into any territorial waters. He demanded the Soviets conduct \enquote{a prompt and thorough investigation} and \enquote{see to it that those responsible for this action [were] promptly and severely punished.}\footcite{the_associated_press_text_1950} The Soviet Union recognized this for the toothless statement that it was, and rejected the demand in its entirety.\footcite{salisbury_kremlin_1950} No further action was taken.

Both sides had reasons to make this a larger political issue. The Soviet Union had fired first on a US Navy aircraft---which could not have been the aggressor because its only armament was a .45-cal pistol carried by one of the crewmen.\footcite[p.~7]{peterson_maybe_1993} Given that the plane was never found, all available evidence suggests that the flight likely did not violate the traditional 12-mile airspace boundary. Based on the last received position report, today the CIA claims that the plane was flying 20-25 miles off the coast of Latvia at the time it was shot down.\footcite[p.~7]{peterson_maybe_1993} At the time, it was theorized that a theoretical breakdown in the plane's navigation system could have caused it to veer off-course into Latvia, but given its last known position, it would have taken \enquote{a navigational error of nearly 90 degrees to cause the craft accidentally to wander over the Baltic states.}\footcite{the_new_york_times_soviet_1950} Meanwhile, the United States knew that the missing plane probably hadn't ended up in Soviet airspace, but the Privateer in question \emph{had} been performing a reconnaissance mission of questionable legality when it was shot down. No traditional boundaries were violated, but a repurposed Navy privateer loaded with electronic reconnaissance equipment flying close to Latvia looked, to use a less technical term, shady. And still, based on the initial exchange of diplomatic protests, neither side was inclined to magnify the issue if they could avoid it.

% The American press had no such reservations. \emph{Washington Post} columnist Walter Lippmann wrote that it could not have been \enquote{a local incident brought on by a local commander but [rather] that it was an act of Soviet policy. The known facts indicate that Soviet intelligence \textelp{} believed it was carrying important electronic equipment and that orders were given to the Soviet fighter command to intercept it.}\footcite{lippmann_baltic_1950} He speculated that there was no way the plane could have violated Soviet airspace, because wreckage would have been recovered, the plane was too slow, and no commander would have sent it on such a dangerous mission intentionally. \enquote{If, when, and as the American command were reconnoitering the Soviet military establishments on the Baltic coast,} Lippmann wrote, \enquote{it would use a plane of a wholly different type.}\footcite{lippmann_baltic_1950}

% A newspaper columnist is not a policymaker, but his argument was likely reflective of the political opinion that Americans held regarding espionage at the time: of course we do these things, we just do them \emph{better} than that---and if the US had been doing the thing that the Soviets claimed it was, then it would have been justified of them to open fire. \enquote{Baltic Plane Mystery,} a \emph{Post} article written a few days later, was almost sympathetic to the Soviet commanders. \enquote{Electronics make the old delimitations for border coastal flights ridiculous. A plane flying on a course perfectly legal by standards accepted today might still be engaged in reconnaissance of the first importance that an unfriendly power would try to frustrate.}\footcite{childs_baltic_1950} This is true, both in terms of the mission itself and the technological environment, but he went even further. \enquote{The only sound and safe assumption is that the Russians have a thorough and far-reaching espionage system. And at the same time we must hope that our system, and particularly on the side of counterespionage, is effective.} Even the press, though outraged at the loss of American life, apparently took it as a given that this kind of espionage is necessary---and expected from both sides.

The general attitudes displayed in response to the Baltic Incident became the normal course of response to Soviet flights. The shootdown of the Navy Privateer was not an isolated incident; from then until the U-2 Incident in 1960, the United States lost nine more aircraft to Soviet air defenses. Each time, the pattern of responses was the same---each side released dueling statements of protest, and the situation continued more or less as before. A B-29 that went missing on October 7, 1952 hit the front page of the \emph{New York Times} two days later. This mission is on Peterson's list as a ferret flight. Farquhar notes that the incident happened during the US election season and that the media thought the attack was intended to lower American prestige, but he does not mention any diplomatic consequences.\footcite[p.~43-44]{farquhar_aerial_2015} The official US response, however, is preserved on the State Department website. As with the Baltic Incident, the US simply demanded that the Soviets pay for the plane and return any survivors.\footcite{the_new_york_times_u.s._1952} Two more shootdowns in March 1953---one involving an American F–84 Thunderjet and the other an RAF Lincoln bomber---actually resulted in conciliatory responses from both sides, including a secret meeting to reduce aerial tensions \enquote{of which little is known.}\footcite[p.~45]{farquhar_aerial_2015}

In another incident on July 29, 1953, the United States protested the shootdown of a B-50 (reconnaissance flight, per Peterson) over the Sea of Japan. The Soviet Union responded by protesting the alleged American shootdown of a Soviet passenger plane. Despite press outrage, no further action was taken.\footcite[p.~47]{farquhar_aerial_2015} A September 4, 1954 shootdown of a Navy P2V Neptune actually led to a call by one US Senator to break off diplomatic ties with the Soviet Union: \enquote{Just another note from our State Department to the Kremlin hierarchy will not impress these uncivilized rulers \textelp{} that this new attack upon an American plane confirms Communist arrogance and aggressiveness to a point where the breaking of diplomatic relations is justified.}\footcite{the_associated_press_ending_1954} Instead of a harsher punishment, President Eisenhower brought the issue to the UN with the full knowledge that an unfavorable judgment would be vetoed by the Soviets. The move demonstrated Eisenhower's ability and commitment to resisting the calls for harsher action while still placating the American public.\footcite[p.~47]{farquhar_aerial_2015} Besides, by late 1954, the \enquote{cycle of hostility} over aerial reconnaissance was already breaking down.\footcite[p.~49]{farquhar_aerial_2015} In terms of both media coverage and diplomatic significance, no shootdown that followed would require more significant consequences than the ones that had already happened.

One incident in the period between 1950-1954 does not fit the broader pattern. On November 20, 1951, an American C-47 transport was forced down in Hungary. This incident is omitted from Peterson's list because, despite Soviet cries of \enquote{spies and saboteurs,} there is no evidence that it was a reconnaissance flight.\footnote{It also wasn't exactly shot down so much as forced to land.} Of all the cases that Farquhar examines, this is the only one that provoked a sustained diplomatic response from the United States---and it is the only case where the plane in question definitively carried no espionage equipment. While the ferret flights were perfectly calibrated to fit as much radio surveillance equipment as possible, the only evidence of espionage that the Soviets were able to produce from the C-47---a plane that they recovered intact---was a portable radio, two extra parachutes, and some packets of warm blankets.\footcite{the_united_press_soviet_1951}

I contend that the lack of espionage equipment is precisely the reason the United States made a sustained public effort to recover the four airmen, which is something that almost never happened with ferret flights.\footnote{While some of the flights were shot down overseas, making recovery difficult, I am referring here to situations where the US was aware that the Soviets likely held their airmen in captivity and still made little effort to get them back.} Farquhar summarizes the American response to this incident as follows: \enquote{Responding to the press attention, the Truman Administration acted swiftly, attempting to gain the fliers release through diplomatic pressure. The President ordered the Hungarian consulates in New York and Cleveland closed and banned private travel to the country. Legislatively, Truman asked Congress to pass a \$100 million Mutual Security Act to aid \enquote{selected persons residing in Soviet bloc states or refugees who wanted to form armed units} in opposition to Communism.}\footcite[p.~43]{farquhar_aerial_2015} Truman immediately took action against material Hungarian interests in the United States and threatened to empower \emph{armed dissidents} if the prisoners were not released. The contrast between this response and the one from just a year prior, when the US demanded \enquote{a prompt and thorough investigation} of the Baltic Incident, is stark.

We also know that the United States conducted many more overflights than the failed ones that are listed here, using a variety of retrofitted planes. In 2001, the Air Force held a symposium honoring and recollecting the pilots who flew the overflights on Soviet territory in the early Cold War, an effort in which they were joined by the British in 1954.\footcite[p.~v]{hall_early_2003} Many of the presenters spoke about entire reconnaissance projects that apparently didn't even register diplomatically. Project Heart Throb, for example, outfitted RB-57s with surveillance equipment and flew between 15 and 19 missions over Eastern Europe from 1955 to 1956.\footcite[p.~194. Based on the recollection of Gerald E. Cooke, a Air Force pilot assigned to the project.]{hall_early_2003} Another project from around the same time, Slick Chick, set up F-100 jets with rapid-fire 20mm cameras. The pilot who presented this project at the symposium, Cecil Rigby, personally flew two such missions and estimated that there were six Slick Chick missions total.\footcite[p.~176]{hall_early_2003}

The USSR knew about these missions, which were tracked by radar and often chased by MiG fighters. Unlike some of the flights that were shot down, these overflights were clear violation of their territorial sovereignty. We will likely never know exactly how many such missions were flown. The Truman and Eisenhower administrations were shockingly unsuccessful at recovering the pilots who were held as POWs by the Soviets. Dino Brugioni, a senior CIA officer involved in the aerial imagery program, paints a grim picture of the institutional accountability involved when the flights went bad. In meetings between the State department and high-level Soviet diplomats, the subject of captured pilots \enquote{was broached only perfunctorily in relation to other things being discussed.}\footcite[p.~72]{brugioni_eyes_2010} Some family remembers of lost airmen received posthumous awards, but most simply received the pilot's personal effects and no explanation. \enquote{And because these were secret missions conducted by a field command,} Brugioni writes, \enquote{a change of field commander meant that the fate of the men was soon forgotten.}\footcite[p.~72]{brugioni_eyes_2010} With missions so secret that we abandoned the men who flew them, we can have no expectation of finding a complete record today.

Even without a complete list of flights, it is clear that the territorial boundaries of the USSR were routinely violated to generate critical intelligence for NATO powers in an early, relatively volatile stage of the Cold War. As time passed these flights appear to have become more accepted, not less. Diplomats of both nations were, as a matter of policy, able to paper over these issues in service of advancing other interests. When the shootdown caused unavoidable diplomatic consequences---such as when it was well-publicized---the countries exchanged the necessary diplomatic cables and engaged in the requisite saber rattling. These incidents undoubtedly altered the climate of Cold War diplomacy, but the fact remains that in 10 years of reconnaissance, the only action the Soviets took to keep the United States from conducting these incredibly provocative reconnaissance flights was to shoot them down.

\subsection{Inventing the spy plane}
While Eisenhower was able to defuse some of the tension surrounding the military reconnaissance flights, even the most aggressive overflights could only hope to skirt the borders of Soviet territory; the United States couldn't just simply fly a B-52 over Moscow. What US intelligence-gathering capabilities needed was some way of photographing suspicious installations deep within Russia while still being able to claim the same benefit-of-the-doubt status that prevented even the most contentious overflights from escalating into a larger conflict. What they needed was a spy plane.

For the spy plane to serve its intended purpose as a minimally provocative agent of espionage two things must be true: its operation must be understood by adversaries to be non-military, and that distinction must meaningfully alter their associated response. To build such an aircraft, Eisenhower approved Project Aquatone---the codename for the U-2. From its conception, the U-2 was purpose-built to be read as a civilian aircraft. Eisenhower required that the pilots of U-2 aircraft be CIA employees, specifically forbidding uniformed Air Force officers.\footcite[p.~33, Though many of the pilots did have Air Force backgrounds.]{lindgren_trust_2000} The CIA also had operational control of the program instead of the Air Force. \enquote{I want this whole thing to be a civilian operation,} Eisenhower wrote, when settling an operational dispute between the two departments. \enquote{If uniformed personnel of the armed services of the United States fly over Russia, it is an act of war---legally---and I don't want any part of it.}\footcite[p.~60. The original source for this quote is an \emph{OSA History} that requires codeword clearance. It is quoted here by the History Staff of the CIA.]{pedlow_central_1992} This allowed the United States to, in the phrasing of the CIA, \enquote{truthfully deny} that any US \enquote{military planes} had flown over the USSR---which they had to do when the inevitable Soviet protest notes were filed after U-2 overflights began in 1956.\footcite[p.~109]{pedlow_central_1992}

The steps Eisenhower took to ensure the U-2 civilian's character went beyond mere deception. If his only goal had been to achieve a level of plausible deniability in the event of a shootdown, then simply not wearing an Air Force uniform during the mission likely would have been sufficient. The classic \enquote{it's a weather reconnaissance craft} excuse had been used before with the ferret flights.\footcite[p.~45]{farquhar_aerial_2015} At the highest levels of American government, however, policymakers didn't just want to pretend that Aquatone was a civilian operation, they wanted it to \emph{be} a civilian operation.\enquote{It is of utmost importance to differentiate in our minds, and to cause the Russians to differentiate in theirs, between Aquatone-type operations and reconnaissance by military aircraft} read a top-secret CIA memo from 1956.\footcite[p.~1]{miller_suggestions_1956} Both halves of that equation were equally important. The Russians must believe that these operations are peacetime civilian operations, and they must be correct about it.

At first, what made the U-2 flights \enquote{count} as espionage appears to be a deeply unsatisfying tautology---they counted as espionage because they were espionage! But Eisenhower's precautions were based on the United State's previous experience with reconnaissance flights. He had already been testing the limits of what the Soviet Union would be willing to accept; if he did everything possible to signal that this plane functioned like a spy, perhaps it would accept this new development as well. No finite list of attributes could ensure that the Soviet Union wouldn't treat the U-2 plane---a massive leap in military surveillance technology---as a military development. But if the people in charge of Project Aquatone could convince \emph{themselves} that it was a civilian project, they stood a better chance of persuading the Russians to interpret it the same way. The U-2 simply had to be a spy plane, in a manner both instantly recognizable and ontologically undefinable.

These precautions had the intended effect, even though it initially seemed like the Soviets were going to extract significant concessions. A Soviet colonel in the USSR Air Defense later recalled that Khrushchev ``clearly viewed the violation of their nation's skies by a foreign reconnaissance aircraft \textelp{} as a political provocation.''\footcite{orlov_u-2_2007} Three days after NASA quietly reported that they had lost a U-2 type weather reconnaissance plane, Khrushchev announced that the USSR had shot down an American plane that had flown into its airspace in \enquote{an aggressive provocation.}\footcite[p.~112]{powers_operation_2004} When the Americans stuck to the \emph{weather craft} cover story, Khrushchev gleefully debunked it by revealing that Powers had been taken alive. In a press conference where Eisenhower memorably declared that espionage was a \enquote{a distasteful but vital necessity} to prevent another attack like Pearl Harbor, the President admitted that the Gary Powers had been flying a spy plane.\footcite{eisenhower_news_1960}

And yet few real consequences followed. No one disputes that the collapse of the Paris Summit was directly attributable to the Gary Powers shootdown but, in practice, summits aren't all that important. From 1953 to the end of the Cold War in 1991, there were 23 total US-USSR summits.\footcite{fain_chronology_2011} The main issue that did not get resolved in 1960 was the status of Berlin. It also did not get resolved at the next summit in 1961, a comparatively warmer affair between Khrushchev and Eisenhower's successor, John F. Kennedy. If the worst that happened because of the U-2 incident was an unproductive summit meeting, then that is a relatively minor punishment. With the late rollout of the captured pilot, the dramatic scene at the summit, and the cancellation of Eisenhower's visit, Khrushchev publicly embarrassed the United States without meaningfully altering Soviet policy towards it.

Nor did the U-2 Incident end all reconnaissance flights. As is characteristic of espionage, when the US really needed intelligence that only an overflight could provide, the diplomatic consequences that the USSR had imposed were not sufficient to discourage it. Two months to the day after Powers was shot down, a Soviet MiG-19 opened fire on an American RB-47H and captured the plane's navigator and co-pilot. While Powers was serving a 10-year sentence in Soviet prison, the two airmen were returned to the US in less than six months with only mild fanfare. Khrushchev wrote in his memoir that he wanted to continue \enquote{our general line of peaceful coexistence} and cited the resolution of this incident as an example of such, because the US was forced to make a formal request for the return of the airmen.\footcite[p.~256-257]{khrushchev_memoirs_2007} The only concessions the US made in return were to announce the discontinuation of its U-2 overflights (which Kennedy was already committed to) and to not make an issue of the illegal detention of the pilots.\footcite{time_cold_1961} For spy flights after Powers, the USSR simply took the necessary immediate countermeasures and shot down the planes that it could. Just as with traditional espionage, the countermeasures did not punish the attempt.


\subsection{Espionage norms in aerial reconnaissance}
The U-2 Incident is often cited as a seminal failure of the Eisenhower administration. In the traditional telling, it significantly worsened US-Soviet relations at time when it seemed like lasting peace might be possible. This interpretation is incomplete, if not inaccurate. Instead, the U-2 Incident should be understood as an embarrassing but otherwise typical failure of an intelligence operation, one consistent with the minimal consequences typically applied to espionage.\footnote{While I argue that the response to the U-2 flight was within the reasonable bounds for espionage, I acknowledge that it was still unprecedented in the context of reconnaissance flights. There are a number of reasonable explanations for why the U-2 shootdown escalated. Partially the Soviets felt humiliated by a technology for which they had no countermeasures and the uniquely insulting way the overflights flew deep into Soviet territory. Eisenhower thought that Khrushchev might have been looking for an excuse to cancel his visit to Russia. See: \cite[p.~555]{eisenhower_waging_1965}. Others in his administration believed that the U-2 Incident was an excuse to justify an intensification of hostilities that had already been decided by hard-liners in the Kremlin. See: \cite[p.~328]{kistiakowsky_scientist_1976}.} What would a major punishment have looked like? One proportional response Khrushchev could have considered was to press harder for the remove of regional air bases. Since the US was using nearby bases to launch invasive reconnaissance missions, the USSR could have demanded that the US either close some of them or agree to halt surveillance flights. This likely would have involved some sort of trade, but it is not an unreasonable hypothetical. The Soviets were incredibly preoccupied with removing the forward-deployed Jupiter missiles from Turkey. \footnote{Although the missiles were deployed to bases well within NATO territory, and arguably less threatening to Soviet security than American overflights of the Soviet bloc, the existence of these missile became a central bargaining chip in the Cuban Missile Crisis.} By contrast, the Soviets simply did not have the same level of interest in curtailing reconnaissance flights. That both sides of the Cold War had strong incentives to avoid military escalation is well-documented, but at least threatening military retaliation for launching provocative military operations would not seem to have been out of place.

Most scholars who study this period argue that the cumulative tension created by these flights contributed to an overall chilly tone in US-Soviet relations. Farquhar, for instance, refers to this as the \enquote{cycle of hostility,} in which the series of aerial incidents increased suspicions of military buildup on both sides, leading to even more aerial incidents as both powers attempted to verify their suspicions.\footcite[p.~43]{farquhar_aerial_2015} While it's true that the there were many aerial incidents, it is not at all clear how this affected any other aspect of US-Soviet diplomacy. Consider the period of time over which these flights took place; if aerial reconnaissance had risen to anything more than a lingering irritant, then it should have to have been resolved in some way before 1960. The US-Soviet relationship saw huge shifts, including periods of optimism, over the 11 years preceding the U-2 incident. During this period, hundreds of flights took place in contentious territory---many of them purposefully violating Soviet boundaries, causing some of them to be shot down---and the most significant consequence was a failed summit meeting, after which the flights continued on almost as before. Both sides were so willing to contain the issue that, tragically, many of the airmen never saw their families again.

Until the day he died, Khrushchev greatly exaggerated the lasting effect of the U-2 Incident. He wrote in his memoir that \enquote{the commander of American forces in West Germany gave the order not to fly any closer than 50 kilometers from the border between East and West Germany. And no more incidents of that kind occurred.}\footcite[p.~256]{khrushchev_memoirs_2007} His son Sergei, who annotated the memoir, admitted that this statement was false: \enquote{In practice such incidents occurred again from 1961 onward, but instead of the U-2 the Americans now used various types of Phantom or SR-71.}\footcite[p.~258]{khrushchev_memoirs_2007} The United States continued to use the U-2 extensively in other regions as well.\footnote{A recently declassified CIA report details the many locations in which U-2 photography was employed to gather intelligence after 1960, including China, India, Indonesia, Thailand, Tibet, Laos, North Vietnam, and Venezuela. Lyndon Johnson even clarified that his predecessor's order to end U-2 flights over the Soviet Bloc was not indefinitely binding---it was valid only until countermanded. For more, see: \cite[p.~195]{pedlow_central_1992}}. On at least two occasions after the U-2 Incident, an American overflight crossed over into Soviet territory and the Soviets responded with... a diplomatic protest. The US quickly apologized, and that settled the matter.\footcite{orlov_u-2_2007}

Both heads of state made clear in their writings that they considered Powers to be spy. On capturing Gary Powers, Khrushchev wrote: ``This was a hostile act by the leaders of the U.S. government, and they made no attempt to conceal it. They didn't think we had the capability of \textelp{} acquiring irrefutable proof that the United States was using methods that were impermissible in peacetime.''\footcite[p.~239]{khrushchev_memoirs_2007} Sergei's annotations once again contradicted his father's claim that these methods were impermissible; more than forty American reconnaissance aircraft were shot down by the USSR during the Cold War. Eisenhower, an insightful thinker on issues of intelligence, understood that beneath the grandstanding the fundamentals of international espionage still applied. \enquote{[Khrushchev's] government,} he wrote, \enquote{had been so notoriously involved in spying, especially in the United States, as to dwarf our activities, but by separating this particular type of espionage from all others he hoped to make convincing his charge of \enquote{warmongering.} To claim that because the equipment employed was an airplane with a camera, and therefore provocative of war, was plain silly, and I felt it necessary that the matter be put in perspective.} \footcite[p.~551]{eisenhower_waging_1965}

% Add slight emphasis here that the U-2 was built as a spy plane because it would have a certain reaction, and it worked

Project Aquatone channeled the norms of espionage in the most crucial respect: its strategic value easily outweighed the consequences of being caught. No one understood this better than President Eisenhower, who, when he was occasionally questioned about the wisdom of the U-2 flights, would always reply: \enquote{Would you be ready to give back all of the information we secured from our U-2 flights over Russia if there had been no disaster to one of our planes in Russia?} In his telling, he never received an affirmative response.\footcite[p.~559]{eisenhower_waging_1965}


\section{Spy satellites}
Gary Powers had been in Soviet prison for less than three months when the United States launched an instrument of espionage far more advanced, and far more damaging, than the U-2 could ever be. The first generation of spy satellite had limitless airspace access, no countermeasures, and no technological antecedents. It was impossible to predict how the the international order would react; in 1955, when the US and the USSR both announced their plans to build an artificial satellite, Outer Space was a true blank slate for international norms.

Only four years after the first U-2 overflight, a familiar dynamic could have played out when the first spy satellite launched in August 1960: the US develops a new photo-reconnaissance technology which its main adversary is unable to prevent; the USSR diplomatically objects to the technology's use; the USSR finally develops a weapon capable of counteracting the new technology; and the USSR employs it. Each one of these steps came to pass except the last---even after developing a feasible anti-satellite (ASAT) weapon, the Soviet Union never made use of it. And the onus was on the Soviets to push back against American efforts to normalize satellite reconnaissance. \emph{Sputnik} briefly put the USSR at the forefront of the final frontier, but the satellite itself was totally innocuous; it collected data about the density of the atmosphere and propagation of radio waves.\footcite{nasa_sputnik_2019} It was the United States that would first launch a satellite capable of photo-reconnaissance---and it was the Soviet Union that allowed it to happen.

\subsection{Freedom of space}
Spy satellites are not just one of the best edge cases for norm against shooting down a satellite, they are the reason it exists in the first place. The reconnaissance satellite succeeded in its mission because the Eisenhower administration purposefully designed its space program to establish a norm of peaceful development in space---with the single goal of legitimizing satellites for intelligence-gathering. No military action was ever been taken against satellites, and the Soviet Union only objected to reconnaissance satellites until it was able to make its own. By protecting the peaceful uses of outer space, the United States finally gained unrestricted access to aerial intelligence. That all satellites, regardless of purpose, are immune to hostilities as a result is just a happy byproduct.

In 1954, President Eisenhower established the Technological Capabilities Panel (TCP), a task force which would tackle the problem of America's vulnerability to a surprise attack. When the panel returned with its results, its report concluded that any great power conflict would result in heavy American losses, so all necessary steps should be taken to avoid one.\footcite[p.~67. The TCP report is also variously referred to as the \enquote{Killian Report,} \enquote{Surprise Attack Study,} or its formal title, \enquote{Meeting the Threat of a Surprise Attack.}]{killian_sputnik_1977} Its recommendations obviously included furthering American missile capabilities---traditional deterrence through force---but intelligence-gathering capabilities took comparable priority. ``We \emph{must} find ways to increase the number of hard facts upon which our intelligence estimates are based,'' the TCP Report states, ``to provide better strategic warning, to minimize surprise in the kind of attack, and to reduce the danger of gross overestimation or gross underestimation of the threat.''\footcite{technological_capabilities_panel_meeting_1955} The TCP recommendations---reviewed personally by President Eisenhower---were hugely consequential across US government policy, but two projects that came out of them are especially relevant here: the U-2 plane and the artificial spy satellite.

The U-2 plane made the leap from concept to reality much faster than the artificial satellite, but even before the Soviet Union became capable of shooting it down, the intelligence that the U-2 provided had limitations. While the U-2 provided CIA imagery analysts with impossibly high-resolution photos of specific locations, it was less helpful in finding unknown military bases throughout the vast expanse of Soviet territory. The first generation of artificial satellites promised to change that. In the time it takes a satellite to complete a full orbit, the Earth has rotated slightly underneath it, altering the satellite's field of view---with each pass an artifical satellite launched into polar orbit will photograph an area that has been longitudinally shifted.\footnote{Think of a winding a piece of string around a ball with your right hand, while slowly rotating the ball with your left; after a series of passes, you've covered a broad slice of the ball with string.} With enough orbits, that satellite can photograph a wide slice of Earth. Together, planes and satellites could operate for complementary photo-reconnaissance purposes: a plane dispatched when high quality localized intelligence was required, and a satellite to form a comprehensive picture of the Soviet military posture. And of course, satellites could surveil dangerous territory without risking the life of a pilot.

Preventing a surprise nuclear attack could not be done without better intelligence and better intelligence could not be achieved without comprehensive photo-reconnaissance---reconnaissance performed using a technique that the Soviet Union would accept. The solution was to spy from space. \enquote{Camera-toting satellites,} historian Walter McDougall explains, \enquote{circling the earth south to north [sic] in a polar orbit, could view the entire surface of the earth as it rotated below, return to any location in a few days' time, hone in on suspicious areas, and do it all under the legal cover of freedom of space---if such legal cover could be established.}\footcite[p.~117]{mcdougall_heavens_1985}

That legal cover could be established was not a given. The Soviets considered U-2 overflights to be a violation of their territorial sovereignty, and continued to register diplomatic protests about them; the overflight program was so risky that it was originally intended to last only a year or two.\footcite[p.~33]{lindgren_trust_2000} The undefined legal status of space presented an opportunity, but also a challenge. The TCP understood that the Soviet Union would recognize the enormous advantage a photo-reconnaissance satellite conferred on the US, and object to its deployment using any international norms possible. It was therefore necessary to present reconnaissance satellites in a manner that would limit the grounds for objection. By the time that a reconnaissance satellite launched, it would have to have already been established that any nation had the unquestioned right to launch a satellite of any kind; this principle would come to be called \enquote{freedom of space.}

\subsection{Space is not a race}
Everything about the administration's space policy was carefully crafted to set the stage for the eventual launch of a spy satellite. From the time that artificial satellites first appeared viable until the day he left office, Eisenhower used the civilian space program to divert attention from the security-related space programs he valued.\footcite[p.~119]{day_eye_2015} Acting on the recommendations of the TCP, the National Security Council (NSC) proposed a scientific satellite program in May 1955, which Eisenhower accepted. In this memo, NSC-5520, the NSC Planning Board argued that, when framed properly, a scientific satellite could be used as diplomatic tool to promote a norm establishing ``freedom of space.'' The proposal tracked closely with a similar one from the 1950 RAND report, which suggested that the US first launch an innocuous ``experimental satellite'' that did not cross over Russian territory, and gauge the diplomatic reaction.\footcite[p.~21]{kecskemetic_satellite_1950} The political framing of the satellite is discussed at the top of NSC-5520, in a section entitled \emph{General Considerations} (emphasis mine):
\newline

\begin{quoteblock}
6. \textelp{} \textbf{Furthermore, a small scientific satellite will provide a test of the principle of “Freedom of Space”. The implications of this principle are being studied within the Executive Branch.} However, preliminary studies indicate that there is no obstacle under international law to the launching of such a satellite.

7. \textbf{It should be emphasized that a satellite would constitute no active military offensive threat to any country over which it might pass.} Although a large satellite might conceivably serve to launch a guided missile at a ground target, it will always be a poor choice for the purpose. A bomb could not be dropped from a satellite on a target below, because anything dropped from a satellite would simply continue alongside in the orbit. \footcite{nsc_planning_board_draft_1955}
\newline
\end{quoteblock}


Even though there existed no international law prohibiting the use of satellites, a theoretical ``freedom of space'' principle depended on the first satellite not having military purposes. What NSC-5520 did not anticipate was that the US wouldn't have to establish that principle on their own. On October 4, 1957, the Soviet Union launched the world's first artificial satellite, a small metal ball named \emph{Sputnik}.

Getting beaten to orbit by \emph{Sputnik} put the United States at an advantage. The Soviets could hardly denounce the American satellite program as suspicious when they very publicly made the first move---and without asking permission to orbit over the United States.\footcite[p.~40]{peebles_corona_1997} The US didn't \emph{try} to lose first leg of the space race exactly, but policymakers were aware well in advance that being the second nation to launch an artificial satellite would further their strategic goals. Reviewing NSC-5520, Nelson Rockefeller wrote that since the Soviets had already announced plans to launch a satellite, the existence of the Soviet project could be used to good effect against inevitable Soviet protests against American photo-reconnaissance satellites.\footcite[p.~120]{mcdougall_heavens_1985} Though ideally the US would set the precedent for ``freedom of space'' itself, senior administration officials understood that if the USSR launched first, then the American satellite program would be almost automatically justified.

Not only did the Eisenhower administration recognize that being second to orbit came with a significant consolation prize, it was not particularly focused on winning the space race to begin with. Launching the world's first artificial satellite was assigned the lowest possible priority because the administration correctly determined that its other goals---finishing the ICBM, monitoring Soviet R\&D, and framing the satellite program as a civilian scientific endeavor---were of greater strategic importance.\footcite[p.~123-124]{mcdougall_heavens_1985} Eisenhower himself repeatedly dismissed concerns about the prestige of being the first nation to orbit an object in space, openly contemptuous of the idea that space was any kind of race at all.\footcite[p.~100]{lindgren_trust_2000} His complete disdain for what he called ``stunts'' would prove to be disastrous politically, but his restraint was crucial to establishing a strategically beneficial precedent.\footcite[p.~134]{day_eye_2015}

In fact, the overriding American priority to legitimize reconnaissance satellites is probably the \emph{reason} that the Soviets started the Space Age instead. Just as Eisenhower limited the involvement of the Air Force in Aquatone to paint the U-2 as a civilian project, he made technical decisions that were intended to make the scientific satellite seem as non-threatening as possible---decisions that directly impacted the project timeline. The group tasked with implementing NSC-5520 had to choose between competing proposals from the Army, Navy, and Air Force for which rocket would launch the satellite. It made the somewhat surprising choice to go with the unproven Navy Research Laboratory (NRL) \emph{Vanguard} rocket, even though the Army Ballistic Missile Agency (ABMA) engineers, headed by famous rocketeer (and ex-Nazi) Wernher von Braun, had been developing their proposal for a year. Among the reasons the committee gave for their decision, two stand out. First, the NRL impressed the committee with its plans for scientific components and a radio tracking system.\footcite[p.~122]{mcdougall_heavens_1985} Second, the panel specifically wanted to avoid using a ballistic missile to launch a scientific satellite.\footcite[p.~129. Though the NRL is run by the Navy, it functioned like a civilian scientific operation and all parties involved believed that the NRL proposal would be seen as a civilian project---especially in comparison with its competetor, the Army agency building nuclear delivery systems.]{day_eye_2015}

Launching the first artificial satellite with a ballistic missile threatened to compromise both of the principal national security priorities. It might delay the development of the ICBM, an existential threat to American military superiority, and it would jeopardize the civilian character of the satellite program, which the committee members knew was necessary to establish ``freedom of space.''\footcite[p.~122]{mcdougall_heavens_1985} The Von Braun engineers vigorously protested the decision, because they believed---correctly---that their proposal would succeed first. But being first in space was not the goal, establishing a legal precedent was. Von Braun's appeals were continually rejected, and always for the same reason: launching a satellite with no scientific purpose other than demonstrating superiority would compromise the moral high ground which the US desperately needed to secure.\footcite[p.~131]{day_eye_2015}

Publicly, \emph{Vanguard} was a disaster. The rocket experienced an embarrassing setback on December 6, 1957, when it failed to launch a three-and-a-half pound satellite, just after the USSR had successfully launched \emph{Sputnik II}.\footcite[p.~119]{killian_sputnik_1977} This time, the satellite carried a dog---the Soviets put a dog in space before the Americans even managed to get a satellite up there.\footcite[Her name was Laika.]{george_sad_2018} When then United States finally succeeded in launching at satellite, it was not thanks to the \emph{Vanguard}, but rather the ABMA team that had been originally shunned by the administration. Finally given the chance, Von Braun and his engineers quickly accomplished what they had always said they would: modify a Jupiter-C ballistic missile and send a satellite into orbit. The Army launched \emph{Explorer 1} on January 31, and it carried a pioneering array of miniaturized electronics, including two micrometeoroid detectors and a Geiger counter.\footcite[p.~168]{mcdougall_heavens_1985} \emph{Vanguard 1}---which finally launched about six weeks later---and \emph{Explorer 1} provided an impressive amount of scientific data in comparison to the Sputniks.\footcite[p.~168]{mcdougall_heavens_1985} More importantly, even though the US did end up launching its first satellite using a ballistic missile, its visible defeat in the space race and clear scientific aims had the intended effect of blunting Soviet efforts to characterize the American satellite program as a military development. It was time to launch a spy satellite.

\subsection{The Corona Project}
The Soviet Union unintentionally set an early precedent for ``freedom of space,'' but that did not automatically justify using space for reconnaissance. As far as the international community was concerned, it had only been established that overflights were permissible for scientific purposes; military reconnaissance was still an open question.\footcite[p.~47-48]{peebles_corona_1997} Now that the space age was underway, the next step was to normalize photographic satellite overflights. Success was going to require tight discipline across all levels of government, but information about WS-117L, the Air Force reconnaissance satellite program, had already begun to leak.\footcite[p.~96]{lindgren_trust_2000}

Like many of Eisenhower's space policy decisions, his solution for reconnaissance satellites was a brilliant long-term success---but a complete mystery to the public and utterly baffling to most of the military establishment. WS-117L was originally intended to be a family of reconnaissance satellites. Eisenhower simply selected the most immediately promising project---a single-use film-retrieval satellite, codenamed Corona---and canceled it. Air Force personnel were ``thunderstruck.''\footcite[p.~45]{peebles_corona_1997} With public bewilderment instilled and American reconnaissance satellites now seemingly a few years out, the administration began a multi-pronged approach to legitimizing spy satellites. Publicly, the US government advocated for a ``freedom of space'' policy that explicitly included reconnaissance satellites; secretly, it resurrected the Corona project under deep-cover CIA administration.

The public half of Eisenhower's strategy required international approval for space-based overflights, so the US advocated for a United Nations Ad Hoc Committee on the Peaceful Uses of Outer Space (COPUOUS) to deal with this issue. The Soviets, worried about having the limitations of their ICBM program exposed, refused to participate for years.\footcite[p.~140]{day_eye_2015} Meanwhile, the actual spy satellite program progressed at an accelerated pace, and with unmatched secrecy. The team started with only 30 personnel and eventually swelled to 300, almost none of whom knew the entire scope of the project. Team members had to take different routes to work to avoid being followed and were never allowed to say the word ``Corona'' on the telephone, or even the abbreviation ``C.''\footcite[p.~51]{peebles_shadow_2000} As far as the government was concerned, Corona's cover was bulletproof. When the satellite began making test launches, the US heavily promoted its new \emph{Discoverer} program as a scientific achievement---confident that its status as a front for the Corona project could not be proven false the way the U-2's cover had been. When it finally succeeded in recovering a \emph{Discoverer} satellite from orbit, the US government released photographs of the first successful capsule triumphantly floating in the water, and an American flag retrieved from within was presented to the president.\footcite[p.~83]{peebles_shadow_2000}

Today's Earth-imaging satellites have enormous solar panels, radio their images down to command stations, and have mission durations in a scale of years.\footnote{This design simply did not work with film. A rival project that developed the film midair and transmitted the data back down failed to provide intelligence of sufficient quality. (\cite[p.~203-204]{brugioni_eyes_2010}) The last film-based orbiting reconnaissance camera developed for the U.S. Government was the KH-9 HEXAGON, which operated between 1971 and 1986. It was famously complicated, with four separate film-reentry pods (\cite{pressel_spy_2013}). I presume the rest since then have been digital, but because the HEXAGON was only declassified in September 2011, it will likely be quite a long time before we find out how they work.} Corona was nothing like that. Instead, each satellite was little more than a monstrously expensive disposable camera. The satellite would be sent into space carrying enough film for a 24-hour mission and its camera would switch on every time the satellite flew over denied territory. After completing seventeen orbits, the satellite would pitch down 60 degrees and eject the film capsule back down to Earth. Various rocket systems would stabilize it, a heat shield would protect it, and a series of parachutes would decelerate it. What remained of the capsule would hopefully descend within a 200-by-60 mile recovery zone (``the ballpark''), but at the very least within the boundaries of an additional 400-mile section (``the outfield''). Ideally, it would be caught in midair by a cargo plane.\footcite[p.~56]{peebles_corona_1997}

Why is that significant? Because at no point was satellite reconnaissance a \emph{fait accompli}---each launch was another opportunity for the Soviet Union to object to the continued use of a potentially illegal surveillance method. And there were a great many launches. From February to August 1960, right in the middle of the U-2 incident, twelve test launches failed before the US achieved the first successful recovery of a Corona satellite---of any object from orbit in the human history, for that matter---when it fished \emph{Discoverer 13} out of the sea. The US repeated that feat with \emph{Discoverer 14} a week later, and this time it carried a camera, bringing invaluable intelligence back from space that dealt a mortal blow to the myth of the missile gap.\footcite[p.~101-102. The previous failures had been so demoralizing that \emph{Discoverer 13} only carried diagnostic equipment, which made its success all the more exciting. Ships on standby were able to retrieve the satellite even though it fell outside the recovery zone---\emph{Discoverer 13} many not have landed in the ballpark or the outfield, but in many respects, the mission was a home run.]{lindgren_trust_2000} The launches kept coming---the next one to return high quality photos was \emph{Discoverer 18} in December. Eisenhower left office soon after, but between \emph{Discoverer 18} and \emph{Discoverer 38}, twenty launches were performed, two of which carried actual radiometry equipment (in support of Corona's cover story), and seven produced film of high enough quality for the mission to be considered a complete success.\footcite[p.~103]{lindgren_trust_2000}

Like the U-2 overflights, this intricate process was immediately identified by the Soviets for what it was: a military reconnaissance effort. The Soviet Union protested the very first \emph{Discoverer} launch, correctly alleging its military utility even though the satellite had failed in its purpose, and would fail eleven more times.\footcite[p.~140]{day_eye_2015} Until 1963, Soviet propaganda continually railed against the Americans, in domestic media and international forums, for attempting to militarize space.\footcite[p.~271]{mcdougall_heavens_1985} The USSR sponsored a UN resolution to prohibit espionage from space, and Khrushchev vowed that his military would shoot down spy satellites, just as they had the U-2.\footcite[p.~166]{day_eye_2015} In the first few years after the United States developed satellite reconnaissance capability, it appeared that the USSR was acting exactly as the 1950 RAND report had predicted, and using the legal means at their disposal to end the satellite overflights.

Khrushchev had every incentive to discourage satellite reconnaissance, because the intelligence that the US gained through Corona was absolutely devastating to Soviet claims about their military prowess. The U-2 overflights had sufficiently debunked the ``bomber gap,'' and seemed to suggest that no ``missile gap'' existed either.\footcite{goodpaster_cold_2003} But as the CIA report on U-2 photography showed, being unable to find ICBM sites was not conclusive proof that they did not exist, and the vast majority of possible missile sites remained un-photographed. Confusion about Soviet long-range attack capabilities still existed at various levels of government, confusion which ultimately led to the doomed flight of Gary Powers.\footcite[p.~344]{brugioni_eyes_2010} It was the \emph{Discoverer 14} and \emph{Discoverer 18} missions that successfully photographed the sites that intelligence analysts considered most likely to host operational ICBMs, and found nothing.\footcite[p.~379]{brugioni_eyes_2010}

To make matters worse for the Soviets, the moment at which the US discovered that the USSR had far fewer missiles than they claimed was the exact moment at which that disparity was greatest. The missile gap actually went in the other direction---in 1962, the US had more than double the ICBMs, maintained a substantial lead in manned bombers, and had just begun to deploy nuclear submarines.\footcite[p.~251]{mcdougall_heavens_1985} By the time of the 1960 presidential election, both sides of the Cold War had a generally accurate read on each others' military capabilities, and they knew that their adversary was properly appraised of the strategic situation as well.  If Khrushchev genuinely feared an American attack, absolutely nothing would have been a greater threat to Soviet security than the collapse of the USSR's most significant deterrent argument---a collapse brought about by American spy satellites.

Not only did Corona make a theoretical offensive attack more likely, it made the plans for such an attack significantly more effective. The impact of the project on military planning is best summarized by how it affected the joint American-British plan for an offensive against the Soviet Union, which aerospace historian Curtis Peebles describes here:

\begin{quoteblock}
Starting in 1962, the Corona photos were also used for nuclear targeting. Corona photos provided the first accurate coordinates of Soviet military facilities. This resulted in a shift in U.S. policy. During the 1950s, little was known about the location of Soviet military bases. The lack of information was reflected in the first join SAC-RAF Bomber Command strike plan. The plan, which went into effect on October 1, 1958, assigned 106 Soviet targets to the RAF bomber force: 69 cities, 17 airfields, and 20 air defense targets. A full two-thirds of the targets were cities. This disparity was ever greater with the unilateral RAF strike plan. If Great Britain should be forced to attack alone, ninety-eight Soviet cities would be bombed.

On August 1, 1962, a revised joint plan went into effect. The changes, which reflected a full two years of Corona operations, were profound. The RAF Thor IRBMs and bomber force would strike forty-four ``offensive capability'' targets such as airfields, ten ``defensive capability'' targets such as air defense control centers, twenty-eight Soviet IRBMs sites, and only sixteen cities. Urban targets had gone from the majority to next to last in number. The unilateral British plan also underwent a similar change.\footcite[p.~139]{peebles_corona_1997}
\end{quoteblock}

The Soviet Union lost its most significant military advantage over the West in a little under two years. No subsequent missile buildup or bunker hardening or second-strike capability could undo the damage done to the Soviet military posture when \emph{Discoverer XIV} and its successors sailed past the Iron Curtain. Aerial reconnaissance neutralized one of Soviet Union's asymmetric structural advantages---the inherent secrecy of Soviet society---and put a dent in their claims to an arms advantage as well. And yet the only thing the Soviet Union did in response was publicly protest the spy satellites, and refuse to accept American proposals for a legal regime that included the rights of reconnaissance overflights. Those protests continued for about 3 years, until the Soviets developed their own working spy satellites, ``freedom of space'' became a mutually-agreeable reality.\footcite[p.~271-275]{mcdougall_heavens_1985}

\subsection{Espionage norms in satellite reconnaissance}
The legitimization of spy satellites dealt an indisputable blow to Soviet standing, so why did the Soviet Union let it happen? The partial explanation is that the US played a good hand very well. If the US had not been engaging in multinational talks about \enquote{scientific} freedom of space while secretly launching spy satellites, then it is very likely that upon developing ASAT capabilities, the USSR would have simply started shooting those satellites down. That would have been an almost identical situation to the U-2 saga, and it was essential that when ASATs became available, the USSR not presume to have the unquestioned right to use them. It was likewise crucial that the US built a successful reconnaissance satellite in secret. The Soviet Union was unlikely to launch the first spy satellite and give the US a perfect excuse to neutralize the intelligence gap, so the US had to do it on their own. The situation that resulted was a perfect storm for the normalization of satellite reconnaissance: the Soviet leadership found that they had the ability to develop a reliable spy satellite much sooner than they could develop a reliable anti-satellite weapon, of which their right to use would still be uncertain. The path of least resistance, but greatest preservation of pride, was to simply start spying themselves.

% But if the US had not been simultaneously running a covert program before the norm was established, then how would the norm have ever been established at all? At what point would the United States have determined that it was officially legal to launch a spy satellite? Soviet intransigence on COPUOS demonstrated that they were unwilling to formally acknowledge the use of spy satellites, so no legal solution was forthcoming.

Nikita Khrushchev played an unwitting role in legitimizing satellite reconnaissance as well. At the infamous four-powers summit in 1960, after watching Khrushchev vent about the U-2 Incident, French president Charles de Gaulle calmly told him: ``You're making too big a fuss about the matter. There is probably a ton of Russian iron [referring to Russian satellites] coming through French space every day without my permissions. I have no idea what is inside those satellites and you have not told me. But I am not making a big fuss.'' Khrushchev admitted that the Soviets had put cameras in some satellites, but not necessarily any that flew over France.\footcite[p.~353]{brugioni_eyes_2010} He went on to say that he only considered airplanes to be out of bounds, not satellites. In Eisenhower's later paraphrasing, Khrushchev said that ``any nation in the world who wanted to photograph the Soviet areas by satellite was completely free to do so.''\footcite[p.~556]{eisenhower_waging_1965} Even being the first nation to space played a role. Of course, it cannot be said that launching the first satellite in human history was a \emph{mistake}---Eisenhower underestimated the symbolic importance of igniting the Space Age, to his significant political detriment---but it is indisputable that \emph{Sputnik} made it much harder for the USSR to argue that reconnaissance satellites were an imperialist invasion of territorial sovereignty.

And yet, American strategy and Khrushchev's missteps still do not fully explain why Soviet Union continued to permit photo-reconnaissance after ASATs were developed. The weapons were feasible; the USSR started testing its ASAT capabilities in 1968.\footcite[p.~149]{karas_new_1983} Given that the Soviet Union was using spy satellites as well, shooting down an American one would have been hard to justify, but it really only risked having some Soviet satellites shot down in response. Instead, the USSR choose to help write and ratify the Outer Space Treaty (OST) in 1967, the first legal regime codifying the peaceful uses of space. While the OST did not ban ASATs, it nonetheless set a precedent for the peaceful uses of space---one that validated the US policy stance that the main military value of space was as a harbor for reconnaissance satellites.\footcite[p~16]{mowthorpe_militarization_2004}

% Took out Welton Chang, ask if he should come back?

Some might argue that the USSR simply miscalculated the relative gains that they received from a norm protecting all satellites. This argument is unpersuasive. Khrushchev made it clear in his memoirs that he rejected the Open Skies treaty because he knew it would reveal the weakness at the heart of the Soviet state---that its missile program was nowhere near as developed as Khrushchev claimed. The USSR would not reach missile parity with the United States until 1969, after both sides had tested ASAT weapons.\footnote{There is obviously a lot more to relative military strength than just adding up the number of missiles that each side has access to---deployment vehicle, missile quality, and susceptibility to a first-strike are crucial. But number of missiles is an easily-digestible metric that politicians pay attention to, and helps visualize just how far behind the Soviet Union was when their actual strength was revealed.} After that, the USSR might have had less to hide, but the fact remains that it still had the opportunity to ``turn off the lights'' by taking action against American spy satellites, something the United States, with its free press and comparatively open borders, could not do. Instead, Open Skies began the moment the USSR launched its first spy satellite, and it never ended.

% The United States fully believed it had more to gain from a ``freedom of space'' norm---and I agree---but is it possible that the Soviet Union had some reason to believe the same? Perhaps, but given the indisputable advantage that the US gained from reciprocal reconnaissance, it is hard to imagine what non-reconnaissance military function a satellite could perform that would advantage the Soviet Union to the same degree. Satellites were also used to improve the accuracy of missile guidance systems, but since both sides relied on nuclear missiles as their main deterrent, satellite-based missile targeting likely provide no relative gains. Or perhaps the Soviet Union thought being the first nation to shoot down a satellite would adversely impact a key propaganda point: that the United States was responsible for the militarization of space. Both the US and the USSR were preoccupied with demonstrating that their respective ideology---capitalism or communism---had an inherent moral superiority. They each believed that this propaganda helped project their power in the third world, but that also raises a question: why would the world react adversely to Soviet defense of its own borders, if the United States did in fact break the glass on space militarization?

Satellites are the purest demonstration of the appeal of espionage norms, because unlike with planes, there were no previously existing norms of territorial integrity. Therefore the only established norm at the dawn of the space age was one from the world of espionage---that intelligence-gathering entities are not inherently aggressive. Space became an environment where a strike against a satellite would constitute the first act of aggression, even if the satellite was operating for a military purpose, gathering information that could significantly compromise a state's security position.

There is a fundamental appeal to a world with tacit mutual reconnaissance, an appeal that is rooted in defensive realism and manifested in espionage norms. Spy satellites are the single best proof that even when the relative gains are lopsided, states will choose to live in a world with more information, rather than less. The advent of mutual reconnaissance solved what had, until then, been the single most intractable problem of the Cold War---treaty verification. Rae Huffstuttler, later a director of the CIA's National Photographic Interpretation Center (the same institution that conducted the survey of U-2 photography), said that ``imagery set the stage for the arms limitation talks \textelp{} With imagery, we could go to the numbers-based strategic arms limitations negotiation with a high degree of confidence.''\footcite[p.~403]{brugioni_eyes_2010} With satellite imagery in hand, the United States could enter the negotiations with appropriate estimates and leave with a means of verifying that the treaty had been carried out. Khrushchev claimed for years to be in favor of arms control, but talks always collapsed because of the US demand for on-site inspections, which the Soviet Union always refused.\footcite[p.~255]{mcdougall_heavens_1985} Satellites served the same purpose, and the Strategic Arms Limitation Treaty of 1972 (otherwise known as The ABM Treaty, or SALT 1) enshrined in law that signatories are not permitted to interfere with each others' ``national technical means of verification,'' a tacit ban on ASAT weapons.\footcite[p.~431]{mcdougall_heavens_1985} Without having to say ``spy satellites,'' SALT 1 made them not just legal, but essential.

Though there is no chance that Khrushchev intended this, the intelligence that satellite photo-reconnaissance provided resulted in the United States building \emph{fewer} missiles and resisting the temptation to engage in an arms race. The success of the SALT treaties was a significant factor in the stabilization of relations between the two countries---it created an atmosphere of trust and cooperation that, while strained by Reagan the and the invasion of Afghanistan, held for the remainder of the Cold War.\footcite[p.~179]{lindgren_trust_2000} ``Freedom of space'' exists because both sides preferred a world where they could each verify each others' intent over one where they were shooting blind.

\section{Conclusion}
\end{document}

% The breadth of information that a satellite could provide might also dispel the Eisenhower administration's political albatross---his supposedly dovish defense spending. Eisnhower faced constant pressure to increase the defense budget and close the alleged ``missile gap'' between the US and the USSR. While U-2 photography (codenamed ``Talent'') had been helpful for pushing back against these demands, the diplomatic volatility of each flight limited the amount of ground that it could cover. In July 1960, the CIA's Photographic Interpretation Center conducted a study to determine how much of the Soviet territory that it deemed suitable for ICBM deployment had been observed by Talent photography since January 1959. As it turned out, very little. The ``intelligence bonanza'' from Aquatone had still only observed roughly 13.6\% of the viable Soviet terrain. Visual observations from other sources added an additional 1.5\%.\footcite[A significant portion of this document is redacted, including, for some reason, the page numbers.]{cia_future_1960} Satellites showed enormous potential for covering that remaining 85\%, and in doing so, shut down discussion of the ``missile gap'' once and for all.




% The peaceful uses of aerial reconnaissance were formalized at the end of the Cold War, when members of NATO and the Warsaw Pact signed the \enquote{Treaty on Open Skies,} which established guidelines for unarmed aerial surveillance flights in the spirit of open information and de-escalation.\footcite{organization_for_security_and_co-operation_in_europe_treaty_1992} This treaty was almost identical to the one first suggested by Eisenhower in 1955.\footcite{center_for_arms_control_and_non-proliferation_fact_2017} With certain restrictions, the type of flight that altered the course of the Cold War is now officially routine and permissible. In February 1962, Francis Gary Powers walked across a bridge into West Berlin at the same time as Russian Colonel Rudolph Abel, an American prisoner. Powers' father had suggested to the State Department that they might be traded, spy for a spy.\footcite[p.~239]{powers_operation_2004} \enquote{This,} Eisenhower wrote, ``was a tacit admission by Khrushchev that our `outrageous' U-2 pilots have their opposite numbers operating within the borders of this country.''\footcite[p.~558]{eisenhower_waging_1965}



% While I make the claim that both the spy plane and the spy satellite were purposefully connected to an intelligence tradition, it cannot be denied that the first time each one was deployed resulted in very different diplomatic outcomes. The first series of U-2 flights led to diplomatic protests, then a shootdown, then a dramatic summit meeting; the first series of Corona launches led to diplomatic protests, then... nothing. In absolute terms, Corona was dramatically more successful. Eisenhower gained the right to orbit a satellite over the Soviet Union whenever he wanted, and after the Gary Powers incident the U-2 was only used sparingly.

% U-2 overflights could have caused a much more damaging reaction, but their framing helped prevent it from becoming an even larger incident than it already way. The norms defining how states should respond to espionage are ambiguous---the principle that states should not fly planes deep into each others' territory is not. When the U-2 was first developed, it was not at all clear which norm would supersede the other. As it happened, neither did---the interaction of existing norms was unique to both the craft itself and the way that it was marketed by the administration that invented it. The U-2 is not 70\% spy and 30\% plane. Instead, it is a complicated intersection of the two. The Soviet Union---and frankly, any nation over which the US chooses to fly a plane without permission---had the undisputed right to shoot down the American spy plane, which they did, but they also chose not to treat as an act of war, despite Eisenhower's explicit concern that they would.
