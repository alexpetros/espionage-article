\begin{abstract}
\begin{em}
Espionage has long been an accepted part of international relations, but the norms might be changing. Cyberattacks vastly increase the potential blast radius of an intelligence operation, and the old rules of the game are looking increasingly out of date. In this article, I defend the longstanding tolerance of espionage in the international community as a rational calculation of each state's incentives, which are biased towards cooperate. I employ defensive realism, in which states use intelligence as a means to solve the security dilemma, to explain how these norms can persist among states of varying power. Then, I test two past technological advancements that revolutionized intelligence---the spy plane and the spy satellite---to demonstrate how the United States and the Soviet Union purposefully employed espionage norms to diminish the consequences for their use.
\end{em}
\end{abstract}
